\chapter*{Introduction} \label{intro:chapter}
%mbxSTARTIGNORE
\addfakecontentsline{Introduction}
\markboth{INTRODUCTION}{INTRODUCTION}
%mbxENDIGNORE

%%%%%%%%%%%%%%%%%%%%%%%%%%%%%%%%%%%%%%%%%%%%%%%%%%%%%%%%%%%%%%%%%%%%%%%%%%%%%%

\section{Notes about these notes}
\label{notes:section}

\sectionnotes{A section for the instructor.}

This book originated from my class notes for Math 286
at the \href{https://www.math.uiuc.edu/}{University of Illinois at
Urbana-Champaign} (UIUC)
in Fall 2008 and
Spring 2009.
It is a first course on differential equations for engineers.
Using this book, I also taught Math 285 at UIUC\@,
Math 20D at
\href{https://www.math.ucsd.edu/}{University of California, San Diego} (UCSD),
and Math 4233 at 
\href{https://math.okstate.edu/}{Oklahoma State University} (OSU).
Normally these courses are taught with
Edwards and Penney, \emph{Differential
Equations and Boundary Value Problems: Computing and Modeling}~\cite{EP}, or
Boyce and DiPrima's
\emph{Elementary
Differential Equations and Boundary Value Problems}~\cite{BD},
and this book aims to be more or less a drop-in replacement.
Other books I used as sources of information and inspiration
are E.L.\ Ince's classic (and inexpensive)
\emph{Ordinary Differential Equations}~\cite{I},
Stanley Farlow's \emph{Differential Equations and Their
Applications}~\cite{F}, now available from Dover,
Berg and McGregor's
\emph{Elementary Partial Differential Equations}~\cite{BM},
and William Trench's free book
\emph{Elementary
Differential Equations with Boundary Value Problems}~\cite{T}.
See the \hyperref[furtherreading:chapter]{Further Reading} chapter at the end of the book.

\subsection{Organization}

The organization of this book to some degree
requires chapters be done in order.
Later chapters can be dropped.
The dependence of the material covered is roughly:

% If changing make sure to also update figures/chapterdiagram.pdf_t
% That's a hopefully short term hack before I figure out how to do it
% better so that it also gets links and such
%mbxSTARTIGNORE
\begin{equation*}
\begin{tikzcd}[cramped, row sep=small]
& {\text{\hyperref[intro:chapter]{Introduction}}} \arrow[d] \\
{\text{\Appendixref{linalg:appendix}}} \arrow[dd, dotted]
& {\text{\Chapterref{fo:chapter}}} \arrow[d] \\
& {\text{\Chapterref{ho:chapter}}} \arrow[dddr] \arrow[dd] \arrow[dl] \arrow[dr] \\
{\text{\Chapterref{sys:chapter}}} \arrow[dr, dotted] \arrow[d] & &
  {\text{\Chapterref{ps:chapter}}} \\
{\text{\Chapterref{nlin:chapter}}} & {\text{\Chapterref{FS:chapter}}} \arrow[d]
\arrow[dr,dotted] \\
& {\text{\Chapterref{SL:chapter}}}
& {\text{\Chapterref{LT:chapter}}}
\end{tikzcd}
\end{equation*}
%mbxENDIGNORE
%mbxlatex \begin{center}
%mbxlatex \inputpdft{chapterdiagram}
%mbxlatex \end{center}

There are a few references in chapters \ref{FS:chapter} and \ref{SL:chapter}
to \chapterref{sys:chapter} (some linear algebra), but these
references are not essential and can be skimmed over,
so \chapterref{sys:chapter}
can safely be dropped, while still covering
chapters \ref{FS:chapter} and \ref{SL:chapter}.
\Chapterref{LT:chapter} does not depend on 
\chapterref{FS:chapter} except that the
PDE section \ref{laplacepde:section} makes a
few references to
\chapterref{FS:chapter},
although it could in theory be covered
separately.
The more in-depth \appendixref{linalg:appendix} on linear algebra
can replace the short review \sectionref{sec:matrix}
for a course that combines linear algebra and ODE\@.

%\medskip
\subsection{Typical types of courses}

There are several typical types of courses that can be run with the book.
There are the two original courses at UIUC, both cover ODE as well some PDE.
Either, 4 hours-a-week for a semester (Math 286 at UIUC):

\medskip

\noindent
\hyperref[intro:chapter]{Intro.} (\ref{introde:section}),
\chapterref{fo:chapter} (\ref{integralsols:section}--\ref{numer:section}),
\chapterref{ho:chapter},
\chapterref{sys:chapter},
\chapterref{FS:chapter} (\ref{bvp:section}--\ref{dirich:section}),
\chapterref{SL:chapter} (or
\ref{LT:chapter} or \ref{ps:chapter} or \ref{nlin:chapter}).

\medskip

The second course at UIUC is at 3 hours-a-week (Math 285 at UIUC):

\medskip

\noindent
\hyperref[intro:chapter]{Intro.} (\ref{introde:section}),
\chapterref{fo:chapter} (\ref{integralsols:section}--\ref{numer:section}),
\chapterref{ho:chapter},
\chapterref{FS:chapter} (\ref{bvp:section}--\ref{dirich:section}),
(and maybe \chapterref{SL:chapter},
\ref{LT:chapter}, or \ref{ps:chapter}).

\medskip

A semester course at 3 hours a week that doesn't cover either systems or PDE
will cover, beyond the introduction,
%\sectionref{introde:section},
\chapterref{fo:chapter},
\chapterref{ho:chapter},
\chapterref{LT:chapter}, and \chapterref{ps:chapter},
(with sections skipped as above).
On the other hand, a typical course that covers 
systems will probably need to skip Laplace and power series
and cover
%\sectionref{introde:section},
\chapterref{fo:chapter},
\chapterref{ho:chapter},
\chapterref{sys:chapter}, and \chapterref{nlin:chapter}.

\medskip

If sections need to be skipped in the beginning, a good core of the 
sections on single ODE is:
\ref{introde:section},
\ref{integralsols:section}--\ref{intfactor:section},
\ref{auteq:section},
\ref{solinear:section},
\ref{sec:ccsol},
\ref{sec:mv}--\ref{forcedo:section}.

\medskip

The complete book can be covered at a reasonably
fast pace at approximately 76 (FIXME?) lectures
(without \appendixref{linalg:appendix})
or FIXME lectures (with \appendixref{linalg:appendix} replacing
\sectionref{sec:matrix}).
This is not accounting for exams, review,
or time spent in computer lab. % (if using IODE for example).
A two quarter or a two semester course can be easily run with the material.
For example (with some sections perhaps strategically skipped):

\medskip

\noindent
Semester 1:
\hyperref[intro:chapter]{Introduction},
\chapterref{fo:chapter},
\chapterref{ho:chapter},
\chapterref{LT:chapter},
\chapterref{ps:chapter}.
\\
Semester 2: 
\Chapterref{sys:chapter},
\chapterref{nlin:chapter},
\chapterref{FS:chapter},
\chapterref{SL:chapter}.

\medskip

A combined course on ODE with linear algebra can run as:

\medskip

\noindent
\hyperref[intro:chapter]{Introduction},
\chapterref{fo:chapter} (\ref{integralsols:section}--\ref{numer:section}),
\chapterref{ho:chapter},
\appendixref{linalg:appendix},
\chapterref{sys:chapter} (w/o \sectionref{sec:matrix}), (possibly 
\chapterref{nlin:chapter}).

\medskip

The chapter on
Laplace transform (\chapterref{LT:chapter}),
the chapter on Sturm-Liouville (\chapterref{SL:chapter}),
the chapter on power series (\chapterref{ps:chapter}),
and the chapter on nonlinear systems (\chapterref{nlin:chapter}),
are more or less interchangeable, and can be treated as ``topics''.
If \chapterref{nlin:chapter} is covered it may be best to place it right 
after \chapterref{sys:chapter},
and \chapterref{SL:chapter} is best covered right after
\chapterref{FS:chapter}.
If time is short, the first two sections of
\chapterref{ps:chapter} make a reasonable self-contained unit.

%\medskip
\subsection{Computer resources}

There are some interactive SAGE demos at the book webpage:
\url{https://www.jirka.org/diffyqs/#sage}. The PDFs of the
figures used in the book are also on the book webpage.

I taught the UIUC courses using IODE\index{IODE software}
(\url{https://faculty.math.illinois.edu/iode/}).
IODE is a free software package that
works with Matlab (proprietary) or Octave (free software).
Unfortunately IODE is not kept up to date at this point, and may have
trouble running on newer versions of Matlab.
The graphs in the book were made with
the Genius\index{Genius software} software
(see \url{https://www.jirka.org/genius.html}).  I use Genius
in class to show these (and other) graphs.

This book
is available from
\url{https://www.jirka.org/diffyqs/}.  Check there for any possible
updates or errata.  The \LaTeX\ source is also available
for possible modification and customization
at github (\url{https://github.com/jirilebl/diffyqs}).

%\medskip

%\textbf{Acknowlegements:}

\subsection{Acknowlegements}

Firstly, I would like to acknowledge Rick Laugesen.  I used his handwritten
class notes
the first time I taught
Math 286.  My organization of this book through chapter 5,
and the choice of
material covered, is heavily influenced by his notes.  Many
examples and computations are taken from his notes.  I am also heavily
indebted to Rick for all the advice he has given me, not just on teaching
Math 286.
For spotting errors and other suggestions,
I would also like to acknowledge (in no particular order):
John P.\ D'Angelo,
Sean Raleigh, Jessica Robinson, Michael Angelini, Leonardo Gomes, Jeff
Winegar, Ian Simon, Thomas Wicklund, Eliot Brenner, Sean Robinson,
Jannett Susberry, Dana Al-Quadi, Cesar Alvarez, Cem Bagdatlioglu,
Nathan Wong, Alison Shive, Shawn White, Wing Yip Ho, Joanne Shin,
Gladys Cruz, Jonathan Gomez, Janelle Louie, Navid Froutan,
Grace Victorine, Paul Pearson, Jared Teague, Ziad Adwan,
Martin Weilandt, S\"{o}nmez \c{S}ahuto\u{g}lu,
Pete Peterson, Thomas Gresham, Prentiss Hyde, Jai Welch,
Simon Tse, Andrew Browning, James Choi, Dusty Grundmeier,
John Marriott,
Jim Kruidenier,
Barry Conrad,
Wesley Snider,
Colton Koop,
Sarah Morse,
Erik Boczko,
Asif Shakeel,
Chris Peterson,
Nicholas Hu,
Paul Seeburger,
Jonathan McCormick,
and probably others I
have forgotten.
Finally I would like
to acknowledge NSF grants DMS-0900885 and DMS-1362337.


%%%%%%%%%%%%%%%%%%%%%%%%%%%%%%%%%%%%%%%%%%%%%%%%%%%%%%%%%%%%%%%%%%%%%%%%%%%%%%

\sectionnewpage
\section{Introduction to differential equations}
\label{introde:section}

\sectionnotes{more than 1 lecture\EPref{, \S1.1 in \cite{EP}}\BDref{,
chapter 1 in \cite{BD}}}

\subsection{Differential equations}

The laws of physics are generally written down as differential
equations.  Therefore, all of science and engineering use
differential equations to some degree.  Understanding
differential equations is essential to understanding almost anything you will
study in your science and engineering classes.
You can think of mathematics as the language of science, and
differential equations are one of the most important parts of this
language as far as science and engineering are concerned.  As an analogy,
suppose all your classes from now on were given in Swahili.  
It would be important to first learn Swahili, or you would have a very
tough time getting a good grade in your classes.

You saw many
differential equations already without perhaps knowing about it.
And you even solved simple
differential equations when you took calculus.
Let us see an example you may not have seen:
\begin{equation} \label{eq1}
\frac{dx}{dt} + x = 2 \cos t .
\end{equation}
Here $x$ is the \emph{\myindex{dependent variable}} and $t$ is the
\emph{\myindex{independent variable}}.
Equation \eqref{eq1}
is a basic example of a \emph{\myindex{differential equation}}.  In fact, it
is an example of a \emph{\myindex{first order differential equation}}, since
it involves only the first derivative of the dependent variable.  This 
equation arises from Newton's law of cooling where the ambient
temperature oscillates with time.

\subsection{Solutions of differential equations}

Solving the differential equation means finding $x$ in terms of $t$.  That
is, we want to find a function of $t$, which we call $x$, such that when
we plug $x$, $t$, and $\frac{dx}{dt}$ into \eqref{eq1}, the equation holds.
It is
the same idea as it would be for a normal (algebraic) equation of just
$x$ and $t$.  We claim that
\begin{equation*}
x = x(t) = \cos t + \sin t
\end{equation*}
is a \emph{\myindex{solution}}.
How do we check?  We simply plug $x$ into equation \eqref{eq1}!  First we
need to compute $\frac{dx}{dt}$.  We find that $\frac{dx}{dt} = 
-\sin t + \cos t$.  Now let us compute the left hand side
of \eqref{eq1}.
\begin{equation*}
\frac{dx}{dt} + x = 
(-\sin t + \cos t)
+
(\cos t + \sin t)
=
2\cos t .
\end{equation*}
Yay!  We got precisely the right hand side.
But there is more!
We claim
$x = \cos t + \sin t + e^{-t}$ is also
a solution.  Let us try,
\begin{equation*}
\frac{dx}{dt} = -\sin t + \cos t - e^{-t} .
\end{equation*}
Again plugging into the left hand side of \eqref{eq1}
\begin{equation*}
\frac{dx}{dt} + x = 
(-\sin t + \cos t - e^{-t}) +
(\cos t + \sin t + e^{-t})
= 2\cos t .
\end{equation*}
And it works yet again!

So there can be many different solutions.  In fact, for this equation all
solutions can be written in the form
\begin{equation*}
x = \cos t + \sin t + C e^{-t}
\end{equation*}
for some constant $C$.  See \figurevref{intro:plotsfig} for the graph of a
few of these solutions. 
We will see how we find these solutions
a few lectures from now.

\begin{mywrapfig}{3.25in}
\capstart
\diffyincludegraphics{width=3in}{width=4.5in}{intro-plots-alt}
\caption{Few solutions of $\frac{dx}{dt} + x = 2 \cos t$.\label{intro:plotsfig}}
\end{mywrapfig}%

\medskip

It turns out that solving differential equations can be quite hard.  
There is no general method that solves every differential equation.  We will
generally focus on how to get exact formulas for solutions of certain
differential
equations, but we will also spend a little bit of time
on getting approximate solutions.

For most of the course we will look at
\emph{ordinary differential equations\index{ordinary differential equation}} or ODEs\index{ODE}, by which we mean that there
is only one independent variable and derivatives are only with respect to
this one variable.
If there are several independent variables, we will get
\emph{partial differential equations\index{partial differential equation}}
or PDEs\index{PDE}.
We will briefly see these near the
end of the course.

Even for ODEs, which are very well understood, it is not a simple question
of turning a crank to get answers.  
It is important to
know when it is easy to find solutions and how to do so.
Although in real applications you will
leave much of the actual calculations to computers, you
need to understand what they are doing.  It is often necessary
to simplify or transform your equations into something that a computer can
understand and solve.
You may need to make certain assumptions and changes in your
model to achieve this.

To be a successful engineer or scientist, you will be required to solve
problems in your job that you never saw before.  It is important to
learn problem solving techniques, so that you may apply those techniques to
new problems.  A common mistake is to expect to learn some prescription for
solving all the problems you will encounter in your later career.  This
course is no exception.


\subsection{Differential equations in practice}

\begin{mywrapfigsimp}{3.05in}{3.35in}
\noindent
\inputpdft{1-1-fig}
\end{mywrapfigsimp}
So how do we use differential equations in science and engineering?  
First, we have some \emph{\myindex{real-world problem}} we wish
to understand.
We make some simplifying assumptions and create a
\emph{\myindex{mathematical model}}.
That is, we translate the real-world situation into a
set of differential equations.
Then we apply mathematics to get some sort of a
\emph{\myindex{mathematical solution}}.
There is still something left to do.  We have to interpret the results.
We have to figure out what the mathematical solution says about the real-world
problem we started with.

Learning how to formulate the mathematical model and how to interpret the
results is what your physics and engineering classes do.  In this
course we will focus mostly on the mathematical analysis.  Sometimes we will
work with simple real-world examples, so that we have some intuition and
motivation about what we are doing.

Let us look at 
an example of this process.
One of the most basic differential equations
is the standard \emph{\myindex{exponential growth model}}.
Let $P$ denote the population 
of some bacteria on a Petri dish.  We assume that there is enough food
and enough space.  Then the rate of growth of bacteria is proportional
to the population---a large population grows quicker.  Let $t$ denote
time (say in seconds) and $P$ the population.  Our model
is
\begin{equation*}
\frac{dP}{dt} = kP ,
\end{equation*}
for some positive constant $k > 0$.

\begin{example}
Suppose there are 100 bacteria at time 0 and 200 bacteria 10 seconds later.
How many bacteria will there be 1 minute from time 0 (in 60 seconds)?

First we have to solve the equation.  We claim that a solution is given by
\begin{equation*}
P(t) = C e^{kt} ,
\end{equation*}
where $C$ is a constant.  Let us try:
\begin{equation*}
\frac{dP}{dt} = C k e^{kt} = k P .
\end{equation*}
And it really is a solution.

%mbxSTARTIGNORE
\begin{mywrapfig}{3.25in}
\capstart
\diffyincludegraphics{width=3in}{width=4.5in}{intro-plotbact}
\caption{Bacteria growth in the first 60 seconds.\label{intro:plotbactfig}}
\end{mywrapfig}
%mbxENDIGNORE
%
% Make sure to keep the above and the the mbx figure below in sync!
%
OK\@, so what now?  We do not know $C$ and we do not know $k$.  But we know
something.  We know $P(0) = 100$, and we also know 
$P(10) = 200$.  Let us plug these conditions in and see what happens.
\begin{align*}
& 100 = P(0) = C e^{k0} = C ,\\
& 200 = P(10) = 100 \, e^{k10} .
\end{align*}
Therefore, $2 = e^{10k}$ or $\frac{\ln 2}{10} = k \approx 0.069$.
So we know that
\begin{equation*}
P(t) = 100 \, e^{(\ln 2) t / 10} \approx 100 \, e^{0.069 t} .
\end{equation*}
At one minute, $t=60$, the population is $P(60) = 6400$.  See
\figurevref{intro:plotbactfig}.


%mbxlatex \begin{myfig}
%mbxlatex \capstart
%mbxlatex \diffyincludegraphics{width=3in}{width=4.5in}{intro-plotbact}
%mbxlatex \caption{Bacteria growth in the first 60 seconds.\label{intro:plotbactfig}}
%mbxlatex \end{myfig}


Let us talk about the interpretation of the results.  Does our solution
mean that
there must be exactly 6400 bacteria on the plate at 60s?  No!  We made
assumptions that might not be true exactly, just approximately.
If our assumptions are reasonable,
then there will be approximately 6400 bacteria.
Also, in real life $P$ is a
discrete quantity, not a real number.  However, our model has no problem saying
that for example at 61 seconds, $P(61) \approx 6859.35$.
%Obviously there 
%are either 6859 bacteria or 6860 bacteria.
\end{example}

Normally, the $k$ in $P' = kP$ is known,
and we want to solve
the equation for different \emph{initial conditions\index{initial condition}}.
What does that mean?
Take $k=1$ for simplicity.  Now suppose we want to solve the equation
$\frac{dP}{dt} = P$ 
subject to $P(0) = 1000$ (the initial condition).
Then the solution turns out to be (exercise)
\begin{equation*}
P(t) = 1000 \, e^t .
\end{equation*}

We call $P(t) = C e^t$ \emph{the \myindex{general solution}},
as every solution
of the equation can be written in this form for some constant $C$.  You
will need an initial condition to find out what $C$ is, in order to find the
\emph{\myindex{particular solution}} we are looking for.  Generally, when we say
\myquote{particular solution,} we just mean some solution.

\medskip

Let us get to what we will call the four fundamental equations.
These equations appear very often and it is useful to just memorize what
their solutions are.
These solutions
are reasonably easy
to guess by recalling properties of exponentials, sines, and cosines.
They are also simple to check, which is something that you should always do.
There is no need to wonder if you remembered the solution correctly.

\medskip

First such equation is,
\begin{equation*}
\frac{dy}{dx} = k y ,
\end{equation*}
for some constant $k > 0$.
Here $y$ is the dependent and $x$ the independent variable.
The general solution for this equation is
\begin{equation*}
y(x) = C e^{kx} .
\end{equation*}
We saw above that this function is a solution, although we used different
variable names.

\medskip

Next,
\begin{equation*}
\frac{dy}{dx} = -k y ,
\end{equation*}
for some constant $k > 0$.
The general solution for this equation is
\begin{equation*}
y(x) = C e^{-kx} .
\end{equation*}

\begin{exercise}
Check that the $y$ given is really a solution to the equation.
\end{exercise}

Next, take the
\emph{\myindex{second order differential equation}}
\begin{equation*}
\frac{d^2y}{{dx}^2} = -k^2 y ,
\end{equation*}
for some constant $k > 0$.
The general solution for this equation is
\begin{equation*}
y(x) = C_1 \cos(kx) + C_2 \sin(kx) .
\end{equation*}
Note that
because
we have a second order differential equation,
we have two constants in our general solution.

\begin{exercise}
Check that the $y$ given is really a solution to the equation.
\end{exercise}

And finally, take the second order differential equation
\begin{equation*}
\frac{d^2y}{{dx}^2} = k^2 y ,
\end{equation*}
for some constant $k > 0$.
The general solution for this equation is
\begin{equation*}
y(x) = C_1 e^{kx} + C_2 e^{-kx} ,
\end{equation*}
or
\begin{equation*}
y(x) = D_1 \cosh(kx) + D_2 \sinh(kx) .
\end{equation*}

For those that do not know, $\cosh$ and $\sinh$ are defined by
\begin{align*}
\cosh x &= \frac{e^{x} + e^{-x}}{2} , \\
\sinh x &= \frac{e^{x} - e^{-x}}{2} .
\end{align*}
These functions are sometimes easier to
work with than exponentials.  They have some nice familiar
properties such as
$\cosh 0 = 1$, $\sinh 0 = 0$, and $\frac{d}{dx} \cosh x = \sinh x$ (no that is
not a typo)
and $\frac{d}{dx} \sinh x = \cosh x$.

\begin{exercise}
Check that both forms of the $y$ given are
really solutions to the equation.
\end{exercise}

An interesting note about $\cosh$:  The graph of $\cosh$ is the exact shape
of a hanging chain.  This shape is called
a \emph{\myindex{catenary}}.
Contrary to popular belief this is not a
parabola.  If you invert the graph of $\cosh$ it is also the ideal arch for
supporting its own weight.
For example, the gateway arch in Saint Louis is an inverted graph of
$\cosh$---if it were just a parabola it might fall down.  The formula
used in the design is
inscribed inside the arch:
\begin{equation*}
y = -127.7 \; \textrm{ft} \cdot \cosh({x / 127.7  \; \textrm{ft}}) + 757.7 \;
\textrm{ft} .
\end{equation*}


\subsection{Exercises}

\begin{exercise}
Show that $x = e^{4t}$ is a solution to $x'''-12 x'' + 48 x' - 64 x = 0$.
\end{exercise}

\begin{exercise}
Show that $x = e^{t}$ is not a solution to $x'''-12 x'' + 48 x' - 64 x = 0$.
\end{exercise}

\begin{exercise}
Is $y = \sin t$ a solution to ${\left( \frac{dy}{dt} \right)}^2 = 1 - y^2$?
Justify.
\end{exercise}

\begin{exercise}
Let $y'' + 2y' - 8y = 0$.  Now try a solution of the form $y = e^{rx}$ for
some (unknown) constant $r$.  Is this a solution
for some $r$?  If so, find all such $r$.
\end{exercise}

\begin{exercise}
Verify that $x = C e^{-2t}$ is a solution to $x' = -2x$.
Find $C$ to solve for the initial condition $x(0) = 100$.
\end{exercise}

\begin{exercise}
Verify that $x = C_1 e^{-t} + C_2 e^{2t}$ is a solution to $x'' - x' -2 x =
0$.  Find $C_1$ and $C_2$ to solve for the initial conditions $x(0) = 10$
and $x'(0) = 0$.
\end{exercise}

\begin{exercise}
Find a solution to
${(x')}^2 + x^2 = 4$
using your knowledge of derivatives of functions that you
know from basic calculus.
\end{exercise}

\begin{exercise}
Solve:
\begin{tasks}(2)
\task $\dfrac{dA}{dt} = -10 A$,~ $A(0)=5$
\task $\dfrac{dH}{dx} = 3 H$,~ $H(0)=1$
\task $\dfrac{d^2y}{dx^2} = 4 y$,~ $y(0)=0$,~ $y'(0)=1$
\task $\dfrac{d^2x}{dy^2} = -9 x$,~ $x(0)=1$,~ $x'(0)=0$
\end{tasks}
\end{exercise}

\begin{exercise}
Is there a solution to $y' = y$, such that $y(0) = y(1)$?
\end{exercise}

%mbxSTARTIGNORE
\noindent
\emph{Note: Exercises with numbers 101 and higher have solutions in the
back of the book.}
%mbxENDIGNORE

%mbx <p><em>Note: Exercises with numbers 101 and higher have solutions.</em></p>

\setcounter{exercise}{100}

\begin{exercise}
Show that $x = e^{-2t}$ is a solution to $x'' + 4x' + 4x = 0$.
\end{exercise}
\exsol{%
Compute $x' = -2e^{-2t}$ and $x'' = 4e^{-2t}$.  Then
$(4e^{-2t}) + 4 (-2e^{-2t}) + 4 (e^{-2t}) = 0$.
}

\begin{exercise}
Is $y = x^2$ a solution to $x^2y'' - 2y = 0$?  Justify.
\end{exercise}
\exsol{%
Yes.
}

\begin{exercise}
Let $xy'' - y' = 0$.  Try a solution of the form $y = x^r$.  Is this a
solution for some $r$?  If so, find all such $r$.
\end{exercise}
\exsol{%
$y=x^r$ is a solution for $r=0$ and $r=2$.
}


\begin{exercise}
Verify that $x=C_1e^t+C_2$ is a solution to $x''-x' = 0$.  Find $C_1$ and
$C_2$ so that $x$ satisfies $x(0) = 10$ and $x'(0) = 100$.
\end{exercise}
\exsol{%
$C_1 = 100$, $C_2 = -90$
}

\begin{exercise}
Solve $\frac{d\varphi}{ds} = 8 \varphi$ and $\varphi(0) = -9$.
\end{exercise}
\exsol{%
$\varphi = -9 e^{8s}$
}

%%%%%%%%%%%%%%%%%%%%%%%%%%%%%%%%%%%%%%%%%%%%%%%%%%%%%%%%%%%%%%%%%%%%%%%%%%%%%%

\sectionnewpage
\section{Classification of differential equations}
\label{classification:section}

%Perhaps no [EP] ref?
\sectionnotes{less than 1 lecture or left as reading\BDref{, \S1.3 in \cite{BD}}}

There are many types of differential equations and we classify them into
different categories based on their properties.  Let us quickly go over
the most basic classification.  We already saw the distinction
between ordinary and partial differential equations:
\begin{itemize}
\item
\emph{Ordinary differential equations}
\index{Ordinary differential equations}\index{ODE} or (ODE) are
equations where the derivatives are taken with respect to only one variable.
That is, there is only one independent variable.
\item
\emph{Partial differential equations}
\index{Partial differential equations}\index{PDE} or (PDE) are
equations that depend on partial derivatives of several variables.
That is, there are several independent variables.
\end{itemize}

Let us see some examples of ordinary differential equations:
\begin{align*}
& \frac{d y}{dt} = ky , & & \text{(\myindex{Newton's law of cooling})} \\
& m \frac{d^2 x}{dt^2} + c \frac{dx}{dt} + kx = f(t) . & &
\text{(Mechanical vibrations\index{mechanical vibrations})}
\end{align*}
And of partial differential equations:
\begin{align*}
& \frac{\partial y}{\partial t} + c \frac{\partial y}{\partial x} = 0 , & & 
\text{(Transport equation\index{transport equation})} \\
& \frac{\partial u}{\partial t} = \frac{\partial^2 u}{\partial x^2} , & & 
\text{(Heat equation\index{heat equation})} \\
& \frac{\partial^2 u}{\partial t^2} = \frac{\partial^2 u}{\partial x^2} +
\frac{\partial^2 u}{\partial y^2} . & & 
\text{(Wave equation in 2 dimensions\index{wave equation in 2 dimensions})}
\end{align*}

If there are several equations working together we have a so-called
\emph{\myindex{system of differential equations}}.  For example,
\begin{equation*}
y' = x , \qquad x' = y
\end{equation*}
is a simple system of ordinary differential equations.
\myindex{Maxwell's equations} for electromagnetics,
\begin{align*}
& \nabla \cdot \vec{D} = \rho, & & \nabla \cdot \vec{B} = 0 , \\
& \nabla \times \vec{E} = - \frac{\partial \vec{B}}{\partial t}, &
& \nabla \times \vec{H} = \vec{J} + \frac{\partial \vec{D}}{\partial t} ,
\end{align*}
are a system of partial differential equations. 
The divergence operator $\nabla \cdot$ and the
curl operator $\nabla \times$ can be written out in partial derivatives of
the functions involved in the $x$, $y$, and $z$ variables.

\medskip

The next bit of information is the \emph{\myindex{order}} of the
equation (or system).  The order is simply the order of the largest
derivative that appears.  If the highest derivative that appears is
the first derivative, the equation is of first order.  If the highest
derivative that appears is the second derivative, then the equation is of second
order.  For example, Newton's law of cooling above is a first order
equation, while the Mechanical vibrations equation is a second order equation.
The equation governing transversal vibrations in a beam,
\begin{equation*}
a^4 \frac{\partial^4 y}{\partial x^4} + \frac{\partial^2 y}{\partial t^2} = 0,
\end{equation*}
is a fourth order partial differential equation.  It is
fourth order since at least one derivative is the fourth derivative.  It
does not matter that derivatives with respect to $t$ are only second order.

In the first chapter we will start attacking first order ordinary
differential equations, that is, equations of the form $\frac{dy}{dx} = f(x,y)$.
In general, lower order equations are easier to work with and have simpler
behavior, which is why we start with them.

\medskip

We also distinguish how the dependent variables appear in the equation (or
system).  In particular, we say an equation is
\emph{linear}\index{linear equation} if the
dependent variable (or variables) and their derivatives appear linearly,
that is only as first powers, they are not multiplied together, and no other functions of the dependent
variables appear.  In other words, the equation is a sum of terms,
where each term is
some function of the independent variables
or 
some function of the independent variables
multiplied by a dependent variable
or its derivative.
Otherwise the equation is called
\emph{nonlinear}\index{nonlinear equation}.
For example,
an ordinary differential equation is linear if it can be
put into the form
\begin{equation} \label{classification:eqlingen}
a_n(x) \frac{d^n y}{dx^n} + 
a_{n-1}(x) \frac{d^{n-1} y}{dx^{n-1}} + 
\cdots
+
a_{1}(x) \frac{dy}{dx}
+
a_{0}(x) y = b(x) .
\end{equation}
The functions $a_0$, $a_1$, \ldots, $a_n$ are called the
\emph{\myindex{coefficients}}.
The equation is allowed to depend arbitrarily on the independent variables.
So 
\begin{equation} \label{classification:eqlinex}
e^x \frac{d^2 y}{dx^2} + 
\sin(x) \frac{d y}{dx} + 
x^2 y
=
\frac{1}{x}
\end{equation}
is still a linear equation as $y$ and its derivatives only appear linearly.

All the equations and systems given above as examples are linear.  
It may not be immediately obvious for Maxwell's equations unless you write out
the divergence and curl in terms of partial derivatives.  Let us see some
nonlinear equations.  For example \myindex{Burger's equation},
\begin{equation*}
\frac{\partial y}{\partial t} + 
y \frac{\partial y}{\partial x} =
\nu \frac{\partial^2 y}{\partial x^2} ,
\end{equation*}
is a nonlinear second order partial differential equation.  It is nonlinear
because $y$ and $\frac{\partial y}{\partial x}$ are multiplied together.
The equation
\begin{equation} \label{classification:eqnonlinode}
\frac{dx}{dt} = x^2
\end{equation}
is a nonlinear first order differential equation as there is a power of
the dependent variable $x$.

\medskip

A linear equation may further be called \emph{\myindex{homogeneous}}, if
all terms depend on the dependent variable.  That is, if there is no
term that is a function of the independent variables alone.  Otherwise the
equation is called \emph{\myindex{nonhomogeneous}} or
\emph{\myindex{inhomogeneous}}.  For example,
Newton's law of cooling, Transport equation, Wave equation, above are homogeneous,
while Mechanical vibrations equation above is nonhomogeneous.
A homogeneous linear ODE can be put into the form
\begin{equation*}
a_n(x) \frac{d^n y}{dx^n} + 
a_{n-1}(x) \frac{d^{n-1} y}{dx^{n-1}} + 
\cdots
+
a_{1}(x) \frac{dy}{dx}
+
a_{0}(x) y = 0 .
\end{equation*}
Compare to \eqref{classification:eqlingen} and notice there is no
function $b(x)$.

\medskip

If the coefficients of a linear equation are actually constant functions,
then the equation is said to have \emph{\myindex{constant coefficients}}.
The coefficients are the functions multiplying the dependent
variable(s) or one of its derivatives, not the function standing alone.
That is, a constant coefficient ODE is
\begin{equation*}
a_n \frac{d^n y}{dx^n} + 
a_{n-1} \frac{d^{n-1} y}{dx^{n-1}} + 
\cdots
+
a_{1} \frac{dy}{dx}
+
a_{0} y = b(x) ,
\end{equation*}
where $a_0, a_1, \ldots, a_n$ are all constants, but $b$ may depend on 
the independent variable $x$.  The Mechanical vibrations equation
above is constant coefficient nonhomogeneous second order ODE\@.
Same nomenclature applies to PDEs, so the Transport equation,
Heat equation and Wave equation are all examples of constant coefficient
linear PDEs.

\medskip

Finally, an equation (or system) is called \emph{\myindex{autonomous}}
if the equation does not depend on the independent variable.
Usually here we only consider ordinary differential equations and the
independent variable is then thought of as time.  Autonomous equation
means an equation that does not change with time.
For example, Newton's law of cooling is autonomous, so is equation
\eqref{classification:eqnonlinode}.  On the other hand, Mechanical
vibrations or 
\eqref{classification:eqlinex} are not autonomous.

\subsection{Exercises}

\begin{exercise}
Classify the following equations.  Are they ODE or PDE?  Is it an equation
or a system?  What is the order?  Is it linear or nonlinear, and if it is
linear, is it homogeneous, constant coefficient?  If it is an ODE\@, is it
autonomous?
\begin{tasks}(2)
\task $\displaystyle \sin(t) \frac{d^2 x}{dt^2} + \cos(t) x = t^2$
\task $\displaystyle \frac{\partial u}{\partial x} + 3 \frac{\partial u}{\partial y} = xy$
\task $\displaystyle y''+3y+5x=0, \quad x''+x-y=0$
\task $\displaystyle \frac{\partial^2 u}{\partial t^2} + u\frac{\partial^2 u}{\partial s^2} =
0$
\task $\displaystyle x''+tx^2=t$
\task $\displaystyle \frac{d^4 x}{dt^4} = 0$
\end{tasks}
\end{exercise}

\begin{exercise}
If $\vec{u} = (u_1,u_2,u_3)$ is a vector, we have the divergence
$\nabla \cdot \vec{u} =
\frac{\partial u_1}{\partial x} +
\frac{\partial u_2}{\partial y} +
\frac{\partial u_3}{\partial z}$ and curl
$\nabla \times \vec{u} =
\Bigl(
\frac{\partial u_3}{\partial y} - \frac{\partial u_2}{\partial z} , ~
\frac{\partial u_1}{\partial z} - \frac{\partial u_3}{\partial x} , ~
\frac{\partial u_2}{\partial x} - \frac{\partial u_1}{\partial y} \Bigr)$.
Notice that curl of a vector is still a vector.  Write out Maxwell's
equations in terms of partial derivatives and classify the system.
\end{exercise}

\begin{exercise}
Suppose $F$ is a linear function, that is,
$F(x,y) = ax+by$ for constants $a$ and $b$.  What is the
classification of equations of the form $F(y',y) = 0$.
\end{exercise}

\begin{exercise}
Write down an explicit example of a third order, linear, nonconstant coefficient,
nonautonomous, nonhomogeneous system of two ODE such that every derivative
that could appear, does appear.
\end{exercise}

\setcounter{exercise}{100}

\pagebreak[2]
\begin{exercise}
Classify the following equations.  Are they ODE or PDE?  Is it an equation
or a system?  What is the order?  Is it linear or nonlinear, and if it is
linear, is it homogeneous, constant coefficient?  If it is an ODE\@, is it
autonomous?
\begin{tasks}(2)
\task $\displaystyle \frac{\partial^2 v}{\partial x^2} + 3 \frac{\partial^2
v}{\partial y^2} = \sin(x)$
\task $\displaystyle \frac{d x}{dt} + \cos(t) x = t^2+t+1$
\task $\displaystyle \frac{d^7 F}{dx^7} = 3F(x)$
\task $\displaystyle y''+8y'=1$
\task $\displaystyle x''+tyx'=0, \quad y''+txy = 0$
\task $\displaystyle \frac{\partial u}{\partial t} = \frac{\partial^2 u}{\partial s^2} + u^2$
\end{tasks}
\end{exercise}
\exsol{%
a) 
PDE\@, equation, second order, linear, nonhomogeneous, constant coefficient.\\
b) 
ODE\@, equation, first order, linear, nonhomogeneous, not constant coefficient, not autonomous.\\
c) 
ODE\@, equation, seventh order, linear, homogeneous, constant coefficient, autonomous.\\
d) 
ODE\@, equation, second order, linear, nonhomogeneous, constant coefficient, autonomous.\\
e) 
ODE\@, system, second order, nonlinear.\\
f) 
PDE\@, equation, second order, nonlinear.
}

\begin{exercise}
Write down the general \emph{zero}th order linear ordinary differential
equation.  Write down the general solution.
\end{exercise}
\exsol{%
equation: $a(x) y = b(x)$, solution: $y = \frac{b(x)}{a(x)}$.
}

