\chapter{The Laplace transform} \label{LT:chapter}

%%%%%%%%%%%%%%%%%%%%%%%%%%%%%%%%%%%%%%%%%%%%%%%%%%%%%%%%%%%%%%%%%%%%%%%%%%%%%%

\section{The Laplace transform}
\label{laplace:section}

%mbxINTROSUBSECTION

\sectionnotes{1.5--2 lectures\EPref{, \S10.1 in \cite{EP}}\BDref{,
\S6.1 and parts of \S6.2 in \cite{BD}}}

\subsection{The transform}

In this chapter, we will discuss the Laplace transform%
\footnote{Just like the Laplace equation and the Laplacian, the Laplace
transform is also named after 
\href{https://en.wikipedia.org/wiki/Laplace}{Pierre-Simon, marquis de Laplace}
(1749--1827).}.
The Laplace transform
is a very efficient method to solve certain ODE or PDE problems.
The transform takes a differential equation and turns it into
an algebraic equation.  If the algebraic equation can be solved, applying the
inverse transform gives us our desired solution.
The Laplace transform also has applications in
the analysis of 
electrical circuits, NMR spectroscopy, signal processing, and elsewhere.
Finally,
understanding the Laplace
transform will also help with understanding the related Fourier transform,
which, however, requires more
understanding of complex numbers.  We will not cover the Fourier transform.

The Laplace transform also gives a lot of insight into the nature of the
equations we are dealing with.  It can be seen as converting between the time
and the frequency domain.  For example, take the standard equation
\begin{equation*}
m x''(t) + c x'(t) + k x(t) = f(t) .
\end{equation*}
We can think of $t$ as time and $f(t)$ as incoming signal.  The Laplace
transform will convert the equation from a differential equation in time to
an algebraic (no derivatives) equation, where the new independent variable
$s$ is the frequency.\footnote{Really, it is the ``frequency'' in terms
of complex numbers, but we digress.}

We can think of the \emph{\myindex{Laplace transform}} as a black box.  It
eats functions and spits out functions in a new variable.  We write
$\mathcal{L} \bigl\{ f(t) \bigr\} = F(s)$ for the Laplace transform of $f(t)$.
It is common to write lower case letters for
functions in the time domain and upper case letters for functions in the
frequency domain.  We use the same letter to denote that one function
is the Laplace transform of the other.  For example, $F(s)$ is the Laplace
transform of $f(t)$.  Let us define the transform.
\begin{equation*}
\mathcal{L} \bigl\{ f(t) \bigr\} =
F(s) \overset{\text{def}}{=} \int_0^\infty e^{-st} f(t) \, dt .
\end{equation*}
We note that we are only considering $t \geq 0$ in the transform.  Of course,
if we think of $t$ as time, there is no problem, we are generally interested in
finding out what will happen in the future (Laplace transform is one place
where it is safe to ignore the past).  Let us compute some simple
transforms.

\begin{example}
Suppose $f(t) = 1$, then
\begin{equation*}
\mathcal{L} \{1\} = \int_0^\infty e^{-st} \, dt
=
\left[ \frac{e^{-st}}{-s} \right]_{t=0}^\infty
=
\lim_{h\to\infty}
\left[ \frac{e^{-st}}{-s} \right]_{t=0}^h
=
\lim_{h\to\infty}
\left( \frac{e^{-sh}}{-s} - \frac{1}{-s} \right)
= \frac{1}{s} .
\end{equation*}
The limit (the improper integral) only exists if $s > 0$.  So 
$\mathcal{L} \{1\}$ is only defined for $s > 0$.
\end{example}

\begin{example}
Suppose $f(t) = e^{-at}$, then
\begin{equation*}
\mathcal{L} \bigl\{e^{-at}\bigr\}
= \int_0^\infty e^{-st} e^{-at} \, dt
= \int_0^\infty e^{-(s+a)t} \, dt
=
\left[ \frac{e^{-(s+a)t}}{-(s+a)} \right]_{t=0}^\infty
= \frac{1}{s+a} .
\end{equation*}
The limit only exists if $s+a > 0$.  So 
$\mathcal{L} \bigl\{e^{-at}\bigr\}$ is only defined for $s+a > 0$.
\end{example}

\begin{example}
Suppose $f(t) = t$, then using integration by parts
\begin{equation*}
\begin{split}
\mathcal{L} \{t\}
& = \int_0^\infty e^{-st} t \, dt \\
& =
\left[ \frac{-te^{-st}}{s} \right]_{t=0}^\infty
+
\frac{1}{s}
\int_0^\infty e^{-st} \,dt \\
& =
0
+
\frac{1}{s}
\left[ \frac{e^{-st}}{-s} \right]_{t=0}^\infty \\
& =
\frac{1}{s^2} .
\end{split}
\end{equation*}
Again, the limit only exists if $s > 0$.
\end{example}

\begin{example}
A common function is the \emph{\myindex{unit step function}}, which is
sometimes called the \emph{\myindex{Heaviside function}}%
\footnote{The function is named after the English mathematician, engineer, and
physicist
\href{https://en.wikipedia.org/wiki/Heaviside}{Oliver Heaviside}
(1850--1925).  Only
by coincidence is the function \myquote{heavy} on \myquote{one side.}}.
This function is generally given as
\begin{equation*}
u(t) = 
\begin{cases}
0 & \text{if } \; t < 0 , \\
1 & \text{if } \; t \geq 0 .
\end{cases}
\end{equation*}
%Note that some authors prefer to define $u(0) = \frac{1}{2}$.
Let us find the Laplace transform of $u(t-a)$, where $a \geq 0$
is some constant.
That is, it is the function that is $0$ for $t < a$ and $1$ for $t \geq a$.
\begin{equation*}
\mathcal{L} \bigl\{ u(t-a) \bigr\}
=
\int_0^{\infty} e^{-st} u(t-a) \, dt
=
\int_a^{\infty} e^{-st} \, dt
=
\left[ \frac{e^{-st}}{-s} \right]_{t=a}^\infty \\
=
\frac{e^{-as}}{s} ,
\end{equation*}
where of course $s > 0$ (and $a \geq 0$ as we said before).
\end{example}

By applying similar procedures, we can compute the transforms of many
elementary functions.  Many basic transforms are listed in
\tablevref{lt:table}.

\begin{table}[h!t]
\mybeginframe
\capstart
\begin{center}
\begin{tabular}{@{}lllll@{}}
\toprule
$f(t)$ & $\mathcal{L} \bigl\{ f(t) \bigr\}$ & $\qquad\quad$ &
$f(t)$ & $\mathcal{L} \bigl\{ f(t) \bigr\}$ \\
\midrule
$C$ & $\frac{C}{s}$
& &
$\sin (\omega t)$ & $\frac{\omega}{s^2+\omega^2}$
\\[4pt]
$t$ & $\frac{1}{s^2}$
& &
$\cos (\omega t)$ & $\frac{s}{s^2+\omega^2}$
\\[4pt]
$t^2$ & $\frac{2}{s^3}$
& &
$\sinh (\omega t)$ & $\frac{\omega}{s^2-\omega^2}$
\\[4pt]
$t^3$ & $\frac{6}{s^4}$
& &
$\cosh (\omega t)$ & $\frac{s}{s^2-\omega^2}$
\\[4pt]
$t^n$ & $\frac{n!}{s^{n+1}}$
& &
$u(t-a)$ & $\frac{e^{-as}}{s}$
\\[4pt]
$e^{-at}$ & $\frac{1}{s+a}$
& & &
\\[4pt]
\bottomrule
\end{tabular}
\end{center}
\caption{Some Laplace transforms ($C$, $\omega$, and $a$ are
constants).\label{lt:table}}
\myendframe
\end{table}

\begin{exercise}
Verify \tablevref{lt:table}.
\end{exercise}

Since the transform is defined by an integral, we can use the linearity
properties of the integral.  For example, suppose $C$ is a constant, then
\begin{equation*}
\mathcal{L} \bigl\{ C f(t) \bigr\} =
\int_0^\infty e^{-st} C f(t) \,dt =
C \int_0^\infty e^{-st} f(t) \,dt =
C \mathcal{L} \bigl\{ f(t) \bigr\} .
\end{equation*}
So we can \myquote{pull out} a constant out of the transform.
Similarly, with addition.
As linearity is important, we state it as a theorem.

\begin{theorem}[Linearity of the Laplace transform]
\index{linearity of the Laplace transform}
Suppose that $A$, $B$, and $C$ are constants, then
\begin{equation*}
\mybxbg{~~
\mathcal{L} \bigl\{ A f(t) + B g(t) \bigr\} =
A \mathcal{L} \bigl\{ f(t) \bigr\} +
B \mathcal{L} \bigl\{ g(t) \bigr\} ,
~~}
\end{equation*}
and in particular,
\begin{equation*}
\mathcal{L} \bigl\{ C f(t) \bigr\} =
C \mathcal{L} \bigl\{ f(t) \bigr\} .
\end{equation*}
\end{theorem}

\begin{exercise}
Verify the theorem.  That is, show that
$\mathcal{L} \bigl\{ A f(t) + B g(t) \bigr\} =
A \mathcal{L} \bigl\{ f(t) \bigr\} +
B \mathcal{L} \bigl\{ g(t) \bigr\}$.
\end{exercise}

These rules together with \tablevref{lt:table} make it easy to find
the Laplace transform of a whole lot of functions already.
For example:
\[
\mathcal{L}\bigl\{
2 + 5t + 9e^{-2t}
\bigr\}
=
\mathcal{L}\bigl\{
2
\bigr\}
+5
\mathcal{L}\bigl\{
t
\bigr\}
+
9
\mathcal{L}\bigl\{
e^{-2t}
\bigr\}
=
\frac{2}{s} + \frac{5}{s^2} + \frac{9}{s+2} .
\]
Be careful!
The Laplace transform of a product
is \emph{not} the product of the transforms.  In general 
\begin{equation*}
\mathcal{L} \bigl\{ f(t) g(t) \bigr\} \not=
\mathcal{L} \bigl\{ f(t) \bigr\}
\mathcal{L} \bigl\{ g(t) \bigr\} .
\end{equation*}
Moreover, not all functions have a Laplace transform.  For
example, the function $\frac{1}{t}$ does not have a Laplace transform as the
integral diverges for all $s$.  Similarly,
$\tan t$ or $e^{t^2}$ do not have Laplace transforms.

\subsection{Existence and uniqueness}

When does the Laplace transform exist?  A function $f(t)$ is of
\emph{\myindex{exponential order}} as $t$ goes to infinity if
\begin{equation*}
\lvert f(t) \rvert \leq M e^{ct} ,
\end{equation*}
for some constants $M$ and $c$, for
sufficiently large $t$ (say for all $t > t_0$ for some $t_0$).  The simplest
way to check this condition is to try and compute
\begin{equation*}
\lim_{t\to \infty} \frac{f(t)}{e^{ct}} .
\end{equation*}
If the limit exists and is finite (usually zero), then $f(t)$ is of
exponential order.

\begin{exercise}
Use L'Hopital's rule from calculus to show that a polynomial is of
exponential order.  Hint: Note that a sum of two exponential order functions
is also of exponential order.  Then show that $t^n$ is of exponential order
for any $n$.
\end{exercise}

For an exponential order function we have existence and uniqueness of the
Laplace transform.

\begin{theorem}[Existence]
Let $f(t)$ be continuous on the interval $[0,\infty)$ and of exponential order for a certain
constant $c$.  Then $F(s) = \mathcal{L} \bigl\{ f(t) \bigr\}$ is defined for
all $s > c$.
\end{theorem}

The existence is not difficult to see.  Let $f(t)$ be of exponential order,
that is $\lvert f(t) \rvert \leq M e^{ct}$ for all $t > 0$ (for simplicity $t_0 = 0$).
Let $s > c$, or in other words $(s-c) > 0$.
By the comparison theorem from calculus, the improper integral defining
$\mathcal{L} \bigl\{ f(t) \bigr\}$ exists because the following integral exists
\begin{equation*}
\int_0^\infty e^{-st} ( M e^{ct} ) \,dt
=
M \int_0^\infty e^{-(s-c)t} \,dt = M \left[ \frac{e^{-(s-c)t}}{-(s-c)}
\right]_{t=0}^\infty = \frac{M}{s-c} .
\end{equation*}

The transform also exists for some other functions
that are not of exponential
order, but that will not be relevant to us.
Before dealing with uniqueness,
we note that for functions of exponential order, their
Laplace transform decays at infinity:
\begin{equation*}
\lim_{s\to\infty} F(s) = 0 .
\end{equation*}

\begin{theorem}[Uniqueness] \label{lt:uniqthm}
Let $f(t)$ and $g(t)$ be continuous and of exponential order.
Suppose that there exists a constant $C$,
such that $F(s) = G(s)$ for all $s > C$.
Then $f(t) = g(t)$ for all $t \geq 0$.
\end{theorem}

Both theorems hold for piecewise continuous functions as well.
Recall that piecewise continuous means that the function
is continuous except perhaps at a discrete set of points, where it has jump
discontinuities like the Heaviside function.  Uniqueness, however, does
not \myquote{see} values at the discontinuities.  So we can only conclude that
$f(t) = g(t)$ outside of discontinuities.  For example, the unit step
function is sometimes defined using $u(0) = \nicefrac{1}{2}$.  This new
step function, however, has the exact same Laplace transform
as the one we defined earlier, where $u(0) = 1$.

\subsection{The inverse transform}

As we said, the Laplace transform will allow us to convert a differential
equation into an algebraic equation.  Once we solve the
algebraic equation in the frequency domain, we will want to get back to the
time domain, as that is what we are interested in.
Given a function
$F(s)$, we wish to find a function
$f(t)$ such that $\mathcal{L} \bigl\{ f(t) \bigr\} = F(s)$.
\thmref{lt:uniqthm} says that the solution $f(t)$ is unique.
So we can without fear make the following definition.

Suppose $F(s) = \mathcal{L} \bigl\{ f(t) \bigr\}$ for some function $f(t)$.
Define the
\emph{\myindex{inverse Laplace transform}} as
\begin{equation*}
{\mathcal{L}}^{-1} \bigl\{ F(s) \bigr\} \overset{\text{def}}{=} f(t) .
\end{equation*}
There is an integral formula for the inverse, but it is not as simple
as the transform itself---it requires complex numbers and path integrals.
For us it will
suffice to
compute the inverse using \tablevref{lt:table}.

\begin{example}
Take
$F(s) = \frac{1}{s+1}$.  Find the inverse Laplace transform.

We look at the table to find
\begin{equation*}
{\mathcal{L}}^{-1} \left\{ \frac{1}{s+1} \right\} = 
e^{-t} .
\end{equation*}
\end{example}

As the Laplace transform is linear, the inverse Laplace
transform is also linear.  That is,
\begin{equation*}
{\mathcal{L}}^{-1} \bigl\{ A F(s) + B G(s) \bigr\} =
A {\mathcal{L}}^{-1} \bigl\{ F(s) \bigr\} +
B {\mathcal{L}}^{-1} \bigl\{ G(s) \bigr\} .
\end{equation*}
Of course, we also have
${\mathcal{L}}^{-1} \bigl\{ A F(s) \bigr\} =
A {\mathcal{L}}^{-1} \bigl\{ F(s) \bigr\}$.
%Let us demonstrate how linearity can be used.

\begin{example}
Take
$F(s) = \frac{s^2+s+1}{s^3+s}$.  Find the inverse Laplace transform.

First we use the \emph{\myindex{method of partial fractions}} to write $F$ in a form where
we can use \tablevref{lt:table}.  We factor the denominator as
$s(s^2+1)$ and write
\begin{equation*}
\frac{s^2+s+1}{s^3+s}
=
\frac{A}{s} + 
\frac{Bs+C}{s^2+1} .
\end{equation*}
Putting the right-hand side over a common
denominator and equating the numerators we get
$A(s^2+1) + s(Bs+C) = s^2+s+1$.  Expanding and equating coefficients,
we obtain $A+B = 1$, $C=1$, $A=1$,
and thus $B=0$.  In
other words,
\begin{equation*}
F(s) =
\frac{s^2+s+1}{s^3+s}
=
\frac{1}{s} +
\frac{1}{s^2+1} .
\end{equation*}
By linearity of the inverse Laplace transform, we get 
\begin{equation*}
{\mathcal{L}}^{-1} \left\{ 
\frac{s^2+s+1}{s^3+s} \right\}
=
{\mathcal{L}}^{-1} \left\{ 
\frac{1}{s} \right\} 
+
{\mathcal{L}}^{-1} \left\{ 
\frac{1}{s^2+1} \right\}
=
1 + 
\sin t .
\end{equation*}
\end{example}

Another useful property is the 
so-called \emph{\myindex{shifting property}} or
the \emph{\myindex{first shifting property}}
\begin{equation*}
\mybxbg{~~
\mathcal{L} \bigl\{ e^{-at} f(t) \bigr\} = F(s+a) ,
~~}
\end{equation*}
where $F(s)$ is the Laplace transform of $f(t)$.

\begin{exercise}
Derive the first shifting property
from the definition of the Laplace transform.
\end{exercise}

The shifting property can be used, for example, when the denominator is a
more complicated quadratic that may come up in the method of partial
fractions.  We complete the square and write such quadratics as ${(s+a)}^2+b$
and then use the shifting property.

\begin{example}
Find
${\mathcal{L}}^{-1} \left\{ \frac{1}{s^2+4s+8} \right\}$.

First, we complete the square to make the denominator ${(s+2)}^2+4$.  
We find
\begin{equation*}
{\mathcal{L}}^{-1} \left\{ \frac{1}{s^2+4} \right\}
=
\frac{1}{2}
{\mathcal{L}}^{-1} \left\{ \frac{2}{s^2+2^2} \right\}
=
\frac{1}{2} \sin (2t) .
\end{equation*}
Finally, we put it all together with the shifting property to find
\begin{equation*}
{\mathcal{L}}^{-1} \left\{ \frac{1}{s^2+4s+8} \right\} = 
{\mathcal{L}}^{-1} \left\{ \frac{1}{{(s+2)}^2+4} \right\}
=
\frac{1}{2}\,e^{-2t} \sin (2t) .
\end{equation*}
\end{example}

Often, we want to be able to apply the inverse Laplace transform to
rational functions, that is, functions of the form
\begin{equation*}
\frac{F(s)}{G(s)}
\end{equation*}
where $F(s)$ and $G(s)$ are polynomials.  If $\frac{F(s)}{G(s)}$ is
the Laplace transform of an exponential order function,
it goes to zero as $s \to \infty$, and so the degree of $F(s)$
must be smaller than that of $G(s)$.  Such rational functions
are called \emph{proper rational functions\index{proper rational function}}
and we can always apply the method of partial fractions without polynomial
division.
Of course, we still need to be able to factor the denominator into
linear and quadratic terms, which involves finding the roots of the
denominator.

\subsection{Exercises}

\begin{exercise}
Find the Laplace transform of $3+t^5+\sin (\pi t)$.
\end{exercise}

\begin{exercise}
Find the Laplace transform of $a+bt+ct^2$ for some constants $a$, $b$, and
$c$.
\end{exercise}

\begin{exercise}
Find the Laplace transform of $A \cos (\omega t) + B \sin (\omega t)$.
\end{exercise}

\begin{exercise}
Find the Laplace transform of $\cos^2 (\omega t)$.
\end{exercise}

\begin{exercise}
Find the inverse Laplace transform of $\frac{4}{s^2-9}$.
\end{exercise}

\begin{exercise}
Find the inverse Laplace transform of $\frac{2s}{s^2-1}$.
\end{exercise}

\begin{exercise}
Find the inverse Laplace transform of $\frac{1}{{(s-1)}^2(s+1)}$.
\end{exercise}

\begin{exercise}
Find the Laplace transform of $f(t) =
\begin{cases}
t & \text{if } \; t \geq 1, \\
0 & \text{if } \; t < 1.
\end{cases}$
\end{exercise}

\begin{exercise}
Find the inverse Laplace transform of $\frac{s}{(s^2+s+2)(s+4)}$.
\end{exercise}

\begin{exercise}
Find the Laplace transform of $\sin\bigl(\omega (t-a)\bigr)$.
\end{exercise}

\begin{exercise}
Find the Laplace transform of $t\sin(\omega t)$.  Hint: Several integrations
by parts.
\end{exercise}

\setcounter{exercise}{100}

\begin{exercise}
Find the Laplace transform of $4{(t+1)}^2$.
\end{exercise}
\exsol{%
$\frac{8}{s^3} + \frac{8}{s^2} + \frac{4}{s}$
}

\begin{exercise}
Find the inverse Laplace transform of $\frac{8}{s^3(s+2)}$.
\end{exercise}
\exsol{%
$2t^2-2t+1-e^{-2t}$
}

\begin{exercise}
Find the Laplace transform of $te^{-t}$.  Hint: Integrate by parts.
\end{exercise}
\exsol{%
$\frac{1}{{(s+1)}^2}$
}

\begin{exercise}
Find the Laplace transform of $\sin(t)e^{-t}$.  Hint: Integrate by parts.
\end{exercise}
\exsol{%
$\frac{1}{s^2+2s+2}$
}


%%%%%%%%%%%%%%%%%%%%%%%%%%%%%%%%%%%%%%%%%%%%%%%%%%%%%%%%%%%%%%%%%%%%%%%%%%%%%%

\sectionnewpage
\section{Transforms of derivatives and ODEs}
\label{transformsofders:section}

%mbxINTROSUBSECTION

\sectionnotes{2 lectures\EPref{, \S7.2--7.3 in \cite{EP}}\BDref{,
\S6.2 and \S6.3 in \cite{BD}}}

\subsection{Transforms of derivatives}

Let us see how the Laplace transform is used for differential equations.
First we find the Laplace transform of the derivative of a function.
Suppose $g(t)$ is a differentiable function
of exponential order, that is, $\lvert g(t) \rvert \leq M e^{ct}$ for some
$M$ and $c$.  So $\mathcal{L} \bigl\{ g(t) \bigr\}$ exists, and what is more,
$\lim_{t\to\infty} e^{-st}g(t) = 0$ when $s > c$.  Then
\begin{equation*}
\mathcal{L} \bigl\{ g'(t) \bigr\}
=
\int_0^\infty
e^{-st}
g'(t) \,dt
=
\Bigl[e^{-st} g(t) \Bigr]_{t=0}^\infty
-
\int_0^\infty
(-s)\,
e^{-st}
g(t) \,dt
=
-g(0) + s \mathcal{L} \bigl\{ g(t) \bigr\} .
\end{equation*}
We repeat this procedure for higher derivatives.
The results are
listed in \tablevref{ltd:table}.  The procedure also works for continuous piecewise
smooth functions, that is, functions that are continuous with a
piecewise continuous derivative.
%The fact that the function is of
%exponential order is used to show that the limits appearing above 
%exist.  We will not worry much about this fact.

\begin{table}[h!t]
\mybeginframe
\capstart
\begin{center}
\begin{tabular}{@{}ll@{}}
\toprule
$f(t)$ & $\mathcal{L} \bigl\{ f(t) \bigr\} = F(s)$ \\
\midrule
$g'(t)$ & $sG(s)-g(0)$ \\[4pt]
$g''(t)$ & $s^2G(s)-sg(0)-g'(0)$ \\[4pt]
$g'''(t)$ & $s^3G(s)-s^2g(0)-sg'(0)-g''(0)$ \\[4pt]
\bottomrule
\end{tabular}
\end{center}
\caption{Laplace transforms of derivatives
($G(s) = \mathcal{L} \bigl\{ g(t) \bigr\}$
as usual).\label{ltd:table}}
\myendframe
\end{table}

\begin{exercise}
Verify \tablevref{ltd:table}.
\end{exercise}

\subsection{Solving ODEs with the Laplace transform}

Notice that the Laplace transform turns differentiation into
multiplication by $s$.
It is this property that makes it useful to apply 
the transform to differential equations.

\begin{example}
Take the equation
\begin{equation*}
x''(t) + x(t) = \cos (2t), \quad x(0) = 0, \quad x'(0) = 1 .
\end{equation*}
We will take the Laplace transform of both sides.
By $X(s)$ we will, as usual, denote the Laplace transform of
$x(t)$.
\begin{align*}
\mathcal{L} \bigl\{ x''(t) + x(t) \bigr\} & = \mathcal{L} \bigl\{ \cos (2t) \bigr\} , \\
s^2 X(s) -sx(0)-x'(0) + X(s) & = \frac{s}{s^2 + 4} .
\end{align*}
We plug in the initial conditions now---making computations more
streamlined---to find
\begin{equation*}
s^2 X(s) -1 + X(s) = \frac{s}{s^2 + 4} .
\end{equation*}
We solve for $X(s)$,
\begin{equation*}
X(s) = \frac{s}{(s^2+1)(s^2 + 4)} + \frac{1}{s^2+1} .
\end{equation*}
We use partial fractions (exercise) to write
\begin{equation*}
X(s) =\frac{1}{3} \, \frac{s}{s^2+1} - 
\frac{1}{3}\, \frac{s}{s^2+4} + \frac{1}{s^2+1} .
\end{equation*}
Now take the inverse Laplace transform to obtain
\begin{equation*}
x(t) =\frac{1}{3}  \cos (t) -
\frac{1}{3} \cos (2t) + \sin (t) .
\end{equation*}
\end{example}

The procedure for linear constant coefficient equations is as follows:
Take an ordinary differential
equation in the time variable $t$.  Apply the Laplace transform
to transform the equation into an algebraic (non differential) equation in
the frequency domain.  All the $x(t)$, $x'(t)$, $x''(t)$, and so on, will
be converted to $X(s)$, $sX(s) - x(0)$, $s^2X(s) - sx(0) - x'(0)$,
and so on.
Solve the equation for $X(s)$.
Then taking the inverse transform, if possible, find $x(t)$.

It should be noted that since not every function has a Laplace transform,
not every equation can be solved in this manner.  Also if the equation
is not a linear constant coefficient ODE\@,
then by applying the Laplace transform we may not
obtain an algebraic equation.

\subsection{Using the Heaviside function}

Before we move on to more general equations
than those we could solve before,
we want to consider the Heaviside function.  See \figurevref{lt:heavisidefig}
for the graph.
\begin{equation*}
u(t) =
\begin{cases}
0 & \text{if } \; t < 0 , \\ 
1 & \text{if } \; t \geq 0 .
\end{cases}
\end{equation*}

\begin{myfig}
\capstart
\diffyincludegraphics{width=3in}{width=4.5in}{lt-heaviside}
\caption{Plot of the Heaviside (unit step) function
$u(t)$.\label{lt:heavisidefig}}
\end{myfig}

The Heaviside function is useful for
putting other functions together or cutting functions off.  Usually we use
$u(t-a)$ for some constant $a$;  we just shift the graph to the
right by $a$.  That is, $u(t-a) = 0$ when $t < a$ and $u(t-a) = 1$
when $t \geq a$.  For example, suppose $f(t)$ is a \myquote{signal},
and we started receiving $\sin t$ at time $t=\pi$.
The function $f(t)$ is then defined as
\begin{equation*}
f(t) =
\begin{cases}
0 & \text{if } \; t < \pi , \\ 
\sin t & \text{if } \; t \geq \pi .
\end{cases}
\end{equation*}
Using the Heaviside function, $f(t)$ can
be written as
\begin{equation*}
f(t) = u(t - \pi) \, \sin t .
\end{equation*}
Similarly, the step function that is $1$ on the interval $[1,2)$ and $0$
elsewhere is written as
\begin{equation*}
u(t - 1) - u(t-2) .
\end{equation*}
With the Heaviside function we can express functions defined piecewise.
If $f(t) = t$ when $t$ is in $[0,1]$,
$f(t) = -t+2$
when $t$ is in $[1,2]$, and $f(t) = 0$ otherwise, then you write
\begin{equation*}
f(t) = t \, \bigl( u(t) - u(t-1) \bigr) + 
(-t+2) \, \bigl( u(t - 1) - u(t-2) \bigr) .
\end{equation*}

How does the Heaviside function interact with the Laplace
transform?  We saw that
\begin{equation*}
\mathcal{L} \bigl\{ u(t-a) \bigr\} = \frac{e^{-as}}{s} .
\end{equation*}
This computation can be generalized into a \emph{\myindex{shifting property}}
or \emph{\myindex{second shifting property}}.
\begin{equation} \label{ltd:sseq}
\mybxbg{~~
\mathcal{L} \bigl\{ f(t-a) \, u(t-a) \bigr\}
= e^{-as} \mathcal{L} \bigl\{ f(t) \bigr\} .
~~}
\end{equation}
For example,
\begin{equation*}
\mathcal{L} \bigl\{ t \, u(t-1) \bigr\}
=
\mathcal{L} \bigl\{ \bigl( (t-1) + 1\bigr) \, u(t-1) \bigr\}
= e^{-as} \mathcal{L} \bigl\{ t+1 \bigr\}
= e^{-as} \left(\frac{1}{s^2} + \frac{1}{s} \right)
%= e^{-as} \Bigl(\frac{1}{s^2} + \frac{1}{s} \Bigr)
%= \frac{e^{-as}}{s^2} + \frac{e^{-as}}{s}
.
\end{equation*}

\begin{example} \label{lt:rocketex}
Consider
the mass-spring system
\begin{equation*}
x''(t) + x(t) = f(t) , \quad x(0) = 0, \quad x'(0) = 0,
\end{equation*}
where $f(t) = 1$ if $1 \leq t < 5$ and zero otherwise.
The $f(t)$ is not periodic and defined piecewise; no problem for Laplace.
Imagine a
rocket attached to the mass is fired for 4 seconds starting at
$t=1$.  Or perhaps imagine an RLC circuit, where the voltage is raised
at a constant rate for 4 seconds starting at $t=1$ and then held steady 
again
starting at $t=5$.

\pagebreak[2]
We write $f(t) = u(t-1) - u(t-5)$.  We transform the equation and we plug in
the initial conditions as before
\begin{equation*}
s^2 X(s) + X(s) = \frac{e^{-s}}{s} - \frac{e^{-5s}}{s} .
\end{equation*}
We solve for $X(s)$,
\begin{equation*}
X(s) = \frac{e^{-s}}{s(s^2+1)} - \frac{e^{-5s}}{s(s^2+1)} .
\end{equation*}
We leave it as an exercise to the reader to show that
\begin{equation*}
{\mathcal{L}}^{-1} \left\{ \frac{1}{s(s^2+1)} \right\}
= 1 - \cos t .
\end{equation*}
In other words,
$\mathcal{L} \{ 1 - \cos t  \} = 
\frac{1}{s(s^2+1)}$.
Using the second shifting property \eqref{ltd:sseq}, we find
\begin{equation*}
{\mathcal{L}}^{-1} \left\{ \frac{e^{-s}}{s(s^2+1)} \right\}
=
{\mathcal{L}}^{-1} \bigl\{
e^{-s}
\mathcal{L} \{ 1 - \cos t \}
\bigr\}
=
\bigl( 1 - \cos (t-1) \bigr) \, u(t-1) .
\end{equation*}
Similarly,
\begin{equation*}
{\mathcal{L}}^{-1} \left\{ \frac{e^{-5s}}{s(s^2+1)} \right\}
=
{\mathcal{L}}^{-1} \bigl\{
e^{-5s}
\mathcal{L} \{ 1 - \cos t \}
\bigr\}
=
\bigl( 1 - \cos (t-5) \bigr) \, u(t-5) .
\end{equation*}
Hence, the solution is
\begin{equation*}
x(t) = 
\bigl( 1 - \cos (t-1) \bigr) \, u(t-1) -
\bigl( 1 - \cos (t-5) \bigr) \, u(t-5) .
\end{equation*}
The plot of this solution is given in \figurevref{lt:heavisideexfig}.

\begin{myfig}
\capstart
\diffyincludegraphics{width=3in}{width=4.5in}{lt-heavisideex}
\caption{Plot of $x(t)$.\label{lt:heavisideexfig}}
\end{myfig}
\end{example}

\subsection{Transfer functions}

The Laplace transform leads to the following useful concept for studying the
steady state behavior of a linear system.  Consider an equation of the
form
\begin{equation*}
L x = f(t) ,
\end{equation*}
where $L$ is a linear constant coefficient differential operator.
Then $f(t)$ is usually thought of as input of the system and $x(t)$ is
thought of as the output of the system.  For example, for a mass-spring
system the input is the forcing function and the output is the behavior of the
mass.  We would like to have a convenient way to study the behavior of
the system for different inputs.

Let us suppose that
all the initial conditions are zero.
We take the Laplace transform of the equation to obtain the equation
\begin{equation*}
A(s) X(s) = F(s) .
\end{equation*}
Solving for the ratio $\nicefrac{X(s)}{F(s)}$, we obtain the so-called
\emph{\myindex{transfer function}}
$H(s) = \nicefrac{1}{A(s)}$,
that is,
\begin{equation*}
H(s) = \frac{X(s)}{F(s)} .
\end{equation*}
In other words, $X(s) = H(s) F(s)$.  We obtain an algebraic dependence of
the output of the system based on the input.  We can now easily study the
steady state behavior of the system given different inputs by simply
multiplying by the transfer function.
Moreover, it is possible to compute the $H(s)$ without knowing exactly what
the equation is by observing the output $X(s)$ for a given input $F(s)$.
Once $H(s)$ is known, you can find the output for any input.

\begin{example}
Given $x'' + \omega_0^2 x = f(t)$ (assume the initial conditions are zero),
let us find the transfer function.

First, we take the Laplace transform of the equation,
\begin{equation*}
s^2 X(s) + \omega_0^2 X(s) = F(s) .
\end{equation*}
Now we solve for the transfer function $\nicefrac{X(s)}{F(s)}$,
\begin{equation*}
H(s) = \frac{X(s)}{F(s)} = \frac{1}{s^2 + \omega_0^2} .
\end{equation*}

Let us see how to use the transfer function.  Suppose we have the constant input
$f(t) = 1$.  Hence $F(s) = \nicefrac{1}{s}$, and
\begin{equation*}
X(s) = H(s) F(s) = \frac{1}{s^2+\omega_0^2} \, \frac{1}{s} .
\end{equation*}
Taking the inverse Laplace transform of $X(s)$, we obtain
\begin{equation*}
x(t) = \frac{1-\cos(\omega_0 t)}{\omega_0^2} .
\end{equation*}
Similarly, for any other input $F(s)$, the output is
$X(s) = H(s) F(s) = \frac{1}{s^2+\omega_0^2} \, F(s)$.
\end{example}

\subsection{Transforms of integrals}

A feature of Laplace transforms is that it is also able to easily deal
with integral equations.  That is, equations in which integrals rather than
derivatives of functions appear.  The basic property, which can be proved
by applying the definition and doing integration by parts, is 
\begin{equation*}
\mybxbg{~~
\mathcal{L} \left\{
\int_0^t f(\tau) \, d\tau
\right\} = \frac{1}{s} \, F(s) .
~~}
\end{equation*}
It is sometimes useful (e.g.\ for computing the inverse transform) to write
this as
\begin{equation*}
\int_0^t f(\tau) \, d\tau
=
{\mathcal{L}}^{-1} \left\{
\frac{1}{s} \, F(s) \right\} .
\end{equation*}

\begin{example}
To compute ${\mathcal{L}}^{-1} \left\{\frac{1}{s(s^2+1)}\right\}$ we could
apply this integration rule:
\begin{equation*}
{\mathcal{L}}^{-1} \left\{
\frac{1}{s} \, \frac{1}{s^2+1} \right\} 
=
\int_0^t 
{\mathcal{L}}^{-1} \left\{
\frac{1}{s^2+1} \right\} \, d\tau
=
\int_0^t 
\sin \tau \, d\tau
=
1 - \cos t .
\end{equation*}
\end{example}


\begin{example}
An equation containing an integral of the unknown function is
called an \emph{\myindex{integral equation}}.
Consider
\begin{equation*}
x(t) - t = \int_0^t x(\tau) \, d\tau ,
\end{equation*}
where we wish to solve for $x(t)$.
We apply the Laplace transform to get
\begin{equation*}
X(s) - \frac{1}{s^2} = \frac{1}{s} X(s) ,
\end{equation*}
where $X(s) = \mathcal{L} \bigl\{ x(t) \bigr\}$.  Thus
\begin{equation*}
X(s) = \frac{1}{s(s-1)} = \frac{1}{s-1} - \frac{1}{s} .
\end{equation*}
The inverse Laplace transform gives
\begin{equation*}
x(t) = e^t - 1 .
\end{equation*}
%More complicated integral
%equations can also be solved using convolution, about which we will
%learn in the next section.
\end{example}

\subsection{Periodic functions}

The reader might ask: What about periodic functions as our input $f(t)$?
That is, a function $f(t)$ where $f(t) = f(t+P)$ for some constant $P$ (the period).
Well, let us compute $F(s)$:
\begin{equation*}
\begin{split}
F(s)
& = \int_0^\infty e^{-st} f(t) \, dt
= \int_0^P e^{-st} f(t) \, dt
+ \int_P^\infty e^{-st} f(t) \, dt
\\
& = \int_0^P e^{-st} f(t) \, dt
+ \int_0^\infty e^{-s(t+P)} f(t+P) \, dt
= \int_0^P e^{-st} f(t) \, dt
+ e^{-Ps} \int_0^\infty e^{-st} f(t) \, dt
\\
& = \int_0^P e^{-st} f(t) \, dt
+ e^{-Ps} F(s) .
\end{split}
\end{equation*}
Solving for $F(s)$, we get
\begin{equation*}
F(s)
=
\frac{1}{1-e^{-Ps}}
\int_0^P e^{-st} f(t) \, dt .
\end{equation*}
As before, computing the inverse would be more complex and possibly
involve consulting a table.  Let us not worry about computing the inverse
here.

\begin{example}
Suppose our function $f(t)$ is a version of the \emph{\myindex{sawtooth}}, that is,
let
$f(t) = t$ for $0 \leq t < 1$
and use $f(t)=f(t+1)$ to extend it periodically.  So
$f(t) = t-1$ for $1 \leq t < 2$, 
$f(t) = t-2$ for $2 \leq t < 3$, etc.  Then $P=1$ and
a short computation with integration by parts gets
\begin{equation*}
F(s) =
\frac{1}{1-e^{-s}}
\int_0^1 e^{-st} t \, dt 
=
\frac{1}{1-e^{-s}}
\left(
\frac{-e^{-s}}{s}-\frac{e^{-s}}{s^2} + \frac{1}{s^2}
\right)
=
\frac{-e^{-s}}{(1-e^{-s})s} + \frac{1}{s^2} .
\end{equation*}
\end{example}

\subsection{Exercises}

\begin{exercise}
Using the Heaviside function write down the piecewise function
that is 0 for $t < 0$, $t^2$ for $t$ in $[0,1]$ and $t$ for $t > 1$.
\end{exercise}

\begin{exercise}
Using the Laplace transform solve
\begin{equation*}
m x'' + c x' + k x = 0 , \quad x(0) = a, \quad x'(0) = b ,
\end{equation*}
where $m > 0$, $c > 0$, $k > 0$, and
$c^2 - 4km > 0$ (system is overdamped).
\end{exercise}

\begin{exercise}
Using the Laplace transform solve
\begin{equation*}
m x'' + c x' + k x = 0 , \quad x(0) = a, \quad x'(0) = b ,
\end{equation*}
where $m > 0$, $c > 0$, $k > 0$, and
$c^2 - 4km < 0$ (system is underdamped).
\end{exercise}

\begin{exercise}
Using the Laplace transform solve
\begin{equation*}
m x'' + c x' + k x = 0 , \quad x(0) = a, \quad x'(0) = b ,
\end{equation*}
where $m > 0$, $c > 0$, $k > 0$, and
$c^2 = 4km$ (system is critically damped).
\end{exercise}

\begin{exercise}
Solve $x'' + x = u(t-1)$ for initial conditions $x(0) = 0$ and $x'(0) = 0$.
\end{exercise}

\begin{exercise}
Show the ``differentiation of the transform'' property.  Suppose
$\mathcal{L} \bigl\{ f(t) \bigr\} = F(s)$, then show
\begin{equation*}
\mathcal{L} \bigl\{ -t f(t) \bigr\} = F'(s) .
\end{equation*}
Hint: Differentiate under the integral sign.
\end{exercise}

\begin{exercise}
Solve $x''' + x = t^3 u(t-1)$ for initial conditions $x(0) = 1$ and $x'(0) =
0$, $x''(0) = 0$.
\end{exercise}

\begin{exercise}
Show the second shifting property: 
$\mathcal{L} \bigl\{ f(t-a) \, u(t-a) \bigr\}
= e^{-as} \mathcal{L} \bigl\{ f(t) \bigr\}$.
\end{exercise}

\begin{exercise}
Let us think of the mass-spring system with a rocket from
\exampleref{lt:rocketex}.  We noticed that the solution kept oscillating
after the rocket stopped running.  The amplitude of the oscillation depends
on the time that the rocket was fired (for 4 seconds in the example).
\begin{tasks}
\task
Find a formula for the amplitude of the resulting oscillation
in terms of the amount of time the rocket is fired.
\task
 Is there
a nonzero time (if so what is it?)
for which the rocket fires and the resulting oscillation
has amplitude 0 (the mass is not moving)?
\end{tasks}
\end{exercise}

\begin{exercise}
Define
\begin{equation*}
f(t) =
\begin{cases}
{(t-1)}^2 & \text{if } \; 1 \leq t < 2, \\
3-t & \text{if } \; 2 \leq t < 3, \\
0 & \text{otherwise} .
\end{cases}
\end{equation*}
\begin{tasks}
\task Sketch the graph of $f(t)$.
\task Write down $f(t)$ using the Heaviside function.
\task Solve $x''+x=f(t)$, $x(0)=0$, $x'(0) = 0$ using Laplace transform.
\end{tasks}
\end{exercise}

\begin{exercise}
Find the transfer function for 
$m x'' + c x' + kx = f(t)$,
$x(0)=0$, $x'(0)=0$.
\end{exercise}

\begin{exercise}
Suppose $Lx = f(t)$, $x(0)=0$, $x'(0)=0$
for the input function $f(t) = t$ has the output
$x(t) = 2e^{-t} + te^{-t} + (t-2)$.
\begin{tasks}
\task Find $F(s)$, $X(s)$, and the transfer function $H(s)$.
\task If the input is instead $f(t) = \sin(t)$ instead, find the new output $x(t)$.
\end{tasks}
\end{exercise}

\begin{exercise}
Suppose $f(t) = 1$ if $0 \leq t < 1$ and $f(t)=0$ if $1 \leq t < 2$, and
then extend periodically for all $t \geq 0$ so that $f(t)=f(t+2)$.  Compute the
Laplace transform $F(s)$.
\end{exercise}

\setcounter{exercise}{100}

\begin{exercise}
Using the Heaviside function $u(t)$, write down the function
\begin{equation*}
f(t) =
\begin{cases}
0 & \text{if } \; \phantom{1 \leq {}} t < 1  , \\
t-1 & \text{if } \; 1 \leq t < 2 , \\
1 & \text{if } \; 2 \leq t .
\end{cases}
\end{equation*}
\end{exercise}
\exsol{%
$f(t) =
(t-1)\bigl(u(t-1) - u(t-2)\bigr) + 
u(t-2)$
}

\begin{exercise}
Solve $x''-x = (t^2-1) u(t-1)$ for initial conditions $x(0)=1$, $x'(0) = 2$
using the Laplace transform.
\end{exercise}
\exsol{%
$x(t) = (2e^{t-1}-t^2-1) u(t-1) -\frac{1}{2}e^{-t}+\frac{3}{2}e^t$
}
 
\begin{exercise}
Find the transfer function for 
$x' + x = f(t)$,
$x(0)=0$.
\end{exercise}
\exsol{%
$H(s) = \frac{1}{s+1}$
}

\begin{exercise}
Suppose $Lx = f(t)$, $x(0)=0$, $x'(0)=0$
for the input function $f(t) = 4$ has the output
$x(t) = 1 - \cos(2 t)$.  Find $F(s)$, $X(s)$, and the transfer function
$H(s)$.
\end{exercise}
\exsol{%
$F(s) = \frac{4}{s}$,
$X(s) = \frac{1}{s} - \frac{s}{s^2+4}$,
$H(s) = \frac{1}{s^2+4}$.
}

%%%%%%%%%%%%%%%%%%%%%%%%%%%%%%%%%%%%%%%%%%%%%%%%%%%%%%%%%%%%%%%%%%%%%%%%%%%%%%

\sectionnewpage
\section{Convolution}
\label{convolution:section}

%mbxINTROSUBSECTION

\sectionnotes{1 or 1.5 lectures\EPref{, \S7.2 in \cite{EP}}\BDref{,
\S6.6 in \cite{BD}}}

\subsection{The convolution}

The Laplace transformation of a product is not the product
of the transforms.  All hope is not lost however.  We simply have to use
a different type of a \myquote{product.}
Take
two functions $f(t)$ and $g(t)$ defined for $t \geq 0$,
and define the \emph{\myindex{convolution}}%
\footnote{%
For those that have seen convolution before, you may have
seen it defined as
$(f * g)(t) =
\int_{-\infty}^\infty f(\tau) g(t-\tau) \, d\tau$.  This definition
agrees with \eqref{ltc:convdef} if you define $f(t)$ and $g(t)$
to be zero for $t < 0$.
When discussing the Laplace transform, the definition we gave is
sufficient.  Convolution does occur in many other applications, however,
where you may have to use the more general definition with infinities.
}
of $f(t)$ and $g(t)$ as
\begin{equation} \label{ltc:convdef}
\mybxbg{~~
(f * g)(t) \overset{\text{def}}{=}
\int_0^t f(\tau) g(t-\tau) \, d\tau .
~~}
\end{equation}
As you can see, the convolution of two functions of $t$ is another function of $t$.


\begin{example}
Take $f(t) = e^t$ and $g(t) = t$ for $t \geq 0$.  Then 
\begin{equation*}
(f*g)(t)
=
\int_0^t e^\tau (t-\tau) \, d\tau
=
e^t - t - 1 .
\end{equation*}
To solve the integral we
did one integration by parts.
\end{example}

\begin{example} \label{ltc:convsincosex}
Take $f(t) = \sin (\omega t)$ and $g(t) = \cos (\omega t)$ for $t \geq 0$.
Then 
\begin{equation*}
(f*g)(t)
=
\int_0^t  \sin ( \omega \tau ) \,
\cos \bigl( \omega (t-\tau) \bigr) \, d\tau .
\end{equation*}
Apply the identity
\begin{equation*}
\cos (\theta) \sin (\psi) =
\frac{1}{2} \, \bigl( \sin (\theta + \psi) - \sin (\theta - \psi) \bigr) ,
\end{equation*}
to get
\begin{equation*}
\begin{split}
(f*g)(t)
& =
\int_0^t
\frac{1}{2} \, \bigl( \sin (\omega t) - \sin (\omega t - 2 \omega \tau
) \bigr) \, d\tau
\\
& =
\left[ \frac{1}{2} \, \tau  \sin (\omega t) + \frac{1}{4\omega} \, \cos (2 \omega \tau -
\omega t) \right]_{\tau=0}^t
\\
& = \frac{1}{2} \, t \sin (\omega t) .
\end{split}
\end{equation*}
The formula holds only for $t \geq 0$.  The functions $f$, $g$,
and $f*g$ are undefined for $t < 0$.
\end{example}

Convolution has many properties that make it behave like a product.
Let $c$ be a constant and $f$, $g$, and $h$ be functions.  Then it is a
calculus exercise to show
\begin{align*}
& f * g = g * f , \\
& (c f) * g = f * (c g) = c (f*g) , \\
& (f+g) * h = f * h + g * h , \\
& ( f * g ) * h = f * ( g * h ) .
\end{align*}
The most interesting property for us is the following theorem.

\begin{theorem}
If $f(t)$ and $g(t)$ are of exponential order, then
so is $(f*g)(t)$ and
\begin{equation*}
\mybxbg{~~
\mathcal{L} \bigl\{ (f*g)(t) \bigr\}
=
\mathcal{L} \left\{ \int_0^t f(\tau) g(t-\tau) \, d\tau \right\}
=
\mathcal{L} \bigl\{ f(t) \bigr\} \mathcal{L} \bigl\{ g(t) \bigr\} .
~~}
\end{equation*}
\end{theorem}

In other words, the Laplace transform of a convolution is the product
of the Laplace transforms.  The simplest way to use this result is in
reverse.

\begin{example}
Suppose we have the function of $s$
defined by
\begin{equation*}
\frac{1}{(s+1)s^2} = 
\frac{1}{s+1}\,
\frac{1}{s^2} .
\end{equation*}
We recognize the two entries of \tableref{ltd:table}.  That is,
\begin{equation*}
\mathcal{L}^{-1} 
\left\{
\frac{1}{s+1} \right\}
= e^{-t}
\qquad \text{and} \qquad
\mathcal{L}^{-1} 
\left\{
\frac{1}{s^2} \right\} 
= t.
\end{equation*}
Therefore,
\begin{equation*}
\mathcal{L}^{-1}
\left\{
\frac{1}{s+1}\,
\frac{1}{s^2} \right\}
=
\int_0^t
\tau e^{-(t-\tau)} \,d\tau
=
e^{-t}+t-1 .
\end{equation*}
The calculation of the integral involved an integration by parts.
\end{example}

\subsection{Solving ODEs}

The next example demonstrates the full power of the convolution and
the Laplace transform.  We can give the solution to
the forced oscillation problem for any forcing function as a definite
integral.

\begin{example} \label{example:undampedforcedbylaplacearbitrhs}
Find the solution to
\begin{equation*}
x'' + \omega_0^2 x = f(t) , \quad x(0) = 0, \quad x'(0) = 0 ,
\end{equation*}
for an arbitrary function $f(t)$.

We first apply the Laplace transform to the equation.  Denote
the transform of $x(t)$ by $X(s)$ and the transform of $f(t)$ by
$F(s)$ as usual.  We get
\begin{equation*}
s^2 X(s) + \omega_0^2 X(s) = F(s) ,
\end{equation*}
or in other words,
\begin{equation*}
X(s) = \frac{1}{s^2+ \omega_0^2} F(s).
\end{equation*}
Recall that $H(s) = \frac{1}{s^2+ \omega_0^2}$ is the transfer function.
We know
\begin{equation*}
{\mathcal{L}}^{-1} \left\{
\frac{1}{s^2+ \omega_0^2}
\right\} = 
\frac{\sin (\omega_0 t)}{\omega_0} .
\end{equation*}
Therefore,
\begin{equation*}
x(t) = 
\int_0^t
\frac{\sin (\omega_0 \tau)}{\omega_0}
f(t-\tau) \, d\tau ,
\end{equation*}
or, if we reverse the order,
\begin{equation*}
x(t) = 
\int_0^t
f(\tau) 
\frac{\sin \bigl( \omega_0 (t-\tau) \bigr)}{\omega_0} \, d\tau .
\end{equation*}
\end{example}

Notice one more feature of the example above.
We can now see how Laplace transform
handles \myindex{resonance}.  Suppose that $f(t) =
\cos (\omega_0 t)$.  Then
\begin{equation*}
x(t) = 
\int_0^t
\frac{\sin (\omega_0 \tau)}{\omega_0} \,
\cos \bigl( \omega_0 (t-\tau) \bigr) \, d\tau
=
\frac{1}{\omega_0}
\int_0^t
\sin ( \omega_0 \tau ) \,
\cos \bigl(\omega_0 (t-\tau) \bigr) \, d\tau .
\end{equation*}
We have computed the convolution of sine and cosine in
\exampleref{ltc:convsincosex}.  Hence
\begin{equation*}
x(t) =
\left(
\frac{1}{\omega_0}
\right) \,
\left(
\frac{1}{2} \,
t
\sin ( \omega_0 t )
\right)
=
\frac{1}{2 \omega_0} \,
t
\sin ( \omega_0 t ).
\end{equation*}
Note the $t$ in front of the sine.  The solution, therefore, grows without
bound as $t$ gets large, meaning we get resonance.

The general idea here is that if $H(s)$ is the transfer function, then
$X(s)=H(s)F(s)$.
If we find the $h(t) = {\mathcal{L}}^{-1}\bigl\{ H(s) \bigr\}$, then
\begin{equation*}
x(t)
= {\mathcal{L}}^{-1}\bigl\{ X(s) \bigr\}
= {\mathcal{L}}^{-1}\bigl\{ F(s)H(s) \bigr\}
= (f * h)(t)
= \int_0^t f(\tau) h(t-\tau) \, d\tau .
\end{equation*}
Hence,
we can solve any constant coefficient equation with an arbitrary forcing
function $f(t)$ as a definite integral using convolution.
A definite integral, rather than a closed form solution, is usually enough
for most practical purposes.  It is
not hard to numerically evaluate a definite integral.

\subsection{Volterra integral equation}

A common integral equation\index{integral equation}
is the \emph{\myindex{Volterra integral equation}}%
\footnote{Named for the Italian mathematician
\href{https://en.wikipedia.org/wiki/Vito_Volterra}{Vito Volterra}
(1860--1940).}
\begin{equation*}
x(t) = f(t) + \int_0^t g(t-\tau) x(\tau) \, d\tau ,
\end{equation*}
where $f(t)$ and $g(t)$ are known functions and $x(t)$ is an unknown we
wish to solve for.
To find $x(t)$,
we apply the Laplace transform to the equation to obtain 
\begin{equation*}
X(s) = F(s) + G(s) X(s) ,
\end{equation*}
where $X(s)$, $F(s)$, and $G(s)$ are the Laplace transforms of $x(t)$, $f(t)$, and
$g(t)$ respectively.  We find
\begin{equation*}
X(s) = \frac{F(s)}{1-G(s)} .
\end{equation*}
To find $x(t)$, we now need to find the 
inverse Laplace transform of $X(s)$.

\begin{example}
Solve
\begin{equation*}
x(t) =  e^{-t} + \int_0^t \sinh(t-\tau) x(\tau) \, d\tau .
\end{equation*}

We apply Laplace transform to obtain
\begin{equation*}
X(s) = \frac{1}{s+1} + \frac{1}{s^2-1} X(s) ,
\end{equation*}
or
\begin{equation*}
X(s) = \frac{\frac{1}{s+1}}{1- \frac{1}{s^2-1}}
=
\frac{s-1}{s^2 - 2}
=
\frac{s}{s^2 - 2}
-
\frac{1}{s^2 - 2} .
\end{equation*}
It is not hard to apply \tablevref{lt:table} to find
\begin{equation*}
x(t) = \cosh \bigl( \sqrt{2} \, t \bigr) -
\frac{1}{\sqrt{2}} \sinh \bigl( \sqrt{2}\, t \bigr).
\end{equation*}
\end{example}

\subsection{Exercises}

\begin{exercise}
Let $f(t) = t^2$ for $t \geq 0$, and $g(t) = u(t-1)$.  Compute
$f * g$.
\end{exercise}

\begin{exercise}
Let $f(t) = t$ for $t \geq 0$, and $g(t) = \sin t $ for $t \geq 0$.  Compute
$f * g$.
\end{exercise}

\begin{exercise}
Find the solution to
\begin{equation*}
m x'' + c x' + k x = f(t) , \quad x(0) = 0, \quad x'(0) = 0 ,
\end{equation*}
for an arbitrary function $f(t)$, where $m > 0$, $c > 0$, $k > 0$,
and $c^2 - 4km > 0$ (the system is overdamped).
Write the solution as a definite integral.
\end{exercise}

\begin{exercise}
Find the solution to
\begin{equation*}
m x'' + c x' + k x = f(t) , \quad x(0) = 0, \quad x'(0) = 0 ,
\end{equation*}
for an arbitrary function $f(t)$, where $m > 0$, $c > 0$, $k > 0$,
and $c^2 - 4km < 0$ (the system is underdamped).
Write the solution as a definite integral.
\end{exercise}

\begin{exercise}
Find the solution to
\begin{equation*}
m x'' + c x' + k x = f(t) , \quad x(0) = 0, \quad x'(0) = 0 ,
\end{equation*}
for an arbitrary function $f(t)$, where $m > 0$, $c > 0$, $k > 0$,
and $c^2 = 4km$ (the system is critically damped).
Write the solution as a definite integral.
\end{exercise}

\begin{exercise}
Solve
\begin{equation*}
x(t) =  e^{-t} + \int_0^t \cos(t-\tau) x(\tau) \, d\tau .
\end{equation*}
\end{exercise}

\begin{exercise}
Solve
\begin{equation*}
x(t) =  \cos t + \int_0^t \cos(t-\tau) x(\tau) \, d\tau .
\end{equation*}
\end{exercise}

\begin{exercise}
Compute ${\mathcal{L}}^{-1} \left\{ \frac{s}{{(s^2+4)}^2} \right\}$ using
convolution.
\end{exercise}

\begin{exercise}
Write down the solution to
$x''-2x=e^{-t^2}$, $x(0)=0$, $x'(0)=0$ as a
definite integral.  Hint: Do not try to compute the
Laplace transform of $e^{-t^2}$.
\end{exercise}

\setcounter{exercise}{100}

\begin{exercise}
Let $f(t) = \cos t$ for $t \geq 0$, and $g(t) = e^{-t}$.  Compute
$f * g$.
\end{exercise}
\exsol{%
$\frac{1}{2}(\cos t + \sin t - e^{-t})$
}

\begin{exercise}
Compute ${\mathcal{L}}^{-1} \left\{ \frac{5}{s^4+s^2} \right\}$ using
convolution.
\end{exercise}
\exsol{%
$5t-5\sin t$
}

\begin{exercise}
Solve $x''+x = \sin t$, $x(0) = 0$, $x'(0)=0$ using convolution.
\end{exercise}
\exsol{%
$\frac{1}{2}(\sin t - t \cos t)$
}

\begin{exercise}
Solve $x'''+x' = f(t)$, $x(0) = 0$, $x'(0)=0$, $x''(0)=0$ using convolution.
Write the result as a definite integral.
\end{exercise}
\exsol{%
$\int_0^t f(\tau) \bigl( 1 - \cos (t-\tau)\bigr)\, d\tau$
}

%%%%%%%%%%%%%%%%%%%%%%%%%%%%%%%%%%%%%%%%%%%%%%%%%%%%%%%%%%%%%%%%%%%%%%%%%%%%%%

\sectionnewpage
\section{Dirac delta and impulse response}
\label{diracdelta:section}

%mbxINTROSUBSECTION

\sectionnotes{1 or 1.5 lecture\EPref{, \S7.6 in \cite{EP}}\BDref{,
\S6.5 in \cite{BD}}}

\subsection{Rectangular pulse}

Often we study a physical system by putting in a short pulse 
and then seeing what the system does.
The resulting behavior is
often called \emph{\myindex{impulse response}},
and understanding it tells us how the system responds to any input.
Let us see what we mean by a pulse.
The simplest kind of a pulse is a simple
\emph{\myindex{rectangular pulse}}\index{pulse}
defined by
\begin{equation*}
\varphi(t) = 
\begin{cases}
0 & \text{if } \; \phantom{a \leq {}} t < a , \\
M & \text{if } \; a \leq t < b , \\
0 & \text{if } \; b \leq t .
\end{cases}
\end{equation*}
See \figurevref{lt:sqpulse} for a graph.

%15 is number of lines, must be adjusted
\begin{mywrapfig}[15]{3.25in}
\capstart
\diffyincludegraphics{width=3in}{width=4.5in}{lt-sqpulse}
\caption{Sample rectangular pulse with $a=0.5$, $b=1$, and $M = 2$.\label{lt:sqpulse}}
\end{mywrapfig}

Notice that
\begin{equation*}
\varphi(t) = M \bigl( u(t-a) - u(t-b) \bigr) ,
\end{equation*}
where $u(t)$ is the unit step function.
Let us take the Laplace transform of a rectangular pulse,
\begin{equation*}
\begin{split}
{\mathcal{L}} \bigl\{ \varphi(t) \bigr\}
& =
{\mathcal{L}} \bigl\{ M \bigl( u(t-a) - u(t-b) \bigr)  \bigr\}
\\
& =
M
\frac{e^{-as} - e^{-bs}}{s} .
\end{split}
\end{equation*}

For simplicity, let $a=0$.
It is also convenient
to set $M = \nicefrac{1}{b}$ so that
\begin{equation*}
\int_0^\infty \varphi(t) \,dt = 1 .
\end{equation*}
That is, to have the pulse have \myquote{unit mass.}
For such a pulse,
\begin{equation*}
{\mathcal{L}} \bigl\{ \varphi(t) \bigr\}
=
{\mathcal{L}} \left\{ \frac{u(t) - u(t-b)}{b}  \right\}
=
\frac{1 - e^{-bs}}{bs} .
\end{equation*}
We want $b$ to be very small; we wish to have
the pulse be very short and very tall.  By letting $b$ go to zero, we arrive
at the concept of the Dirac delta function.  The limit
of $\frac{1-e^{-bs}}{bs}$
as $b \to \infty$ is $1$, so we are looking for a \myquote{function} whose Laplace
transform is $1$.

\subsection{The delta function}

The \emph{\myindex{Dirac delta function}}\index{delta function}%
\footnote{Named after the English physicist and mathematician
\href{https://en.wikipedia.org/wiki/Paul_Dirac}{Paul Adrien Maurice Dirac}
(1902--1984).}
is not exactly a function; it is sometimes called a
\emph{\myindex{generalized function}}.  We
avoid unnecessary details and simply say that it is an object
that does not really make sense unless we integrate it.  The motivation is
that we would like a \myquote{function} $\delta(t)$
such that 
for any continuous function $f(t)$,
\begin{equation*}
\mybxbg{~~
\int_{-\infty}^\infty \delta(t) f(t) \,dt = f(0) .
~~}
\end{equation*}
The formula should hold if we integrate over any interval that contains 0,
not just $(-\infty,\infty)$.
So $\delta(t)$ is a \myquote{function} 
with all its \myquote{mass} at the single point $t=0$.  For any
interval\footnote{%
It is important that we consider $c$ and $d$ as part of the
interval.  One could write this integral as $\int_{c-}^{d+}$.}
$[c,d]$,
\begin{equation*}
\int_c^d \delta(t) \,dt = 
\begin{cases}
1 & \text{if the interval $[c,d]$ contains 0, i.e.\ } c \leq 0 \leq d, \\
0 & \text{otherwise.}
\end{cases}
\end{equation*}
Unfortunately there is no such function in the classical sense.  You could
informally think that $\delta(t)$ is zero for $t\not=0$ and somehow
infinite at $t=0$.

A good way to think about $\delta(t)$ is as a limit of pulses
of decreasing length
whose integral is $1$.  For example, 
consider a rectangular pulse
$\varphi(t)$ as above with $a=0$ and
$M=\nicefrac{1}{b}$, that is, $\varphi(t) = \frac{u(t) - u(t-b)}{b}$.
Compute
\begin{equation*}
\int_{-\infty}^\infty \varphi(t) f(t) \,dt =
\int_{-\infty}^\infty \frac{u(t) - u(t-b)}{b} f(t) \,dt =
\frac{1}{b} \int_{0}^b f(t) \,dt .
\end{equation*}
If $f(t)$ is continuous at $t=0$, then
for very small $b$, the function $f(t)$ is approximately equal to $f(0)$ on
the interval $[0,b]$.  We approximate the integral
\begin{equation*}
\frac{1}{b} \int_{0}^b f(t) \,dt \approx
\frac{1}{b} \int_{0}^b f(0) \,dt = f(0) .
\end{equation*}
Hence,
\begin{equation*}
\lim_{b\to 0}
\int_{-\infty}^\infty \varphi(t) f(t) \,dt =
\lim_{b\to 0}
\frac{1}{b} \int_{0}^b f(t) \,dt  = f(0) .
\end{equation*}

Let us therefore accept $\delta(t)$ as an object that is possible to
integrate.  We often want to shift $\delta$ to another point, for example
$\delta(t-a)$.  In that case,
\begin{equation*}
\int_{-\infty}^\infty \delta(t-a) f(t) \,dt = f(a) .
\end{equation*}
Note that $\delta(a-t)$ is the same object as $\delta(t-a)$.
With that, the convolution of $\delta(t)$ with $f(t)$ is again $f(t)$,
\begin{equation*}
(f * \delta) (t) = 
\int_{0}^t f(\tau) \delta(t-\tau) \,d\tau
= f(t) .
\end{equation*}

As we can integrate $\delta(t)$, we compute its Laplace transform:
\begin{equation*}
\mybxbg{~~
{\mathcal{L}} \bigl\{ \delta(t-a) \bigr\}
=
\int_{0}^\infty e^{-st} \delta(t-a) \,dt = e^{-as} .
~~}
\end{equation*}
In particular,
\begin{equation*}
{\mathcal{L}} \bigl\{ \delta(t) \bigr\} = 1 .
\end{equation*}

\begin{remark} \label{remark:deltaderivofjump}
The Laplace transform of $\delta(t-a)$ would be 
the Laplace transform of the derivative of the Heaviside function
$u(t-a)$, if the Heaviside function had a derivative.
First,
\begin{equation*}
{\mathcal{L}} \bigl\{ u(t-a) \bigr\} = \frac{e^{-as}}{s}.
\end{equation*}
To obtain what the Laplace transform of the derivative would be,
we multiply by $s$, to obtain $e^{-as}$, which is the Laplace transform
of $\delta(t-a)$.
We see the same thing using integration,
\begin{equation*}
\int_0^t \delta(s-a)\,ds = u(t-a) .
\end{equation*}
So in a certain sense
\begin{equation*}
%mbxSTARTIGNORE
\text{``}
%mbxENDIGNORE
%mbxlatex \text{"}
\quad \frac{d}{dt} \Bigl[ u(t-a) \Bigr] = \delta(t-a) . \quad
%mbxSTARTIGNORE
\text{''}
%mbxENDIGNORE
%mbxlatex \text{"}
\end{equation*}
This line of reasoning allows us to talk about derivatives of functions with jump
discontinuities.
We can think of
the derivative of the Heaviside function $u(t-a)$ as being somehow infinite
at $a$, which is precisely our intuitive understanding of the delta
function.
\end{remark}

\begin{example}
Let us compute ${\mathcal{L}}^{-1} \left\{ \frac{s+1}{s} \right\}$.  So
far we only computed the inverse transform of proper rational functions
in the $s$ variable.
That is, the numerator was of lower degree than the denominator.
Not so with $\frac{s+1}{s}$.
We can use the delta function to compute,
\begin{equation*}
{\mathcal{L}}^{-1} \left\{ \frac{s+1}{s} \right\}
=
{\mathcal{L}}^{-1} \left\{ 1 + \frac{1}{s} \right\}
=
{\mathcal{L}}^{-1} \{ 1 \}
+
{\mathcal{L}}^{-1} \left\{ \frac{1}{s} \right\}
=
\delta(t) + 1 .
\end{equation*}
The resulting object is a generalized
function and only makes sense when put underneath an integral.
\end{example}

\subsection{Impulse response}

As we said before, in the differential equation
$L x = f(t)$,
we think of $f(t)$ as input, and $x(t)$ as the output.
We think of the delta function as an impulse, and so
to find the response to an impulse, we use
the delta function in place of $f(t)$.
The solution to
\begin{equation*}
L x = \delta(t)
\end{equation*}
is called the
\emph{\myindex{impulse response}}.

\begin{example}
Solve (find the impulse response)
\begin{equation} \label{eq:lteximpulseresp}
x'' + \omega_0^2 x = \delta(t) , \quad x(0) = 0, \quad x'(0) = 0 .
\end{equation}

Apply the Laplace transform to the equation, and denote
the transform of $x(t)$ by $X(s)$:
\begin{equation*}
s^2 X(s) + \omega_0^2 X(s) = 1 ,
\qquad \text{and so} \qquad
X(s) = \frac{1}{s^2+ \omega_0^2} .
\end{equation*}
The inverse Laplace transform produces (for $t > 0$)
\begin{equation*}
x(t) = 
\frac{\sin (\omega_0 t)}{\omega_0} .
\end{equation*}
\end{example}

\begin{remark}
Perhaps an astute reader will notice that it does not seem like $x'(0)=0$.
However, we really want to think of as $x(t)=0$ for $t \leq 0$, so $x(t)$
has a \myquote{corner} at $t=0$, so $x'$ has a jump discontinuity there,
which is what produces the $\delta(t)$ when we take $x''$,
see \remarkref{remark:deltaderivofjump}.
The initial condition really is $x'(0-) = \lim_{t \uparrow 0} x'(0) = 0$.
\end{remark}

Let us notice something about the example above. 
In \exampleref{example:undampedforcedbylaplacearbitrhs},
we found that
when the input is $f(t)$, the solution to $Lx = f(t)$
is given by
\begin{equation*}
x(t) = 
\int_0^t
f(\tau) 
\frac{\sin \bigl( \omega_0 (t-\tau) \bigr)}{\omega_0} \, d\tau .
\end{equation*}
That is, \emph{the solution for an arbitrary input is given as
convolution with the impulse response}.  Let us see why.
The key is to notice that for functions $f(t)$ and $h(t)$,
\begin{equation*}
(f * h)''(t) =
\frac{d^2}{dt^2}\left[
\int_0^t
f(\tau) 
h(t-\tau) \, d\tau \right]
=
\int_0^t
f(\tau) 
h''(t-\tau) \, d\tau
= (f * h'')(t) .
\end{equation*}
We simply differentiate twice under the
integral\footnote{You should really think of the integral going over
$(-\infty,\infty)$ rather than over $[0,t]$ and simply assume that $f(t)$ and
$h(t)$ are zero for negative $t$.}.
Suppose that $h(t)$ is the impulse response (solution to $h''+\omega_0^2 h = \delta(t)$).
If we convolve the entire equation \eqref{eq:lteximpulseresp},
the left-hand side becomes
\begin{equation*}
f * (h'' + \omega_0^2 h) =
(f * h'') + \omega_0^2 (f * h) =
(f * h)'' + \omega_0^2 (f * h) .
\end{equation*}
The right-hand side becomes
\begin{equation*}
(f * \delta)(t) = f(t).
\end{equation*}
Therefore, if $h$ is the impulse response,
then $x(t) = (f * h)(t)$ is the solution to
\begin{equation*}
x'' + \omega_0^2 x = f(t) .
\end{equation*}

The procedure works for any constant coefficient linear
equation $Lx = f(t)$.
If you find the impulse response $h$
(solution to $Lh = \delta(t)$),
you also know how to obtain the output $x(t)$ for any input $f(t)$.
We simply convolve, $x(t) = (f * h)(t)$.
As you may have noticed in the example,
the impulse response is in fact just the inverse Laplace
transform of the transfer function, that is,
$h(t) = {\mathcal{L}}^{-1} \bigl\{ H(s) \bigr\}$.

\subsection{Three-point beam bending}
\index{three-point beam bending}

Let us give another quite different
example where the delta function turns up:
Representing point loads on a steel beam.  Consider a beam
of length $L$, resting on two simple supports at the ends.  Let $x$ denote
the position on the beam, and let $y(x)$ denote the deflection of the beam in
the vertical direction.  The deflection $y(x)$ satisfies the
\emph{\myindex{Euler--Bernoulli equation}}%
\footnote{Named for the Swiss mathematicians
\href{https://en.wikipedia.org/wiki/Jacob_Bernoulli}{Jacob Bernoulli}
(1654--1705),
\href{https://en.wikipedia.org/wiki/Daniel_Bernoulli}{Daniel Bernoulli}
(1700--1782), the nephew of Jacob,
and
\href{https://en.wikipedia.org/wiki/Euler}{Leonhard Paul Euler}
(1707--1783).},
\begin{equation*}
EI \frac{d^4 y}{dx^4} = F(x) ,
\end{equation*}
where $E$ and $I$ are constants\footnote{$E$ is the elastic modulus and $I$
is the second moment of area.  Let us not worry about the details and simply
think of these as some given constants.} and
$F(x)$ is the force applied per unit length at position $x$.  The situation
we are interested in is when the force is applied at a single point as in
\figurevref{lt:beambendingfig}.

\begin{myfig}
\capstart
\inputpdft{beam-bending}
\caption{Three-point bending.\label{lt:beambendingfig}}
\end{myfig}

The equation becomes
\begin{equation*}
EI \frac{d^4 y}{dx^4} = -F \delta(x-a) ,
\end{equation*}
where $x=a$ is the point where the mass is applied.  The constant $F$ is the force
applied and the minus sign indicates that the force is downward, that is, in the
negative $y$ direction.  The end points of the
beam satisfy the conditions,
\begin{align*}
& y(0) = 0, \qquad y''(0) = 0, \\
& y(L) = 0, \qquad y''(L) = 0.
\end{align*}
See \sectionref{sec:appeig} for further information about endpoint
conditions applied to beams.


\begin{example} \label{lt:examplebeam}
Suppose that length of the beam is 2, and $EI=1$ for
simplicity.  Further suppose that the force $F=1$ is applied at $x=1$.
That is, we have the equation
\begin{equation*}
\frac{d^4 y}{dx^4} = -\delta(x-1) ,
\end{equation*}
and the endpoint conditions are
\begin{equation*}
y(0) = 0, \qquad y''(0) = 0, \qquad
y(2) = 0, \qquad y''(2) = 0.
\end{equation*}

We could integrate, but using the Laplace transform
is even easier.
We apply the transform
in the $x$ variable rather than the $t$ variable.  Let us again denote the
transform of $y(x)$ as $Y(s)$.
\begin{equation*}
s^4Y(s)-s^3y(0)-s^2y'(0)-sy''(0)-y'''(0)
= -e^{-s}.
\end{equation*}
We notice that $y(0) = 0$ and $y''(0) = 0$.  Let us
call $C_1 = y'(0)$ and $C_2=y'''(0)$.
We solve for $Y(s)$,
\begin{equation*}
Y(s) = \frac{-e^{-s}}{s^4} + \frac{C_1}{s^2}+ \frac{C_2}{s^4} .
\end{equation*}
We take the inverse Laplace transform utilizing the 
second shifting property \eqref{ltd:sseq} to take the inverse of the first
term.
\begin{equation*}
y(x) = \frac{-{(x-1)}^3}{6} u(x-1) + C_1 x + \frac{C_2}{6} x^3 .
\end{equation*}
We still need to apply two of the endpoint conditions.  As the conditions
are at $x=2$ we can simply replace $u(x-1) = 1$ when taking
the derivatives.  Therefore,
\begin{equation*}
0 = y(2) = \frac{-{(2-1)}^3}{6} + C_1 (2) + \frac{C_2}{6} 2^3 =
\frac{-1}{6} + 2 C_1 + \frac{4}{3} C_2 ,
\end{equation*}
and
\begin{equation*}
0 = y''(2) = \frac{-3\cdot 2 \cdot (2-1)}{6} + \frac{C_2}{6} 3\cdot 2 \cdot 2
 = -1 + 2 C_2 .
\end{equation*}
Hence $C_2 = \frac{1}{2}$ and solving for $C_1$ using the first
equation we obtain
$C_1 = \frac{-1}{4}$.  Our solution for the beam deflection is
\begin{equation*}
y(x) = \frac{-{(x-1)}^3}{6} u(x-1) - \frac{x}{4} + \frac{x^3}{12} .
\end{equation*}
\end{example}

\subsection{Exercises}

\begin{exercise}
Solve (find the impulse response)
$x'' + x' + x = \delta(t)$, $x(0) = 0$, $x'(0)=0$.
\end{exercise}

\begin{exercise}
Solve (find the impulse response)
$x'' + 2 x' + x = \delta(t)$, $x(0) = 0$, $x'(0)=0$.
\end{exercise}

\begin{exercise}
A pulse can come later and can be bigger.
Solve 
$x'' + 4 x = 4\delta(t-1)$, $x(0) = 0$, $x'(0)=0$.
\end{exercise}

\begin{exercise}
Suppose that $f(t)$ and $g(t)$ are differentiable functions
(and the derivatives are continuous)
and suppose that $f(t) = g(t) = 0$ for all $t \leq 0$.  Show that
\begin{equation*}
(f * g)'(t) = (f' * g)(t) = (f * g')(t) .
\end{equation*}
\end{exercise}

\begin{exercise}
Suppose that $L x = \delta(t)$, $x(0) = 0$, $x'(0) = 0$, has the solution
$x(t) = t e^{-t}$ for $t > 0$.  Find the solution to
$Lx = t^2$, $x(0) = 0$, $x'(0) = 0$ for $t > 0$.
\end{exercise}

\begin{exercise}
Compute
${\mathcal{L}}^{-1} \left\{ \frac{s^2+s+1}{s^2} \right\}$.
\end{exercise}

\begin{exercise}[challenging]
Solve \exampleref{lt:examplebeam} via integrating 4 times in the $x$ variable.
\end{exercise}

\begin{exercise}
Suppose we have a beam of length $1$ simply supported at the ends and
suppose that force $F=1$ is applied at $x=\nicefrac{3}{4}$ in the downward
direction.  Suppose that $EI=1$ for simplicity.  Find the beam deflection
$y(x)$.
\end{exercise}

\setcounter{exercise}{100}

\begin{exercise}
Solve (find the impulse response)
$x'' = \delta(t)$, $x(0) = 0$, $x'(0)=0$.
\end{exercise}
\exsol{%
$x(t) = t$
}

\begin{exercise}
Solve (find the impulse response)
$x' + a x = \delta(t)$, $x(0) = 0$, $x'(0)=0$.
\end{exercise}
\exsol{%
$x(t) = e^{-at}$
}
%$sX+aX = 1$
%$X = \frac{1}{s+a}$

\begin{exercise}
Suppose that $L x = \delta(t)$, $x(0) = 0$, $x'(0) = 0$, has the solution
$x(t) = e^t \sin(t)$ for $t > 0$.  Find (in closed form) the solution to
$Lx = e^t$, $x(0) = 0$, $x'(0) = 0$ for $t > 0$.
\end{exercise}
\exsol{%
$x(t) = (e^t) * (e^t \sin(t) ) = e^t (1-\cos(t))$
}

\begin{exercise}
Compute
${\mathcal{L}}^{-1} \left\{ \frac{s^2}{s^2+1} \right\}$.
\end{exercise}
\exsol{%
$\delta(t) - \sin(t)$
}
%1 - 1/(s^2+1)

\begin{exercise}
Compute
${\mathcal{L}}^{-1} \left\{ \frac{3 s^2 e^{-s} + 2}{s^2} \right\}$.
\end{exercise}
\exsol{%
$3 \delta(t-1) + 2 t$
}

%%%%%%%%%%%%%%%%%%%%%%%%%%%%%%%%%%%%%%%%%%%%%%%%%%%%%%%%%%%%%%%%%%%%%%%%%%%%%%

\sectionnewpage
\section{Solving PDEs with the Laplace transform}
\label{laplacepde:section}

\sectionnotes{1--1.5 lecture, can be skipped}

The Laplace transform comes from the same family of transforms as does the
Fourier series\footnote{There is also a Fourier transform
on the real line that looks sort of like the Laplace transform.},
which we used in \chapterref{FS:chapter} to solve partial differential
equations (PDEs).
It is therefore not surprising that we can also solve PDEs with the Laplace
transform.

Given a PDE in two independent variables $x$ and $t$,  we use the
Laplace transform on one of the variables (taking the transform
of everything in sight), and derivatives in that variable become
multiplications by the transformed variable $s$.
The PDE becomes an ODE, which we solve.
Afterwards, we invert the transform to find a solution to the original
problem.
It is best to see the procedure on an example.

\begin{example}
Consider the first order PDE
\begin{equation*}
y_t = - \alpha y_x, \qquad \text{for } x > 0, \enspace t > 0,
\end{equation*}
with side conditions
\begin{equation*}
y(0,t) = C, \qquad y(x,0) = 0 .
\end{equation*}
We will assume $\alpha > 0$ is a constant.
This equation is 
called the \emph{\myindex{convection equation}} or sometimes
the \emph{\myindex{transport equation}}, and it already made
an appearance in \sectionref{fopde:section}, with different conditions.
See \figureref{lt:half-infinite-goo-river} for a diagram of the setup.

\begin{mywrapfig}{3.1in}
\capstart
\inputpdft{half-infinite-goo-river}
\caption{Transport equation on a half line.\label{lt:half-infinite-goo-river}}
\end{mywrapfig}

A physical setup of this equation is a river of solid goo,
as we do not want anything to diffuse.  The function
$y$ is the concentration of
some toxic substance\footnote{It's a river of goo already,
we're not hurting the environment much more.}.
The variable $x$ denotes position where $x=0$
is the location of a factory spewing the toxic substance into the
river.  The toxic substance flows into the river so that at $x=0$ the
concentration is always $C$.  We wish to see what happens past the factory,
that is at $x > 0$.  Let $t$ be the time, and assume
the factory started operations at $t=0$, so that at $t=0$ the river is just
pure goo.

Consider a function of two variables $y(x,t)$.
Let us fix $x$ and transform the $t$ variable.
For convenience, we treat the transformed $s$
variable as a parameter, since there are no derivatives in $s$.
That is, we write $Y(x)$ for the transformed function,
and treat it as a function of $x$, leaving $s$ as a parameter.
\begin{equation*}
Y(x)
= {\mathcal L} \bigl\{ y(x,t) \bigr\}
= \int_0^\infty y(x,t) e^{-st} \,dt .
\end{equation*}
The transform of a derivative with respect to $x$ is just differentiating 
the transformed function:
\begin{equation*}
{\mathcal L} \bigl\{ y_x(x,t) \bigr\} =
\int_0^\infty y_x(x,t) e^{-st} \,dt
=
\frac{d}{dx} \left[\int_0^\infty y(x,t) e^{-st} \,dt \right]
=
Y'(x) .
\end{equation*}
To transform the derivative in $t$ (the variable being transformed),
we use the rules from \sectionref{transformsofders:section}:
\begin{equation*}
{\mathcal L} \bigl\{ y_t(x,t) \bigr\} 
=
sY(x) - y(x,0) .
\end{equation*}

In our specific case,
$y(x,0)=0$, and so
${\mathcal L} \bigl\{ y_t(x,t) \bigr\} = sY(x)$.  We transform the equation
to find
\begin{equation*}
sY(x) = -\alpha Y'(x) .
\end{equation*}
This ODE needs an initial condition.  The initial condition is the
other side condition of the PDE\@, the one that depends on $x$.  Everything
is transformed, so we must also transform this condition
\begin{equation*}
Y(0) = 
{\mathcal L} \bigl\{ y(0,t) \bigr\} 
=
{\mathcal L} \bigl\{ C \bigr\} 
=
\frac{C}{s} .
\end{equation*}

We solve the ODE problem $sY(x) = -\alpha Y'(x)$, $Y(0) = \frac{C}{s}$, to find
\begin{equation*}
Y(x) = \frac{C}{s} e^{-\frac{s}{\alpha} x} .
\end{equation*}
We are not done, we have $Y(x)$, but we really want $y(x,t)$.  We 
transform the $s$ variable back to $t$.
Let 
\begin{equation*}
u(t) = \begin{cases} 0 & \text{if $t < 0$}, \\ 1 & \text{otherwise}
\end{cases}
\end{equation*}
be the Heaviside function.
As
\begin{equation*}
{\mathcal L} \bigl\{ u(t-a) \bigr\} =
\int_0^\infty u(t-a) \, e^{-st} \,dt
=
\int_a^\infty e^{-st} \,dt
=
\frac{e^{-as}}{s} ,
\end{equation*}
then
\begin{equation*}
y(x,t) = {\mathcal L}^{-1} \left\{
\frac{C}{s} e^{-\frac{s}{\alpha} x}
\right\}
=
C u\bigl(t-\nicefrac{x}{\alpha}\bigr) .
\end{equation*}
In other words,
\begin{equation*}
y(x,t) =
\begin{cases}
0 & \text{if $t < \nicefrac{x}{\alpha}$}, \\
C & \text{otherwise.}
\end{cases}
\end{equation*}
See \figurevref{lt:half-infinite-goo-river-wavefront} for a diagram of this
solution.  The line of slope $\nicefrac{1}{\alpha}$ indicates
the wavefront of the toxic substance in the picture as it is leaving the
factory.
What the equation does is simply move the initial condition to the right at
speed $\alpha$.

\begin{myfig}
\capstart
\inputpdft{half-infinite-goo-river-wavefront}
\caption{Wavefront of toxic substance is a line of slope
$\nicefrac{1}{\alpha}$.\label{lt:half-infinite-goo-river-wavefront}}
\end{myfig}


Shhh\ldots $y$ is not differentiable, it is not even continuous
(nobody ever seems to notice).  How could we plug something that's not
differentiable into the equation?
Well, just think of a differentiable function very very close to $y$.
Or, if you recognize the derivative of the Heaviside function as the delta
function, then all is well too:
\begin{equation*}
y_t (x,t) = \frac{\partial}{\partial t} \left[
C  u\bigl(t-\nicefrac{x}{\alpha}\bigr)
\right]
=
C  u'\bigl(t-\nicefrac{x}{\alpha}\bigr)
=
C \delta\bigl(t-\nicefrac{x}{\alpha}\bigr)
\end{equation*}
and
\begin{equation*}
y_x (x,t) = \frac{\partial}{\partial x} \left[
C  u\bigl(t-\nicefrac{x}{\alpha}\bigr)
\right]
=
- \frac{C}{\alpha}  u'\bigl(t-\nicefrac{x}{\alpha}\bigr)
=
- \frac{C}{\alpha} \delta\bigl(t-\nicefrac{x}{\alpha}\bigr) .
\end{equation*}
So $y_t = - \alpha y_x$.
\end{example}

Laplace equation is very good with constant coefficient equations.  One
advantage of Laplace is that it easily handles
nonhomogeneous side conditions.
Let us try a more complicated example.

\begin{example}
Consider
\begin{align*}
& y_t + y_x + y = 0, \qquad \text{for } x > 0, \enspace t > 0,
\\
& y(0,t) = \sin(t), \qquad y(x,0) = 0 .
\end{align*}

Again, we transform $t$, and we write $Y(x)$ for the
transformed function.  As $y(x,0) = 0$, we find
\begin{equation*}
sY(x) + Y'(x) + Y(x) = 0, \qquad 
Y(0) = \frac{1}{s^2+1} .
\end{equation*}
The solution of the transformed equation is
\begin{equation*}
Y(x) =
\frac{1}{s^2+1} e^{-(s+1) x}
=
\frac{1}{s^2+1} e^{-xs}
e^{-x}
.
\end{equation*}
Using the second shifting property \eqref{ltd:sseq}
and linearity of the transform,
we obtain the solution
\begin{equation*}
y(x,t) 
=
e^{-x}
\sin(t-x)
u(t-x) .
\end{equation*}
\end{example}

We can also detect when the problem is
\emph{\myindex{ill-posed}}
in the sense that it has no solution.
Let us change the equation to
\begin{align*}
& -y_t + y_x = 0, \qquad \text{for } x > 0, \enspace t > 0,
\\
& y(0,t) = \sin(t), \qquad y(x,0) = 0 .
\end{align*}
Then the problem
has no solution.  First, let us see this in the language of
\sectionref{fopde:section}.
The characteristic curves are $t=-x+C$.  If $\tau$ is
the characteristic coordinate, then we find the equation $y_\tau = 0$
along the curve,
meaning a solution is constant along characteristic curves.
But these curves intersect both the $x$-axis and the $t$-axis.
For example,
the curve $t=-x+1$ intersects at $(1,0)$ and $(0,1)$.  The solution is
constant along the curve so $y(1,0)$ should equal $y(0,1)$.  But
$y(1,0) = 0$ and $y(0,1) = \sin(1) \not= 0$.
See \figurevref{lt:half-infinite-ill-posed}.

\begin{myfig}
\capstart
\inputpdft{half-infinite-ill-posed}
\caption{Ill-posed problem.\label{lt:half-infinite-ill-posed}}
\end{myfig}

Now consider the transform.  The transformed problem is
\begin{equation*}
-sY(x) + Y'(x) = 0, \enspace Y(0) = \frac{1}{s^2+1} ,
\end{equation*}
and the solution ought to be
\begin{equation*}
Y(x) = \frac{1}{s^2+1} e^{sx} .
\end{equation*}
Importantly, this Laplace transform does not decay to zero at infinity!  That is,
since $x > 0$ in the region of interest, then
\begin{equation*}
\lim_{s \to \infty}
\frac{1}{s^2+1} e^{sx}
= \infty \not= 0 .
\end{equation*}
It almost
looks as if we could use the shifting property, but notice that the shift
is in the wrong direction.

\medskip

Of course, we need not restrict ourselves to first order equations, although
the computations become more involved for higher order equations.

\begin{example}
Let us use Laplace for the following problem:
\begin{align*}
& y_t = y_{xx}, \qquad 0 < x < \infty, \enspace t > 0,\\
& y_x(0,t) = f(t), \\
%& \text{for every $t$, $y(x,t)$ is bounded: } \lvert y(x,t) \rvert < M \\
%FIXME: perhaps limit as $x \to \infty$?
& y(x,0) = 0 .
\end{align*}
Really, we also impose other conditions on the solution so that, for
example,
the Laplace transform exists.
For instance, we might impose that $y$ is
bounded for each fixed time $t$.

Transform the equation in the $t$ variable to find
\begin{equation*}
sY(x) = Y''(x) .
\end{equation*}
The general solution to this ODE is
\begin{equation*}
Y(x) = A e^{\sqrt{s}x} + B e^{-\sqrt{s} x} .
\end{equation*}
First $A=0$, since otherwise $Y$ does not decay to zero as $s \to \infty$.

Now consider the boundary condition.
Transform $Y'(0) = F(s)$ and so $-\sqrt{s}B = F(s)$.
In other words,
\begin{equation*}
Y(x) = - F(s) \frac{1}{\sqrt{s}} e^{-\sqrt{s} x} .
\end{equation*}

If we look up the inverse transform in a table such as the one in
\Appendixref{laplacelist:appendix} (or we spend the afternoon
doing calculus), we find 
\begin{equation*}
{\mathcal L}^{-1}\left[e^{-\sqrt{s} x}\right] =
\frac{x}{\sqrt{4\pi t^3}} e^{\frac{-x^2}{4t}} ,
\end{equation*}
or
\begin{equation*}
{\mathcal L}^{-1}\left[\frac{1}{\sqrt{s}} e^{-\sqrt{s} x}\right] =
\frac{1}{\sqrt{\pi t}} e^{\frac{-x^2}{4t}} .
\end{equation*}
So
\begin{equation*}
y(x,t) =
{\mathcal L}^{-1} \left[
F(s) e^{-\sqrt{s} x}\right]
=
\int_0^t
f(\tau) 
\frac{1}{\sqrt{\pi {(t-\tau)}}} e^{\frac{-x^2}{4(t-\tau)}} \, d\tau .
\end{equation*}
\end{example}

Laplace can solve problems where separation of variables fails.
Laplace does not mind nonhomogeneity, but it is essentially only useful for
constant coefficient equations.

\subsection{Exercises}

\begin{exercise}
Solve
\begin{align*}
& y_t + y_x = 1, \qquad 0 < x < \infty, \enspace t > 0,
\\
& y(0,t) = 1, \quad y(x,0) = 0 .
\end{align*}
\end{exercise}

\begin{exercise}
Solve
\begin{align*}
& y_t + \alpha y_x = 0, \qquad 0 < x < \infty, \enspace t > 0,
\\
& y(0,t) = t, \quad y(x,0) = 0 .
\end{align*}
\end{exercise}

\begin{exercise}
Solve
\begin{align*}
& y_t + 2 y_x = x+t, \qquad 0 < x < \infty, \enspace t > 0,
\\
& y(0,t) = 0, \quad y(x,0) = 0 .
\end{align*}
\end{exercise}

\begin{exercise}
For an $\alpha > 0$, solve
\begin{align*}
& y_t + \alpha y_x + y = 0, \qquad 0 < x < \infty, \enspace t > 0,
\\
& y(0,t) = \sin(t), \quad y(x,0) = 0 .
\end{align*}
\end{exercise}

\begin{exercise}
Find the corresponding ODE problem for $Y(x)$, after transforming the $t$
variable
\begin{align*}
& y_{tt} + 3y_{xx} + y_{xt} + 3 y_x + y = \sin(x) + t, \qquad 0 < x < 1, \enspace t > 0,
\\
& y(0,t) = 1, \quad y(1,t) =  t, \quad y(x,0) = 1-x, \quad y_t(x,0) = 1 .
\end{align*}
Do not solve the problem.
\end{exercise}

\begin{exercise}
Write down a solution to
\begin{align*}
& y_t = y_{xx}, \qquad 0 < x < \infty, \enspace t > 0,\\
& y_x(0,t) = e^{-t}, \quad y(x,0) = 0 ,
\end{align*}
as an definite integral (convolution).
\end{exercise}

\begin{exercise}
Use the Laplace transform in $t$ to solve
\begin{align*}
& y_{tt} = y_{xx}, \qquad -\infty < x < \infty, \enspace t > 0,\\
& y_t(x,0) = \sin(x), \quad  y(x,0) = 0 .
\end{align*}
Hint: Note that
$e^{sx}$ does not go to zero as $s \to \infty$ for positive
$x$, and 
$e^{-sx}$ does not go to zero as $s \to \infty$ for negative $x$.
\end{exercise}

\setcounter{exercise}{100}

\begin{exercise}
Solve
\begin{align*}
& y_t + y_x = 1, \qquad 0 < x < \infty, \enspace t > 0,
\\
& y(0,t) = 0, \quad y(x,0) = 0 .
\end{align*}
\end{exercise}
\exsol{%
$y=(x-t)u(t-x)+t$
}

\begin{exercise}
For a $c > 0$, solve
\begin{align*}
& y_t + y_x + c y = 0, \qquad 0 < x < \infty, \enspace t > 0,
\\
& y(0,t) = \sin(t), \quad y(x,0) = 0 .
\end{align*}
\end{exercise}
\exsol{%
$y=e^{-cx}\sin(t-x)u(t-x)$
}

\begin{exercise}
Find the corresponding ODE problem for $Y(x)$, after transforming the $t$
variable
\begin{align*}
& y_{tt} + 3y_{xx} + y = x+t, \qquad -1 < x < 1, \enspace t > 0,
\\
& y(-1,t) = 0, \quad y(1,t) = 0, \quad y(x,0) = (1-x^2) , \quad y_t(x,0) = 0.
\end{align*}
Do not solve the problem.
\end{exercise}
\exsol{%
$s^2Y(x) - s(1-x^2) + 3Y''(x) + Y(x) = \frac{x}{s} + \frac{1}{s^2}, \quad
Y(-1) = 0, \quad Y(1)=0$.
}

\begin{exercise}
Use the Laplace transform in $t$ to solve
\begin{align*}
& y_{tt} = y_{xx}, \qquad -\infty < x < \infty, \enspace t > 0,\\
& y_t(x,0) = x^2, \quad  y(x,0) = 0 .
\end{align*}
Hint: Note that
$e^{sx}$ does not go to zero as $s \to \infty$ for positive
$x$, and 
$e^{-sx}$ does not go to zero as $s \to \infty$ for negative $x$.
\end{exercise}
\exsol{%
$y=tx^2+\frac{t^3}{3}$
}
