\documentclass[10pt,aspectratio=169]{beamer}

% All the boilerplate is in deslides.sty
\usepackage{deslides}

\author{Ji\v{r}\'i Lebl}

\institute[OSU]{%
Oklahoma State University%
%Departemento pri Matematiko de Oklahoma {\^S}tata Universitato%
}

\title{24. The Laplace transform, part 2\\(Notes on Diffy Qs, 6.1)}

\date{}

\begin{document}

\begin{frame}
\titlepage

%\bigskip

\begin{center}
The textbook: \url{https://www.jirka.org/diffyqs/}
\end{center}
\end{frame}

\begin{frame}
As we said, not all functions have Laplace transforms.  But many do.

\medskip
\pause

A function $f(t)$ is of
\emph{\myindex{exponential order}} as $t \to \infty$ if
there are constants $M$ and $c$ such that
\begin{equation*}
\lvert f(t) \rvert \leq M e^{ct} ,
\qquad
\text{for sufficiently large $t$} .
\end{equation*}
\pause
One way to check this condition is to compute
\begin{equation*}
\lim_{t\to \infty} \frac{f(t)}{e^{ct}} .
\end{equation*}
\pause
If the limit exists and is finite (usually zero), then $f(t)$ is of
exponential order.

\medskip
\pause

A sum of exponential order functions is of exponential order.

\pause
\medskip

Every polynomial is of exponential order.

\pause
\medskip

$e^{at}$ is of exponential order for any $a$.

\pause
\medskip

$e^{t^2}$ is \textbf{not} of exponential order for any $a$.

\end{frame}

\begin{frame}

\textbf{Theorem} (Existence)\textbf{:}
Let $f(t)$ be continuous on the interval $[0,\infty)$ and of exponential order for a certain
constant $c$.  Then $F(s) = \mathcal{L} \bigl\{ f(t) \bigr\}$ is defined for
all $s > c$.

\medskip
\pause

The theorem is not difficult to see:

\medskip

Suppose
$\lvert f(t) \rvert \leq M e^{ct}$ for all $t > 0$.
Let $s > c$, or in other words $(s-c) > 0$.

\medskip
\pause

By the comparison theorem from calculus, the improper integral
$\mathcal{L} \bigl\{ f(t) \bigr\}$ exists because the following integral exists
\begin{equation*}
\int_0^\infty e^{-st} ( M e^{ct} ) \,dt
\pause
=
M \int_0^\infty e^{-(s-c)t} \,dt
\pause
= M \left[ \frac{e^{-(s-c)t}}{-(s-c)}
\right]_{t=0}^\infty
\pause
= \frac{M}{s-c} .
\end{equation*}
\pause

Moreover,
if $f$ is of exponential order, then
\begin{equation*}
\lim_{s\to\infty} F(s) = 0 .
\end{equation*}
\end{frame}

\begin{frame}

\textbf{Theorem} (Uniqueness)\textbf{:}
Let $f(t)$ and $g(t)$ be continuous and of exponential order.
Suppose that there exists a constant $C$,
such that $F(s) = G(s)$ for all $s > C$.
Then $f(t) = g(t)$ for all $t \geq 0$.

\medskip
\pause

The theorems also hold for piecewise continuous functions:
functions with a discrete set of jump discontinuities like the
Heaviside.

\medskip

However, for uniqueness, we can only conclude
$f(t) = g(t)$ outside of discontinuities.

\medskip
\pause

Uniqueness says it makes sense to invert the transform:
Given $F(s)$, find the unique $f(t)$.

\medskip
\pause

Suppose $F(s) = \mathcal{L} \bigl\{ f(t) \bigr\}$.
Define the
\emph{\myindex{inverse Laplace transform}} as
\begin{equation*}
{\mathcal{L}}^{-1} \bigl\{ F(s) \bigr\} \overset{\text{def}}{=} f(t) .
\end{equation*}

\pause

There is an integral formula for the inverse, but it is complicated.

We will be satisfied with using the table of transforms we computed.

\medskip
\pause

\textbf{Example:}
Via the table,
\begin{equation*}
{\mathcal{L}}^{-1} \left\{ \frac{1}{s+1} \right\} = 
e^{-t} .
\end{equation*}

\end{frame}

\begin{frame}
The inverse is also linear:
\begin{equation*}
{\mathcal{L}}^{-1} \bigl\{ A F(s) + B G(s) \bigr\} =
A {\mathcal{L}}^{-1} \bigl\{ F(s) \bigr\} +
B {\mathcal{L}}^{-1} \bigl\{ G(s) \bigr\} .
\end{equation*}
Of course, also
${\mathcal{L}}^{-1} \bigl\{ A F(s) \bigr\} =
A {\mathcal{L}}^{-1} \bigl\{ F(s) \bigr\}$.

\medskip
\pause

\textbf{Example:}
Let us find the inverse Laplace transform of
$F(s) = \dfrac{s^2+s+1}{s^3+s}$.

\medskip
\pause

Use the \emph{method of partial fractions} to split up $F(s)$:
\begin{equation*}
\frac{s^2+s+1}{s^3+s}
\pause
=
\frac{s^2+s+1}{s(s^2+1)}
\pause
=
\frac{A}{s} + 
\frac{Bs+C}{s^2+1} .
\end{equation*}
\pause
Clear denominators to find
$s^2+s+1 = A(s^2+1) + s(Bs+C)$.
\pause
Expand and equate coefficients to find
$A+B = 1$, $C=1$, $A=1$, and $B=0$.
\pause
That is,
\begin{equation*}
F(s) =
\frac{s^2+s+1}{s^3+s}
=
\frac{1}{s} +
\frac{1}{s^2+1} .
\end{equation*}
\pause
By linearity,
\begin{equation*}
{\mathcal{L}}^{-1} \left\{ 
\frac{s^2+s+1}{s^3+s} \right\}
\pause
=
{\mathcal{L}}^{-1} \left\{ 
\frac{1}{s} \right\} 
+
{\mathcal{L}}^{-1} \left\{ 
\frac{1}{s^2+1} \right\}
\pause
=
1 + 
\sin t .
\end{equation*}
\end{frame}

\begin{frame}

A useful property is the 
\emph{shifting property} (or \emph{first shifting property}).
If $F(s)$ is the Laplace transform of $f(t)$, then
\begin{equation*}
\mathcal{L} \bigl\{ e^{-at} f(t) \bigr\} = F(s+a) .
\end{equation*}

\pause
This property is useful if the denominator is more complicated, e.g.,
an irreducible quadratic.  Then complete the square
 ${(s+a)}^2+b$
and use
the shifting property.

\medskip
\pause

\textbf{Example:}
Find
$\displaystyle {\mathcal{L}}^{-1} \left\{ \frac{1}{s^2+4s+8} \right\}$.

\medskip
\pause

Complete the square in the denominator:
$s^2+4s+8 = {(s+2)}^2+4$.

\medskip
\pause

Next find
\begin{equation*}
{\mathcal{L}}^{-1} \left\{ \frac{1}{s^2+4} \right\}
\pause
=
\frac{1}{2}
{\mathcal{L}}^{-1} \left\{ \frac{2}{s^2+2^2} \right\}
\pause
=
\frac{1}{2} \sin (2t) .
\end{equation*}
\pause
Now put it together with the shifting property
\begin{equation*}
{\mathcal{L}}^{-1} \left\{ \frac{1}{s^2+4s+8} \right\}
\pause
= 
{\mathcal{L}}^{-1} \left\{ \frac{1}{{(s+2)}^2+4} \right\}
\pause
=
\frac{1}{2}\,e^{-2t} \sin (2t) .
\end{equation*}
\end{frame}

\begin{frame}
We often try to 
apply the inverse Laplace transform to
rational functions:
\begin{equation*}
\frac{F(s)}{G(s)}
\end{equation*}
where $F(s)$ and $G(s)$ are polynomials.

\medskip
\pause

If $\frac{F(s)}{G(s)}$ is the Laplace transform of an exponential order function,
it goes to zero as $s \to \infty$:

\pause
So the degree of $F(s)$
is smaller than that of $G(s)$.
That is,
$\frac{F(s)}{G(s)}$ is a \emph{proper rational function}.

\medskip
\pause

For proper rational functions, partial fractions method applies without
polynomial division, and the techniques above always work.

\medskip
\pause

Though for partial fractions, you still need to factor the denominator,
which can be hard (finding roots).

\end{frame}


\end{document}
