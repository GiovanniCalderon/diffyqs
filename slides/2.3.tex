\documentclass[10pt,aspectratio=169]{beamer}

% All the boilerplate is in deslides.sty
\usepackage{deslides}

\author{Ji\v{r}\'i Lebl}

\institute[OSU]{%
Oklahoma State University%
%Departemento pri Matematiko de Oklahoma {\^S}tata Universitato%
}

\title{16. Higher order linear ODEs\\(Notes on Diffy Qs, 2.3)}

\date{}

\begin{document}

\begin{frame}
\titlepage

%\bigskip

\begin{center}
The textbook: \url{https://www.jirka.org/diffyqs/}
\end{center}
\end{frame}

\begin{frame}
Let us consider linear equations of order higher than 2 --- Mostly the same
idea.

\pause

\begin{theorem}[Superposition]
If $y_1$, $y_2$, \ldots, $y_n$ are solutions of the
homogeneous equation
\[
y^{(n)} + p_{n-1}(x)y^{(n-1)} + \cdots + p_1(x) y' + p_0(x) y = 0 ,
\]
then the \emph{linear combination}
\[
y(x) = C_1 y_1(x) + C_2 y_2(x) + \cdots + C_n y_n(x) 
\]
is also a solution
for arbitrary constants $C_1, C_2, \ldots, C_n$.
\end{theorem}

\end{frame}

\begin{frame}

\begin{theorem}[Existence and uniqueness]
Suppose $p_0$ through $p_{n-1}$, and $f$ are continuous functions
on some interval $I$,
$a$ is a number in $I$,
and $b_0, b_1, \ldots, b_{n-1}$ are constants.
Then the equation
\[
y^{(n)} + p_{n-1}(x)y^{(n-1)} + \cdots + p_1(x) y' + p_0(x) y = f(x) 
\]
has exactly one solution $y(x)$ defined on the same interval $I$
satisfying the initial conditions
\begin{equation*}
y(a) = b_0, \quad y'(a) = b_1, \quad \ldots, \quad y^{(n-1)}(a) = b_{n-1} .
\end{equation*}
\end{theorem}
\end{frame}

\begin{frame}

To use superposition, we need enough \emph{linearly independent} solutions.

\medskip
\pause

The functions $y_1$, $y_2$, \ldots, $y_n$ are \emph{linearly independent} if
the equation
\[
c_1 y_1 + c_2 y_2 + \cdots + c_n y_n = 0 
\]
has only the trivial solution $c_1 = c_2 = \cdots = c_n = 0$.

\medskip
\pause

If there is a solution where, e.g.,
$c_1 \not= 0$, then we
can solve for $y_1$ as a linear combination of the others.
\pause
The functions are then
\emph{linearly dependent}.

\medskip
\pause

\textbf{Example:} Show that $e^x, e^{2x}, e^{3x}$ are linearly independent.

\medskip
\pause

Consider
\[
c_1 e^x + c_2 e^{2x} + c_3 e^{3x} = 0.
\]
\pause
Write $z = e^x$
\pause
\wthus
$z^2 = e^{2x}$ and $z^3 = e^{3x}$ \pause \wthus
\[
c_1 z + c_2 z^2 + c_3 z^3 = 0.
\]
\pause
This polynomial must be zero for any positive number $z=e^{x}$
\pause
\wthus
it is identically zero

\pause
\thus
\quad
$c_1 = c_2 = c_3 = 0$
\pause
\wthus
the functions are linearly independent.
\end{frame}

\begin{frame}

Let's show that $e^x, e^{2x}, e^{3x}$ are linearly independent in another
way:

\medskip
\pause

Start with
\[
c_1 e^x + c_2 e^{2x} + c_3 e^{3x} = 0.
\]
\pause
Divide by $e^{3x}$:
\[
c_1 e^{-2x} + c_2 e^{-x} + c_3 = 0.
\]
\pause
True for all $x$, so let $x \to \infty$
\pause
\wthus $c_3 = 0$.

\medskip
\pause

So $c_1 e^x + c_2 e^{2x} = 0$.
\pause Rinse, repeat!

\medskip
\pause

A third way:  Suppose
\[
c_1 e^x + c_2 e^{2x} + c_3 e^{3x} = 0.
\]
This has to be true for all $x$, so plug in $x=0$, $x=1$, and $x=2$.

\medskip
\pause

We must have
\[
c_1 + c_2 + c_3 = 0
\quad\text{and}\quad
c_1 e + c_2 e^{2} + c_3 e^{3} = 0
\quad\text{and}\quad
c_1 e^2 + c_2 e^{4} + c_3 e^{6} = 0.
\]
\pause
Solving the three equations yields $c_1=c_2=c_3=0$.

\medskip
\pause

A so-called \emph{Wronskian} can also be used, but let's skip it.
\pause

The main thing to understand is the meaning of linear independence!

\end{frame}

\begin{frame}
\textbf{Example:}
The functions $e^x$, $e^{-x}$, and $\cosh x$ are linearly
dependent:
\pause
\[
\cosh x = \frac{e^x + e^{-x}}{2} 
\qquad
\text{or}
\qquad
2 \cosh x - e^x - e^{-x} = 0.
\]

\medskip
\pause

Note: You must consider all the functions at once.
The functions
$e^x$ and $e^{-x}$ are linearly independent.

\pause
\medskip

\begin{theorem}
If $y_1$, $y_2$, \ldots, $y_n$ are linearly independent solutions of the
homogeneous equation 
\[
y^{(n)} + p_{n-1}(x)y^{(n-1)} + \cdots + p_1(x) y' + p_0(x) y = 0 ,
\]
then the general solution can be written as
\begin{equation*}
y(x) = C_1 y_1(x) + C_2 y_2(x) + \cdots + C_n y_n(x) .
\end{equation*}
\end{theorem}
\end{frame}

\begin{frame}
\textbf{Example:}
Solve
~$y''' - 3 y'' - y' + 3y = 0$
~~subject to~~
$y(0) = 1$, $y'(0) = 2$, and $y''(0) = 3$.

\medskip
\pause

Try: $y = e^{rx}$:
\[
\underbrace{r^3 e^{rx}}_{y'''} - 3 \underbrace{r^2 e^{rx}}_{y''} -
\underbrace{r e^{rx}}_{y'} + 3 \underbrace{e^{rx}}_{y} = 0 .
\]
\pause
We divide by $e^{rx}$: \qquad
$r^3 - 3 r^2 - r + 3 = 0$.

\medskip
\pause

Find the roots! \pause By trial and error $r=-1,1,3$.

\medskip
\pause

So solutions are \quad $y_1 = e^{-x}$, \quad $y_2 = e^{x}$, \quad $y_3 =
e^{3x}$.

\medskip
\pause

The general solution is
\[
y = C_1 e^{-x} + C_2 e^{x} + C_3 e^{3x} .
\]
\pause
Initial conditions say
\[
1 = y(0)  = C_1 + C_2 + C_3 , \quad
\pause
2 = y'(0)  = -C_1 + C_2 + 3C_3 , \quad
\pause
3 = y''(0)  = C_1 + C_2 + 9C_3 .
\]
\pause
We solve these to find
$C_1 = -\nicefrac{1}{4}$, $C_2 = 1$, and $C_3 = \nicefrac{1}{4}$.

\medskip
\pause

So the particular solution to our problem is:
\[
y = \frac{-1}{4}\, e^{-x} + e^x + \frac{1}{4}\, e^{3x} .
\]
\end{frame}

\begin{frame}
The real trick is finding the roots.

\medskip
\pause

There are complicated formulas for degree
3 and 4 polynomials.

\medskip
\pause

There is \textbf{no} formula for degree 5 or higher.

\medskip
\pause

But there are always $n$ roots for an 
$n^{\text{th}}$ degree polynomial,
though they can be repeated, and they may be complex.

\medskip
\pause

One often uses a computer, but a good strategy to do it by hand is
to plug in some easy numbers to start with:
Start with $0$, then try $1$ and $-1$, then try other integers.

\medskip
\pause

E.g., for $r^3 - 3 r^2 - r + 3 = 0$, we find $r_1 = 1$ and $r_2 = -1$ are
roots by trying.

\pause
The third root $r_3$ is easy to find:
\pause
\[
r^3 - 3 r^2 - r + 3 = (r-r_1)(r-r_2)(r-r_3)
\]
\pause
So the constant term is
\[
3 = (-r_1)(-r_2)(-r_3) .
\]
\pause
So $3 = (-1)(1)(-r_3)$ or $r_3 = 3$.

\end{frame}

\begin{frame}
If a real root $r$
is repeated $k$ times, then we have solutions
\[
e^{rx}, \quad xe^{rx}, \quad x^2 e^{rx}, \quad \ldots, \quad x^{k-1} e^{rx} .
\]

\pause
\textbf{Example:}
Solve $y^{(4)} - 3 y''' + 3 y'' - y' =  0$.

\medskip
\pause

The characteristic equation is \quad $r^4 - 3r^3 + 3r^2 -r = 0$.

\medskip
\pause

By inspection \quad $r^4 - 3r^3 + 3r^2 -r = r{(r-1)}^3$.

\medskip
\pause

The roots are $r = 0, 1, 1, 1$. \quad ($r=1$ has multiplicity $3$)

\medskip
\pause

The general solution is
\[
y = \underbrace{(C_1 + C_2 x + C_3 x^2)\, e^x}_{\text{terms coming from }
r=1} + \underbrace{C_4}_{\text{from } r=0} .
\]

\end{frame}

\begin{frame}
Complex roots come in pairs $r = \alpha \pm i \beta$.

\medskip
\pause

If we have such a pair each repeated $k$ times,
the corresponding solution is
\[
( C_0 + C_1 x + \cdots + C_{k-1} x^{k-1} ) \, e^{\alpha x} \cos (\beta x)
+
( D_0 + D_1 x + \cdots + D_{k-1} x^{k-1} ) \, e^{\alpha x} \sin (\beta x) .
\]

\pause
\textbf{Example:}
Solve
\quad
$y^{(4)} - 4 y''' + 8 y'' - 8 y' + 4y = 0$.

\medskip
\pause

The characteristic equation is
\quad
$r^4 - 4 r^3 + 8 r^2 - 8 r + 4 = 0$

\pause
\thus
\quad
${(r^2-2r+2)}^2 = 0$
\pause
\wthus
${\bigl({(r-1)}^2+1\bigr)}^2 = 0$.

\medskip
\pause

Roots are $1 \pm i$, both with multiplicity $2$.

\medskip
\pause

The general solution is
\[
y = 
( C_1 + C_2 x ) \, e^{x} \cos x
+
( C_3 + C_4 x ) \, e^{x} \sin x .
\]
\end{frame}

\end{document}
