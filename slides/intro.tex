\documentclass[10pt,aspectratio=169]{beamer}

% All the boilerplate is in deslides.sty
\usepackage{deslides}

\author{Ji\v{r}\'i Lebl}

\institute[OSU]{%
Oklahoma State University%
%Departemento pri Matematiko de Oklahoma {\^S}tata Universitato%
}

\title{1. Syllabus: Notes on Diffy Qs\\Differential Equations for Engineers}

\date{}

\begin{document}

\begin{frame}
\titlepage

%\bigskip

\begin{center}
The textbook: \url{https://www.jirka.org/diffyqs/}
\end{center}
\end{frame}

\begin{frame}
First undergraduate course in differential equations, aimed at engineers,
typically at the end of the calculus sequence.

\pause
\medskip

Based on a free textbook, available as a PDF, HTML, or an inexpensive
paperback:

\medskip

Ji\v{r}\'i Lebl, \emph{Notes on Diffy Qs: Differential Equations for Engineers,}

\url{https://www.jirka.org/diffyqs/}

\pause
\medskip

I developed the book
to teach UIUC Math 285/286, UCSD Math 20D, and OSU Math 4233.

\pause
\medskip

It is similar to a course often taught with other books such as
those by
\emph{Edwards and Penney} or \emph{Boyce and DiPrima}.

\end{frame}

\begin{frame}
Syllabus (chapters in the book):

\begin{itemize}
\item\pause
Introduction to and classification of differential equations
\item\pause
First order equations
\item\pause
Higher order ODEs
\item\pause
Systems of ODEs
\item\pause
Fourier series and PDEs
\item\pause
More eigenvalue problems
\item\pause
The Laplace transform
\item\pause
Power series methods
\item\pause
Nonlinear systems
\end{itemize}
\end{frame}

\begin{frame}
Prerequisite:

\medskip

Basic sequence in calculus.

\medskip

For systems, some prior exposure to linear algebra will be useful.

\end{frame}

\end{document}
