\documentclass[10pt,aspectratio=169]{beamer}

% All the boilerplate is in deslides.sty
\usepackage{deslides}

\author{Ji\v{r}\'i Lebl}

\institute[OSU]{%
Oklahoma State University%
%Departemento pri Matematiko de Oklahoma {\^S}tata Universitato%
}

\title{20. Nonhomogeneous equations,\\part 2: variation of parameters\\(Notes on Diffy Qs, 2.5)}

\date{}

\begin{document}

\begin{frame}
\titlepage

%\bigskip

\begin{center}
The textbook: \url{https://www.jirka.org/diffyqs/}
\end{center}
\end{frame}

\begin{frame}
Trying to solve $Ly = f(x)$.

\medskip

Undetermined coefficients needs $f(x)$ and its derivatives
to be of the same form.

\medskip
\pause

What about $Ly = \tan x$, that is, $f(x) = \tan x$?

\medskip
\pause

$f'(x) = \sec^2 x$,

\pause
$f''(x) = 2\sec^2 x \, \tan x$,

\pause
$f'''(x) = 4 \sec^2 x \, \tan^2 x + 2 \sec^4 x$,

\pause
$f^{(4)}(x) = 8 \sec^2 x \, \tan^3 x + 16 \sec^4 x \, \tan x$,

\pause
$f^{(5)}(x) = 16\sec^2 x \, \tan^4 x + 88 \sec^4 x \tan^2 x + 16 \sec^6 x$,

\ldots

\medskip
\pause

No luck ...

\end{frame}

\begin{frame}
\emph{Variation of parameters} can handle $\tan x$ or any other $f(x)$ (but
computations can get tedious).

\medskip
\pause

The technique works for the general linear equations of any order,
but we stick to constant coefficient second order equations.

\medskip
\pause

Suppose $Ly= 0$ (associated homogeneous equation) has general
solution
\[
C_1 y_1 + C_2 y_2 ,
\]
where $C_1,C_2$ are constants.

\medskip
\pause

We look for a solution of $Ly=f(x)$ of the form
\[
y  = u_1 y_1 + u_2 y_2 ,
\]
where $u_1,u_2$ are functions.

\end{frame}

\begin{frame}
\textbf{Example:}
Solve $y''+y=\tan x$.

\medskip
\pause

The complementary solution is
\[
y_c = C_1 \cos x + C_2 \sin x .
\]
\pause
So let's look for a solution of the form
\[
y = u_1 \cos x + u_2 \sin x = u_1 y_1 + u_2 y_2 .
\]
\pause
Compute
\[
y' = (u_1' y_1 + u_2' y_2) + (u_1 y_1' + u_2 y_2').
\]
\pause
We have two unknown functions ($u_1,u_2$) and one condition ($Ly=f(x)$),
so we can still impose another condition.
\pause
Wouldn't it be convenient if
\[
(u_1' y_1 + u_2' y_2) = 0
\]
\pause
Yes?  So let's assume that!
\pause
Then
\[
y' = u_1 y_1' + u_2 y_2' .
\]

\end{frame}

\begin{frame}
Start with

\quad
$y' = u_1 y_1' + u_2 y_2'$.

\medskip
\pause

Differentiate

\quad
$y'' = (u_1' y_1' + u_2' y_2') + (u_1 y_1'' + u_2 y_2'')$

\medskip
\pause

$y_1$ and $y_2$ solve $y''+y = 0$, so $y_1'' = - y_1$ and $y_2'' = - y_2$.

\medskip
\pause

So

\quad
$y'' =
(u_1' y_1' + u_2' y_2') - (u_1 y_1 + u_2 y_2) 
\pause
=
(u_1' y_1' + u_2' y_2') - y
$

\medskip
\pause

In other words

\quad
$y'' + y = Ly = u_1' y_1' + u_2' y_2'$.

\medskip

\pause
Soooo ...  for $y$ to satisfy $Ly = f(x)$, we must have
$f(x) = u_1' y_1' + u_2' y_2'$.

\medskip
\pause

To summarize we need to solve the two imposed conditions:
\[
\boxed{~~
\begin{aligned}
& u_1' y_1 + u_2' y_2 = 0 ,\\
& u_1' y_1' + u_2' y_2' = f(x) .
\end{aligned}
~~}
\]

\end{frame}

\begin{frame}
In our case
\[
\begin{aligned}
u_1' \cos x + u_2' \sin x &= 0 ,\\
-u_1' \sin x + u_2' \cos x &= \tan x .
\end{aligned}
\]
\pause
Hence
\[
\begin{aligned}
u_1' \cos x \sin x + u_2' \sin^2 x & = 0 ,\\
-u_1' \sin x \cos x + u_2' \cos^2 x & = \tan x \cos x = \sin x .
\end{aligned}
\]
\pause
And thus
\[
\begin{aligned}
& u_2' \bigl(\sin^2 x + \cos^2 x\bigr) = \sin x , \\
& u_2' = \sin x , \\
& u_1' = \frac{- \sin^2 x}{\cos x} = \cos x - \sec x .
\end{aligned}
\]
\pause
Integrate %$u_1'$ and $u_2'$ to get $u_1$ and $u_2$.
\[
\begin{aligned}
& u_1 = \int u_1'\,dx 
= \int ( \cos x-\sec x ) \,dx
= \sin x -
\ln \left\lvert \sec x + \tan x \right\rvert
, \\
& u_2 = \int u_2'\,dx 
= \int \sin x\,dx = -\cos x .
\end{aligned}
\]
\pause
(Forget about constants of integration, we want a particular solution)

\end{frame}

\begin{frame}
So

\medskip

\quad
$
y_p = u_1 y_1 + u_2 y_2$

\pause
\medskip

\quad
$
\phantom{y_p}
=
\cos x \sin x
-
\cos x
\,
\ln \lvert
\sec x + \tan x
\rvert
-\cos x \sin x
$

\pause
\medskip

\quad
$
\phantom{y_p}
=
-
\cos x \, \ln \lvert
\sec x + \tan x
\rvert $.

\pause
\medskip

The general solution to $y'' + y = \tan x$ is
\[
y = C_1 \cos x + C_2 \sin x
-
\cos x \, \ln \lvert
\sec x + \tan x
\rvert .
\]

\medskip
\pause

In general, the procedure for finding $y_p$ using variation parameters (2nd
order) is

1) Solve the associated homogeneous equation.

\pause
2) Solve for $u_1'$ and $u_2'$ in
\[
%\boxed{~~
\begin{aligned}
& u_1' y_1 + u_2' y_2 = 0 ,\\
& u_1' y_1' + u_2' y_2' = f(x) .
\end{aligned}
%~~}
\]
\pause
3) Integrate to get $u_1$ and $u_2$ to find $y_p = u_1 y_1 + u_2 y_2$.

\medskip
\pause

\textbf{Note:} Undetermined coefficients are usually less tedious if
applicable.
\end{frame}

\end{document}
