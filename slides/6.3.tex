\documentclass[10pt,aspectratio=169]{beamer}

% All the boilerplate is in deslides.sty
\usepackage{deslides}

\author{Ji\v{r}\'i Lebl}

\institute[OSU]{%
Oklahoma State University%
%Departemento pri Matematiko de Oklahoma {\^S}tata Universitato%
}

\title{27. Convolution (Laplace transform)\\(Notes on Diffy Qs, 6.3)}

\date{}

\begin{document}

\begin{frame}
\titlepage

%\bigskip

\begin{center}
The textbook: \url{https://www.jirka.org/diffyqs/}
\end{center}
\end{frame}

\begin{frame}
Laplace transform of a product is \textbf{not} the product of the transforms.

\medskip
\pause

Suppose $f(t)$ and $g(t)$ are defined for $t \geq 0$.  Define
the \emph{convolution} as a new function $f * g$:
\[
(f * g)(t) \overset{\text{def}}{=}
\int_0^t f(\tau) g(t-\tau) \, d\tau .
\]

\pause

\emph{Remark:}  Same formula as
$\int_{-\infty}^\infty f(\tau) g(t-\tau) \, d\tau$
assuming $f(t)=g(t)=0$ for $t < 0$.

\medskip
\pause

\textbf{Example:}
Let $f(t) = e^t$ and $g(t) = t$ for $t \geq 0$.  Then 
\[
(f*g)(t)
=
\int_0^t e^\tau (t-\tau) \, d\tau
\pause
=
\text{(integration by parts)}
=
e^t - t - 1 .
\]

%\medskip
\pause

\textbf{Example:}
Take $f(t) = \sin (\omega t)$ and $g(t) = \cos (\omega t)$ for $t \geq 0$.
%Then (use the identity
%$\cos (\theta) \sin (\psi) =
%\frac{1}{2} \, \bigl( \sin (\theta + \psi) - \sin (\theta - \psi) \bigr)$)

\smallskip

$\displaystyle
\quad\qquad
(f*g)(t)
=
\int_0^t  \sin ( \omega \tau ) \,
\cos \bigl( \omega (t-\tau) \bigr) \, d\tau
\pause
=
%\underset{\mathclap{\substack{\\\ \\\uparrow\\\cos (\theta) \sin (\psi) =
%\frac{1}{2} \, \bigl( \sin (\theta + \psi) - \sin (\theta - \psi) \bigr)}}}{=}
\int_0^t
\underbrace{
\tfrac{1}{2} \, \bigl( \sin (\omega t) - \sin (\omega t - 2 \omega \tau
) \bigr)
}_{\mathclap{\text{used } \cos (\theta) \sin (\psi) =
\frac{1}{2} \, \left( \sin (\theta + \psi) - \sin (\theta - \psi) \right)}}
\, d\tau
$

\medskip
\pause

$\displaystyle
\quad\qquad
\phantom{(f*g)(t)}
%Apply the identity
%\begin{equation*}
%\cos (\theta) \sin (\psi) =
%\frac{1}{2} \, \bigl( \sin (\theta + \psi) - \sin (\theta - \psi) \bigr) ,
%\end{equation*}
%to get
{} =
\left[ \tfrac{1}{2} \, \tau  \sin (\omega t) + \tfrac{1}{4\omega} \, \cos (2 \omega \tau -
\omega t) \right]_{\tau=0}^t
\pause
= \tfrac{1}{2} \, t \sin (\omega t) .
$

\medskip
\pause

Only for $t \geq 0$. \quad  $f$, $g$, and $f*g$ are undefined for $t < 0$.

\end{frame}

\begin{frame}
Convolution is sort of like a product.  If $f,g,h$ are functions and $c$
is a constant, then
\begin{align*}
& f * g = g * f , \\
& (c f) * g = f * (c g) = c (f*g) , \\
& (f+g) * h = f * h + g * h , \\
& ( f * g ) * h = f * ( g * h ) .
\end{align*}


\medskip
\pause

For us, the most
interesting property is

\medskip

\textbf{Theorem:}
If $f(t)$ and $g(t)$ are of exponential order, then
so is $(f*g)(t)$ and
\begin{equation*}
\mathcal{L} \bigl\{ (f*g)(t) \bigr\}
=
\mathcal{L} \left\{ \int_0^t f(\tau) g(t-\tau) \, d\tau \right\}
=
\mathcal{L} \bigl\{ f(t) \bigr\} \mathcal{L} \bigl\{ g(t) \bigr\} .
\end{equation*}

\medskip
\pause

So the Laplace transform of a convolution is the product
of the Laplace transforms:

\medskip
\qquad
${\mathcal{L}} \bigl\{ (f*g)(t) \bigr\} = F(s)G(s)$,
\pause
\qquad
or in reverse,
\qquad
${\mathcal{L}}^{-1} \bigl\{ F(s) G(s) \bigr\} = (f*g)(t)$.



\end{frame}

\begin{frame}
\textbf{Example:}
Find the inverse Laplace transform of
\[
\frac{1}{s^2 (s+1)} \pause = 
\frac{1}{s^2}
\,
\frac{1}{s+1}
.
\]
\pause
We recognize
\[
\mathcal{L}^{-1} 
\left\{
\frac{1}{s^2} \right\} 
= t
\qquad \text{and} \qquad
\mathcal{L}^{-1} 
\left\{
\frac{1}{s+1} \right\}
= e^{-t} .
\]
\pause
Therefore,
\[
\mathcal{L}^{-1}
\left\{
\frac{1}{s^2(s+1)}
\right\}
\pause
=
\mathcal{L}^{-1}
\left\{
\frac{1}{s^2}
\,
\frac{1}{s+1}
\right\}
\pause
=
\underbrace{
\int_0^t
\tau e^{-(t-\tau)} \,d\tau
}_{\mathclap{\text{convolution of } t \text{ and } e^{-t}}}
\pause
=
\text{(integration by parts)}
=
\pause
e^{-t}+t-1 .
\]

\end{frame}

\begin{frame}
\textbf{Example:}
Solve \quad 
$x'' + \omega_0^2 x = f(t) , \quad x(0) = 0, \quad x'(0) = 0$.

\medskip
\pause

Apply Laplace: \quad
$s^2 X(s) + \omega_0^2 X(s) = F(s)$,
\pause
\quad
or
\quad
$X(s) = \dfrac{1}{s^2+ \omega_0^2} F(s)$.

\medskip
\pause

$H(s) = \frac{1}{s^2+ \omega_0^2}$ is the transfer function
and
$
{\mathcal{L}}^{-1} \left\{
\frac{1}{s^2+ \omega_0^2}
\right\} = 
\frac{\sin (\omega_0 t)}{\omega_0}$.

\medskip
\pause

To find $
x(t) = 
{\mathcal{L}}^{-1} \bigl\{
X(s)
\bigr\}
=
{\mathcal{L}}^{-1} \bigl\{
H(s) F(s)
\bigr\}$, we use convolution:
\[
x(t) = 
\int_0^t
\frac{\sin (\omega_0 \tau)}{\omega_0}
f(t-\tau) \, d\tau 
\pause
=
\int_0^t
f(\tau) 
\frac{\sin \bigl( \omega_0 (t-\tau) \bigr)}{\omega_0} \, d\tau .
\]
\pause
Let's see how Laplace handles resonance:
\quad
Suppose $f(t) =
\cos (\omega_0 t)$.
\pause\quad
Then
\[
x(t) = 
\int_0^t
\frac{\sin (\omega_0 \tau)}{\omega_0} \,
\cos \bigl( \omega_0 (t-\tau) \bigr) \, d\tau
=
\frac{1}{\omega_0}
\int_0^t
\sin ( \omega_0 \tau ) \,
\cos \bigl(\omega_0 (t-\tau) \bigr) \, d\tau .
\]
\pause
We just computed convolution of sine and cosine a moment ago!
\[
x(t) =
\left(
\frac{1}{\omega_0}
\right) \,
\left(
\frac{1}{2} \,
t
\sin ( \omega_0 t )
\right)
=
\frac{1}{2 \omega_0} \,
t
\sin ( \omega_0 t ).
\]
\pause
The $t$ in front of the sine causes oscillations growing without
bound, that is, resonance.

\end{frame}

\begin{frame}
The general idea for a constant coefficient
\quad $Lx = f(t)$, \quad $x(0)=0$, \quad $x'(0)=0$, \quad
is

\medskip
\pause

Find the transfer function $H(s)$.

\medskip
\pause

Then $X(s) = H(s)F(s)$.

\medskip
\pause

If
$h(t) = {\mathcal{L}}^{-1}\bigl\{ H(s) \bigr\}$, then
\[
x(t)
= {\mathcal{L}}^{-1}\bigl\{ X(s) \bigr\}
= {\mathcal{L}}^{-1}\bigl\{ F(s)H(s) \bigr\}
= (f * h)(t)
= \int_0^t f(\tau) h(t-\tau) \, d\tau .
\]
\pause
We solve for an arbitrary 
forcing function $f(t)$ as a definite integral using convolution.

\end{frame}

\begin{frame}
A integral equation could contain convolution, e.g., 
the \emph{Volterra integral equation}:

\medskip

\quad
$\displaystyle
x(t) = f(t) + \int_0^t g(t-\tau) x(\tau) \, d\tau$,
\quad 
where $f(t)$ and $g(t)$ are known,
and $x(t)$ is the unknown.

\medskip
\pause

To find $x(t)$ apply Laplace:
\quad
$X(s) = F(s) + G(s) X(s)$,
\pause
\quad
or
\quad
$X(s) = \dfrac{F(s)}{1-G(s)}$, \pause \quad then invert.

\medskip
\pause

\textbf{Example:}
Solve
\quad
$\displaystyle
x(t) =  e^{-t} + \int_0^t \sinh(t-\tau) x(\tau) \, d\tau$.

\medskip
\pause

Apply Laplace:
\[
X(s) = \frac{1}{s+1} + \frac{1}{s^2-1} X(s) .
\]
\pause
Solve for $X(s)$:
\[
X(s) = \frac{\frac{1}{s+1}}{1- \frac{1}{s^2-1}}
\pause
=
\frac{s-1}{s^2 - 2}
\pause
=
\frac{s}{s^2 - 2}
-
\frac{1}{s^2 - 2} .
\]
\pause
Using the table, invert:
\[
x(t) = \cosh \bigl( \sqrt{2} \, t \bigr) -
\frac{1}{\sqrt{2}} \sinh \bigl( \sqrt{2}\, t \bigr).
\]

\end{frame}


\end{document}
