\documentclass[10pt,aspectratio=169]{beamer}

% All the boilerplate is in deslides.sty
\usepackage{deslides}

\author{Ji\v{r}\'i Lebl}

\institute[OSU]{%
Oklahoma State University%
%Departemento pri Matematiko de Oklahoma {\^S}tata Universitato%
}

\title{17. Mechanical vibrations, part 2: free damped motion\\(Notes on Diffy Qs, 2.4)}

\date{}

\begin{document}

\begin{frame}
\titlepage

%\bigskip

\begin{center}
The textbook: \url{https://www.jirka.org/diffyqs/}
\end{center}
\end{frame}

\begin{frame}
\textbf{Review:}

\medskip

\textbf{Mass on a spring:}

\vspace*{-12pt}
\hspace*{3in}%
\scalebox{0.9}{\subimport*{../figures}{massfigforce.pdf_t}}

\vspace*{-0.55in}
$t = {}$ time in seconds.\\
$x(t) = {}$ displacement from rest, in meters.\\
$m = {}$ mass in kilograms.\\
$k = {}$ spring constant in newtons per meter.\\
$c = {}$ damping (friction) in newton-seconds per meter.\\
$F(t) = {}$ external force on the mass in newtons.

\medskip

\[
mx'' + cx' + kx = F(t) .
\]

\medskip
\pause

We are considering free motion, $F \equiv 0$.

\medskip
\pause

Last time we looked at undamped motion, $c=0$.
\end{frame}

\begin{frame}
%\subsection{Free damped motion}

%mbxINTROSUBSUBSECTION

Let us now focus on \myindex{damped} motion.  Let us rewrite the equation
\begin{equation*}
m x'' + c x' + kx = 0,
\end{equation*}
as
\begin{equation*}
x'' + 2p x' + \omega_0^2 x = 0,
\end{equation*}
where
\begin{equation*}
\omega_0 = \sqrt{\frac{k}{m}}, \qquad p = \frac{c}{2m} .
\end{equation*}
The characteristic equation is
\begin{equation*}
r^2 + 2 pr + \omega_0^2 = 0 .
\end{equation*}
Using the quadratic formula we get that the roots are
\begin{equation*}
r = -p \pm \sqrt{p^2 - \omega_0^2} .
\end{equation*}
The form of the solution depends on whether we get complex or real roots.
We get real roots if and only if the following number is nonnegative:
\begin{equation*}
p^2 - \omega_0^2 = {\left( \frac{c}{2m} \right)}^2 - \frac{k}{m}
= \frac{c^2 - 4km}{4m^2} .
\end{equation*}
The sign of $p^2-\omega_0^2$ is the same as the sign of
$c^2 - 4km$.  Thus we get real roots if and only if $c^2-4km$ is
nonnegative, or in other words if $c^2 \geq 4km$.

%\subsubsection{Overdamping}

%15 is the number of lines, must be adjusted
%mbxSTARTIGNORE
\begin{mywrapfig}[15]{3.25in}
\capstart
\diffyincludegraphics{width=3in}{width=4.5in}{mv-overdamped}
\caption{Overdamped motion for several different initial conditions.\label{mv:overdampedfig}}
\end{mywrapfig}
%mbxENDIGNORE
%
% make sure the MBX below is synced!
%

When
$c^2 - 4km > 0$, the system is \emph{\myindex{overdamped}}.  In this case,
there are two distinct real roots $r_1$ and $r_2$.  Both roots are
negative:  As $\sqrt{p^2 - \omega_0^2}$ is always less than $p$,
then
$-p \pm \sqrt{p^2 - \omega_0^2}$ is negative in either case.


The solution is
\begin{equation*}
x(t) = C_1 e^{r_1 t} + C_2 e^{r_2 t} .
\end{equation*}
Since $r_1, r_2$ are negative, $x(t) \to 0$ as $t \to \infty$.
Thus the mass will tend towards the rest position as
time goes to infinity.  For a few sample plots for different initial
conditions, see \figurevref{mv:overdampedfig}.

%mbxlatex \begin{myfig}
%mbxlatex \diffyincludegraphics{width=3in}{width=4.5in}{mv-overdamped}
%mbxlatex \caption{Overdamped motion for several different initial conditions.\label{mv:overdampedfig}}
%mbxlatex \end{myfig}

No oscillation happens.  In fact, the graph crosses the
$x$-axis at most once.  To see why, we try to solve
$0 = C_1 e^{r_1 t} + C_2 e^{r_2 t}$.
Therefore, $C_1 e^{r_1 t} = - C_2 e^{r_2 t}$ and using laws of exponents we
obtain
\begin{equation*}
\frac{-C_1}{C_2} = e^{(r_2-r_1) t} .
\end{equation*}
This equation has at most one solution $t \geq 0$.
For some initial conditions the graph never crosses the $x$-axis, as is
evident from the sample graphs.

\begin{example}
Suppose the mass is released from rest.  That is,
$x(0) = x_0$ and $x'(0) = 0$.
Then
\begin{equation*}
x(t) = \frac{x_0}{r_1-r_2} \left(r_1 e^{r_2 t} - r_2 e^{r_1 t} \right) .
\end{equation*}
It is not hard to see that this satisfies the initial conditions.
\end{example}

%\subsubsection{Critical damping}

When
$c^2 - 4km = 0$, the system is \emph{\myindex{critically damped}}.  In this case,
there is one root of multiplicity 2 and this root is $-p$.  Our solution is
\begin{equation*}
x(t) = C_1 e^{-pt} + C_2 t e^{-pt} .
\end{equation*}
The behavior of a critically damped system is very similar to an overdamped
system.  After all a critically damped system is in some sense a limit
of overdamped systems.  Since these equations are really only an
approximation to the real world, in reality we are never critically
damped, it is a place we can only reach in theory.  We are always
a little bit underdamped or a little bit overdamped.  It is better not to
dwell on critical damping.

%\subsubsection{Underdamping}

%13 is the number of lines, must be adjusted
%mbxSTARTIGNORE
\begin{mywrapfig}[13]{3.25in}
\capstart
\diffyincludegraphics{width=3in}{width=4.5in}{mv-underdamped}
\caption{Underdamped motion with the envelope curves shown.\label{mv:underdampedfig}}
\end{mywrapfig}
%mbxENDIGNORE
%
% make sure the MBX below is synced!
%
When
$c^2 - 4km < 0$, the system is \emph{\myindex{underdamped}}.  In this case,
the roots are complex.
\begin{equation*}
\begin{split}
r & =
-p \pm \sqrt{p^2 - \omega_0^2} \\
& = 
-p \pm \sqrt{-1}\sqrt{\omega_0^2 - p^2} \\
& = 
-p \pm i \omega_1 ,
\end{split}
\end{equation*}
where $\omega_1 =\sqrt{\omega_0^2 - p^2}$.  Our solution is
\begin{equation*}
x(t) = e^{-pt} \bigl( A \cos (\omega_1 t) + B \sin (\omega_1 t) \bigr) ,
\end{equation*}
or
\begin{equation*}
x(t) = C e^{-pt} \cos ( \omega_1 t - \gamma ) .
\end{equation*}
An example plot is given in \figurevref{mv:underdampedfig}.  Note that we
still have that $x(t) \to 0$ as $t \to \infty$.

%mbxlatex \begin{myfig}
%mbxlatex \diffyincludegraphics{width=3in}{width=4.5in}{mv-underdamped}
%mbxlatex \caption{Underdamped motion with the envelope curves shown.\label{mv:underdampedfig}}
%mbxlatex \end{myfig}

The figure also 
shows the \emph{\myindex{envelope curves}}
$C e^{-pt}$ and $-C e^{-pt}$.  The solution
is the oscillating line between the two envelope curves.
The envelope curves give
the maximum amplitude of the oscillation at any given point in time.  For
example, if you are bungee jumping, you are really interested in computing the
envelope curve as not to hit the concrete with your head.

The phase shift $\gamma$ shifts the oscillation left or right, but within the
envelope curves (the envelope curves do not change if $\gamma$
changes).


Notice that the angular
\emph{\myindex{pseudo-frequency}}\footnote{We do not call $\omega_1$ a frequency
since the solution is not really a periodic function.}  becomes
smaller when the damping $c$ (and hence $p$) becomes larger.  This makes sense.
When we change the damping just a little bit, we do not
expect the behavior of the solution to change dramatically.
If we keep making $c$ larger, then
at some point the solution should start looking 
like the solution for critical damping or overdamping, where no oscillation
happens.  So if $c^2$ approaches $4km$, we want $\omega_1$ to approach 0.

On the other hand, when $c$ gets smaller, $\omega_1$ approaches $\omega_0$
($\omega_1$ is always smaller than $\omega_0$), and the solution looks more and more like the steady
periodic motion of the undamped case.  The envelope curves become flatter and
flatter as $c$ (and hence $p$) goes to 0.

\end{frame}

\end{document}
