\chapter{Higher order linear ODEs} \label{ho:chapter}

%%%%%%%%%%%%%%%%%%%%%%%%%%%%%%%%%%%%%%%%%%%%%%%%%%%%%%%%%%%%%%%%%%%%%%%%%%%%%%

\section{Second order linear ODEs}
\label{solinear:section}

\sectionnotes{less than 1 lecture\EPref{,
first part of \S3.1 in \cite{EP}}\BDref{,
parts of \S3.1 and \S3.2 in \cite{BD}}}

Let us consider the general
\emph{\myindex{second order linear differential equation}}
\begin{equation*}
A(x) y'' + B(x)y' + C(x)y = F(x) .
\end{equation*}
We usually divide through by $A(x)$ to get
\begin{equation} \label{sol:eqlin}
y'' + p(x)y' + q(x)y = f(x) ,
\end{equation}
where $p(x) = \nicefrac{B(x)}{A(x)}$, $q(x) = \nicefrac{C(x)}{A(x)}$, and
$f(x) = \nicefrac{F(x)}{A(x)}$.
The word \emph{linear\index{linear equation}} means that the equation contains no powers nor
functions of $y$, $y'$, and $y''$.

In the special case when $f(x) = 0$, we have a so-called
\emph{homogeneous\index{homogeneous linear equation}}
equation
\begin{equation} \label{sol:eqlinhom}
y'' + p(x)y' + q(x)y = 0 .
\end{equation}
We have already seen some second order linear homogeneous equations.
\begin{align*}
\qquad y'' + k^2 y & = 0 &
& \text{Two solutions are:} \quad y_1 = \cos (kx), \quad y_2 = \sin(kx) . \qquad \\
\qquad y'' - k^2 y & = 0 &
& \text{Two solutions are:} \quad y_1 = e^{kx}, \quad y_2 = e^{-kx} . \qquad
\end{align*}

If we know two solutions of a linear homogeneous equation, we know a lot
more of them.

\begin{theorem}[Superposition]\index{superposition}
Suppose $y_1$ and $y_2$ are two solutions of the
homogeneous equation \eqref{sol:eqlinhom}.  Then 
\begin{equation*}
y(x) = C_1 y_1(x) + C_2 y_2(x) ,
\end{equation*}
also solves \eqref{sol:eqlinhom} for arbitrary constants $C_1$ and $C_2$.
\end{theorem}

That is, we can add solutions together and multiply them by constants to
obtain new and different solutions.  We call
the expression $C_1 y_1 + C_2 y_2$ a
\emph{\myindex{linear combination}} of $y_1$ and $y_2$.
Let us
prove this theorem; the
proof is very enlightening and illustrates how linear equations work.

\medskip

Proof:
Let 
$y = C_1 y_1 + C_2 y_2$.  Then
\begin{equation*}
\begin{split}
y'' + py' + qy & =
(C_1 y_1 + C_2 y_2)'' + p(C_1 y_1 + C_2 y_2)' + q(C_1 y_1 + C_2 y_2) \\
& = C_1 y_1'' + C_2 y_2'' + C_1 p y_1' + C_2 p y_2' + C_1 q y_1 + C_2 q y_2 \\
& = C_1 ( y_1'' + p y_1' + q y_1 ) + C_2 ( y_2'' + p y_2' + q y_2 ) \\
& = C_1 \cdot 0 + C_2 \cdot 0 = 0 . \qed
\end{split}
\end{equation*}

\medskip

The proof becomes even simpler to state if we use the
operator notation.
An \emph{\myindex{operator}} is an object that eats functions and spits out functions (kind of
like what a function is, but a function eats numbers and spits out numbers).
Define the operator $L$ by
\begin{equation*}
Ly = y'' + py' + qy .
\end{equation*}
The differential equation now becomes $Ly=0$.
The operator (and the equation)
$L$ being \emph{linear}\index{linear operator} means that $L(C_1y_1 + C_2y_2) = 
C_1 Ly_1 + C_2 Ly_2$.  The proof above becomes
\begin{equation*}
Ly = L(C_1y_1 + C_2y_2) = 
C_1 Ly_1 + C_2 Ly_2 = C_1 \cdot 0 + C_2 \cdot 0 = 0 .
\end{equation*}

\medskip

Two different solutions to the second equation $y'' - k^2y = 0$ are
$y_1 = \cosh (kx)$ and $y_2 = \sinh (kx)$.  Let us remind ourselves of the
definition, $\cosh x = \frac{e^x  + e^{-x}}{2}$ and
$\sinh x = \frac{e^x - e^{-x}}{2}$.  Therefore, these are solutions by
superposition as they
are linear combinations of the two
exponential solutions.

The functions $\sinh$ and $\cosh$ are sometimes more convenient to use than the
exponential.  Let us review some of their properties.
\begin{align*}
& \cosh 0  = 1 &   & \sinh 0 = 0 \\
& \frac{d}{dx} \cosh x = \sinh x &  & \frac{d}{dx} \sinh x = \cosh x \\
& \cosh^2 x - \sinh^2 x = 1
\end{align*}


\begin{exercise}
Derive these properties using the definitions of $\sinh$
and $\cosh$ in terms of exponentials.
\end{exercise}


Linear equations have nice and simple
answers to the existence and uniqueness question.

\begin{theorem}[Existence and uniqueness]\index{existence and uniqueness}
Suppose $p, q, f$ are continuous functions on some interval
$I$, $a$ is a number in $I$,
and $a, b_0, b_1$ are constants.
The equation
\begin{equation*}
y'' + p(x) y' + q(x) y = f(x) ,
\end{equation*}
has exactly one solution $y(x)$ defined on the same interval $I$ satisfying the initial conditions
\begin{equation*}
y(a) = b_0 , \qquad y'(a) = b_1 .
\end{equation*}
\end{theorem}

For example, the equation $y'' + k^2 y = 0$ with $y(0) = b_0$ and $y'(0) = b_1$
has the solution
\begin{equation*}
y(x) = b_0 \cos (kx) + \frac{b_1}{k} \sin (kx) .
\end{equation*}

The equation $y'' - k^2 y = 0$ with $y(0) = b_0$ and $y'(0) = b_1$
has the solution
\begin{equation*}
y(x) = b_0 \cosh (kx) + \frac{b_1}{k} \sinh (kx) .
\end{equation*}
Using $\cosh$ and $\sinh$ in this solution allows us to solve for
the initial conditions
in a cleaner way
than if we have used the exponentials.

\medskip

The initial conditions for a second order ODE consist of two
equations.  Common sense tells us that
if we have two arbitrary constants and two equations, then we should 
be able to solve
for the constants and find a solution to the differential equation
satisfying the initial conditions.

Question: Suppose we find two different solutions $y_1$ and $y_2$ to the
homogeneous equation \eqref{sol:eqlinhom}.  Can every solution
be written (using superposition) in the form
$y = C_1 y_1 + C_2 y_2$?

Answer is affirmative!  Provided that $y_1$ and $y_2$ are different enough in
the following sense.  We say $y_1$ and $y_2$ are \emph{\myindex{linearly
independent}} if one is not a constant multiple of the other.

\begin{theorem}
Let $p, q$ be continuous functions.
Let $y_1$ and $y_2$ be two linearly independent
solutions to the homogeneous equation \eqref{sol:eqlinhom}. 
Then every other solution is 
of the form
\begin{equation*}
y = C_1 y_1 + C_2 y_2 .
\end{equation*}
That is, $y = C_1 y_1 + C_2 y_2$ is the general solution.
\end{theorem}

For example, we found the solutions
$y_1 = \sin x$ and $y_2 = \cos x$ for the
equation $y'' + y = 0$.  It is not hard to see that sine and cosine are not
constant
multiples of each other.  If $\sin x = A \cos x$ for some constant $A$,
we let $x=0$ and this would imply $A = 0$.  But then $\sin x = 0$ for all
$x$, which is preposterous.
So $y_1$ and $y_2$ are linearly independent.  Hence
\begin{equation*}
y = C_1 \cos x + C_2 \sin x 
\end{equation*}
is the general solution to $y'' + y = 0$.

\medskip

We will study the solution of nonhomogeneous equations in
\sectionref{sec:nonhom}.  We will first focus on finding general solutions to
homogeneous equations.

\subsection{Exercises}

\begin{exercise}
Show that $y=e^x$ and $y=e^{2x}$ are linearly independent.
\end{exercise}

\begin{exercise}
Take $y'' + 5 y = 10 x + 5$.  Find (guess!) a solution.
\end{exercise}

\begin{exercise}
Prove the superposition principle for nonhomogeneous equations.  Suppose that
$y_1$ is a solution to $L y_1 = f(x)$ and $y_2$ is a solution to
$L y_2 = g(x)$ (same linear operator $L$).  Show that $y = y_1+y_2$ solves
$Ly = f(x) + g(x)$.
\end{exercise}

\begin{exercise}
For the equation $x^2 y'' - x y' = 0$, find two solutions, show that they
are linearly independent and find the general solution.
Hint: Try $y = x^r$.
\end{exercise}

Equations of the form $a x^2 y'' + b x y' + c y = 0$ are called
\emph{Euler's equations\index{Euler's equation}} or
\emph{Cauchy--Euler equations\index{Cauchy--Euler equation}}.
They are solved by trying
$y=x^r$ and solving for $r$ (assume that $x \geq 0$ for simplicity).

\begin{exercise} \label{sol:eulerex}
Suppose that ${(b-a)}^2-4ac > 0$.
\begin{tasks}
\task Find a formula for the general solution
of $a x^2 y'' + b x y' + c y = 0$.  Hint: Try $y=x^r$ and find a formula for
$r$.
\task What happens when ${(b-a)}^2-4ac = 0$ or ${(b-a)}^2-4ac < 0$?
\end{tasks}
\end{exercise}

We will revisit the case when ${(b-a)}^2-4ac < 0$ later.

\begin{exercise} \label{sol:eulerexln}
Same equation as in \exerciseref{sol:eulerex}.
Suppose ${(b-a)}^2-4ac = 0$.  Find a formula for the general solution
of $a x^2 y'' + b x y' + c y = 0$.  Hint: Try $y=x^r \ln x$ for the second
solution.
\end{exercise}

If you have one solution to a second order linear homogeneous
equation you can find another one.  This is the \emph{\myindex{reduction of
order method}}.

\begin{exercise}[reduction of order] \label{exercise:reductionoforder}
Suppose $y_1$ is a solution to $y'' + p(x) y' + q(x) y = 0$.
Show that
\begin{equation*}
y_2(x) = y_1(x) \int \frac{e^{-\int p(x)\,dx}}{{\bigl(y_1(x)\bigr)}^2} ~dx
\end{equation*}
is also a solution.
\end{exercise}

Note: If you wish to come up with the formula for reduction of order
yourself,
start by trying $y_2(x) = y_1(x) v(x)$.  Then plug $y_2$ into the
equation, use the fact that $y_1$ is a solution, substitute
$w = v'$, and you have a first order linear equation in $w$.  Solve for
$w$ and then for $v$.  When solving for $w$, make sure to include a constant
of integration.
Let us solve some famous equations using the method.

\begin{exercise}[\myindex{Chebyshev's equation of order 1}]
Take 
$(1-x^2)y''-xy' + y = 0$.
\begin{tasks}
\task Show that $y=x$ is a solution.
\task Use reduction of order to find a second linearly independent solution.
\task Write down the general solution.
\end{tasks}
\end{exercise}

\begin{exercise}[\myindex{Hermite's equation of order 2}]
Take 
$y''-2xy' + 4y = 0$.
\begin{tasks}
\task Show that $y=1-2x^2$ is a solution.  
\task Use reduction of order to find a second linearly independent solution.
\task Write down the general solution.
\end{tasks}
\end{exercise}

\setcounter{exercise}{100}

\begin{exercise}
Are $\sin(x)$ and $e^x$ linearly independent?  Justify.
\end{exercise}
\exsol{%
Yes.  To justify try to find a constant $A$ such that $\sin(x) = A e^x$
for all $x$.
}

\begin{exercise}
Are $e^x$ and $e^{x+2}$ linearly independent?  Justify.
\end{exercise}
\exsol{%
No.  $e^{x+2} = e^2 e^x$.
}

\begin{exercise}
Guess a solution to $y'' + y' + y= 5$.
\end{exercise}
\exsol{%
$y=5$
}

\begin{exercise}
Find the general solution to
$x y'' + y' = 0$.  Hint: Notice that it is a first order ODE in $y'$.
\end{exercise}
\exsol{%
$y=C_1 \ln(x) + C_2$
}

\begin{exercise}
Write down an equation (guess) for which we have the solutions
$e^x$ and $e^{2x}$.  Hint: Try an equation of the form
$y''+Ay'+By = 0$ for constants $A$ and $B$,
plug in both $e^x$ and $e^{2x}$ and solve for $A$ and $B$.
\end{exercise}
\exsol{%
$y''-3y'+2y = 0$
}


%%%%%%%%%%%%%%%%%%%%%%%%%%%%%%%%%%%%%%%%%%%%%%%%%%%%%%%%%%%%%%%%%%%%%%%%%%%%%%

\sectionnewpage
\section{Constant coefficient second order linear ODEs}
\label{sec:ccsol}

\sectionnotes{more than 1 lecture\EPref{,
second part of \S3.1 in \cite{EP}}\BDref{,
\S3.1 in \cite{BD}}}

Suppose we have the problem
\begin{equation*}
y''-6y'+8y = 0, \qquad y(0) = - 2, \qquad y'(0) = 6 .
\end{equation*}
This is a second order linear homogeneous equation with constant
coefficients.  \emph{Constant coefficients\index{constant coefficient}}
means that the functions 
in front of $y''$, $y'$, and $y$ are constants, not depending on $x$.

To guess a solution, think of a function that you know stays essentially the
same when we differentiate it, so that we can take the function and its
derivatives, add some multiples of these together, and end up with zero.

Let us try\footnote{%
Making an educated guess with some parameters to solve for 
is such a central technique in differential equations, that people sometimes use
a fancy name for such a guess: \emph{\myindex{ansatz}}, German for \myquote{initial
placement of a tool at a work piece.}  Yes, Germans have a word for that.}
a solution of the form $y = e^{rx}$.  Then $y' = r e^{rx}$ and
$y'' = r^2 e^{rx}$.  Plug in to get
\begin{align*}
y''-6y'+8y & = 0 , \\
r^2 e^{rx} -6 r e^{rx}+8 e^{rx} & = 0 , \\
r^2 -6 r +8 & = 0 \qquad \text{(divide through by } e^{rx} \text{)},\\
(r-2)(r-4) & = 0 .
\end{align*}
Hence, if $r=2$ or $r=4$, then $e^{rx}$ is a solution.  So let $y_1 = e^{2x}$
and $y_2 = e^{4x}$.

\begin{exercise}
Check that $y_1$ and $y_2$ are solutions.
\end{exercise}

The functions $e^{2x}$ and $e^{4x}$ are linearly independent.  If they
were not linearly independent we could write $e^{4x} = C e^{2x}$ for
some constant $C$,
implying that $e^{2x} = C$ for all $x$, which is clearly not possible. 
Hence, we can write the general solution as
\begin{equation*}
y = C_1 e^{2x} + C_2 e^{4x} .
\end{equation*}
We need to solve for $C_1$ and $C_2$.  To apply the initial conditions
we first find $y' = 2 C_1 e^{2x} + 4 C_2 e^{4x}$.  We plug in $x=0$ and
solve.
\begin{align*}
-2 & = y(0) = C_1 + C_2 , \\
6 & = y'(0) = 2 C_1 + 4 C_2 .
\end{align*}
Either apply some matrix algebra, or just solve these by high school
math.  For example, divide the second equation by 2
to obtain $3 = C_1 + 2 C_2$, and subtract the two equations to
get $5 = C_2$.  Then $C_1 = -7$ as $-2 = C_1 + 5$.  Hence, the solution we
are
looking for is
\begin{equation*}
y = -7 e^{2x} + 5 e^{4x} .
\end{equation*}

\medskip

Let us generalize this example into a method.
Suppose that we have an equation
\begin{equation} \label{ccsol:eq}
a y'' + b y' + c y = 0 ,
\end{equation}
where $a, b, c$ are constants.  Try the solution $y = e^{rx}$ to obtain
\begin{align*}
a r^2 e^{rx} + 
b r e^{rx} + 
c e^{rx} & = 0 , \\
a r^2 + 
b r + 
c & = 0 .
\end{align*}
The equation $a r^2 + b r + c = 0$ is called the
\emph{\myindex{characteristic equation}} of the ODE\@.
Solve for the $r$ by using the \myindex{quadratic formula}.
\begin{equation*}
r_1, r_2 = \frac{-b \pm \sqrt{b^2 - 4ac}}{2a} .
\end{equation*}
So $e^{r_1 x}$ and $e^{r_2 x}$ are solutions.  There is
still a difficulty if $r_1 = r_2$, but it is not hard to overcome.

\begin{theorem}
Suppose that $r_1$ and $r_2$ are the roots of the characteristic equation.
\begin{enumerate}[(i)]
\item If $r_1$ and $r_2$ are distinct and real (when $b^2 - 4ac > 0$),
then \eqref{ccsol:eq} has the general solution
\begin{equation*}
y = C_1 e^{r_1 x} + C_2 e^{r_2 x} .
\end{equation*}
\item If $r_1 = r_2$ (happens when $b^2 - 4ac = 0$), 
then \eqref{ccsol:eq} has the general solution
\begin{equation*}
y = (C_1 + C_2 x)\, e^{r_1 x} .
\end{equation*}
\end{enumerate}
\end{theorem}

\begin{example} \label{example:expsecondorder}
Solve
\begin{equation*}
y'' - k^2 y = 0 .
\end{equation*}
The characteristic equation is $r^2 - k^2 = 0$ or 
$(r-k)(r+k) = 0$.  Consequently, $e^{-k x}$ and $e^{kx}$ are the two
linearly independent solutions, and the general solution is
\begin{equation*}
y = C_1 e^{kx} + C_2e^{-kx} .
\end{equation*}
Since
$\cosh s = \frac{e^s+e^{-s}}{2}$
and
$\sinh s = \frac{e^s-e^{-s}}{2}$,
we can also write the general solution
as
\begin{equation*}
y = D_1 \cosh(kx) + D_2 \sinh(kx) .
\end{equation*}
\end{example}

\begin{example}
Find the general solution of
\begin{equation*}
y'' -8 y' + 16 y = 0 .
\end{equation*}

The characteristic equation is $r^2 - 8 r + 16 = {(r-4)}^2 = 0$.
The equation has a 
double root $r_1 = r_2 = 4$.  The general solution is, therefore,
\begin{equation*}
y = (C_1 + C_2 x)\, e^{4 x} = C_1 e^{4x} + C_2 x e^{4x} .
\end{equation*}

\begin{exercise}
Check that $e^{4x}$ and $x e^{4x}$ are linearly independent.
\end{exercise}

That $e^{4x}$ solves the equation is clear.  If $x e^{4x}$ solves the
equation, then we know we are done.  Let us compute
$y' = e^{4x} + 4xe^{4x}$ and
$y'' = 8 e^{4x} + 16xe^{4x}$.  Plug in
\begin{equation*}
y'' - 8 y' + 16 y = 
8 e^{4x} + 16xe^{4x} - 8(e^{4x} + 4xe^{4x}) + 16 xe^{4x} = 
0 .
\end{equation*}
\end{example}

In some sense, a doubled root rarely happens.  If coefficients are 
picked truly randomly we are unlikely to get a doubled root.
There are, however, some natural phenomena (such as resonance as we will see)
where a doubled root does happen, so we cannot just dismiss this case.

Let us give a short argument for why the solution $x e^{r x}$ works when the
root is doubled.  This case is really a limiting case of when
the two roots are distinct and very close.  Note that 
$\frac{e^{r_2 x} - e^{r_1 x}}{r_2 - r_1}$ is a solution when the roots are
distinct.  When we take the limit as $r_1$ goes to $r_2$, we are really
taking the
derivative of $e^{rx}$ using $r$ as the variable.  Therefore, the limit is 
$x e^{rx}$, and hence this is a solution in the doubled root case.

\subsection{Complex numbers and Euler's formula}

It may happen that a polynomial has some complex roots.  The
equation $r^2 + 1 = 0$ has no real roots, but it does have two complex roots.
Here we review some properties of complex numbers\index{complex number}.

\medskip

Complex numbers may seem a strange concept, especially because of the
terminology.  There is nothing imaginary or really complicated about complex
numbers.
A complex number is simply a pair of real numbers, $(a,b)$.  
Think of a complex number as a point in the plane.  We add complex numbers
in the straightforward way: $(a,b)+(c,d)=(a+c,b+d)$.  We define
multiplication\index{multiplication of complex numbers} by
\begin{equation*}
(a,b) \times (c,d) \overset{\text{def}}{=} (ac-bd,ad+bc) .
\end{equation*}
It turns out that with this multiplication rule, all the standard properties
of arithmetic hold.  Further, and most importantly $(0,1) \times (0,1) =
(-1,0)$.

Generally we just write $(a,b)$ as $a+ib$, and we treat $i$ as if it were an
unknown.  When $b$ is zero, then $(a,0)$ is just the number $a$.
We can do arithmetic with complex numbers just as we would
with polynomials.
The property we just mentioned becomes $i^2 = -1$.
So whenever we see $i^2$, we replace it by $-1$.
For example,
\begin{equation*}
(2+3i)(4i) - 5i = 
(2\times 4)i + (3 \times 4) i^2 - 5i
=
8i + 12 (-1) - 5i
=
-12 + 3i .
\end{equation*}

The numbers
$i$ and $-i$ are the two roots of $r^2 + 1 = 0$.
Note that engineers often use the letter $j$ instead of $i$ for the square
root of $-1$.  We will use the mathematicians' convention and use $i$.

\begin{exercise}
Make sure you understand (that you can justify)
the following identities:
\begin{tasks}(2)
\task $i^2 = -1$, $i^3 = -i$, $i^4 = 1$,
\task $\dfrac{1}{i} = -i$,
\task $(3-7i)(-2-9i) = \cdots = -69-13i$,
\task $(3-2i)(3+2i) = 3^2 - {(2i)}^2 = 3^2 + 2^2 = 13$,
\task $\frac{1}{3-2i} = \frac{1}{3-2i} \frac{3+2i}{3+2i} = \frac{3+2i}{13}
= \frac{3}{13}+\frac{2}{13}i$.
\end{tasks}
\end{exercise}

We also define the exponential $e^{a+ib}$ of a complex number.  We do
this by writing down the Taylor series and plugging in the complex
number.  Because most properties of the exponential can be proved by looking
at the Taylor series, these
properties still hold for the complex
exponential.  For example the very important property: $e^{x+y} = e^x e^y$.  This means that
$e^{a+ib} = e^a e^{ib}$.  Hence if we can compute $e^{ib}$, we can
compute $e^{a+ib}$.  For $e^{ib}$ we use the so-called
\emph{\myindex{Euler's formula}}.

\begin{theorem}[Euler's formula] \label{eulersformula}
\begin{equation*}
\boxed{~~
e^{i \theta} = \cos \theta + i \sin \theta
\qquad \text{ and } \qquad
e^{- i \theta} = \cos \theta - i \sin \theta .
~~}
\end{equation*}
\end{theorem}

In other words, $e^{a+ib} = e^a \bigl( \cos(b) + i \sin(b) \bigr) = e^a \cos(b) + i e^a \sin(b)$.

\begin{exercise}
Using Euler's formula, check the identities:
\begin{equation*}
\cos \theta = \frac{e^{i \theta} + e^{-i \theta}}{2}
\qquad \text{and} \qquad
\sin \theta = \frac{e^{i \theta} - e^{-i \theta}}{2i}.
\end{equation*}
\end{exercise}

\begin{exercise}
Double angle identities:
Start with $e^{i(2\theta)} = {\bigl(e^{i \theta} \bigr)}^2$.  Use Euler on
each side and deduce:
\begin{equation*}
\cos (2\theta) = \cos^2 \theta - \sin^2 \theta
\qquad \text{and} \qquad
\sin (2\theta) = 2 \sin \theta \cos \theta .
\end{equation*}
\end{exercise}

For a complex number $a+ib$ we call
$a$ the \emph{\myindex{real part}} and $b$ the \emph{\myindex{imaginary part}} of the number.
Often the following notation is used,
\begin{equation*}
\operatorname{Re}(a+ib) = a
\qquad \text{and} \qquad
\operatorname{Im}(a+ib) = b.
\end{equation*}

\subsection{Complex roots}

Suppose the equation $ay'' + by' + cy = 0$ has the 
characteristic equation
$a r^2 + b r + c = 0$ that has \myindex{complex roots}.
By the quadratic
formula, the roots are
$\frac{-b \pm \sqrt{b^2 - 4ac}}{2a}$.
These roots are complex if $b^2 - 4ac < 0$.  In this case the
roots are
\begin{equation*}
r_1, r_2 = \frac{-b}{2a} \pm i\frac{\sqrt{4ac - b^2}}{2a} .
\end{equation*}
As you can see, we always get a pair of roots of the form $\alpha \pm i
\beta$.  In this case we can still write the solution as
\begin{equation*}
y = C_1 e^{(\alpha+i\beta)x} + C_2 e^{(\alpha-i\beta)x} .
\end{equation*}
However, the exponential is now complex-valued.  We need to allow
$C_1$ and $C_2$ to be complex numbers to obtain a real-valued solution (which
is what we are after).  While there is nothing particularly wrong with this
approach,
it can make calculations harder and it is generally preferred
to find two real-valued
solutions.

Here we can use \hyperref[eulersformula]{Euler's formula}.  Let
\begin{equation*}
y_1 = e^{(\alpha+i\beta)x} \qquad \text{and} \qquad y_2 = e^{(\alpha-i\beta)x} .
\end{equation*}
Then 
\begin{align*}
y_1 & = e^{\alpha x} \cos (\beta x) + i e^{\alpha x} \sin (\beta x) , \\
y_2 & = e^{\alpha x} \cos (\beta x) - i e^{\alpha x} \sin (\beta x) .
\end{align*}

Linear combinations of solutions are also solutions.  Hence,
\begin{align*}
y_3 & = \frac{y_1 + y_2}{2} = e^{\alpha x} \cos (\beta x) , \\ 
y_4 & = \frac{y_1 - y_2}{2i} = e^{\alpha x} \sin (\beta x) ,
\end{align*}
are also solutions.  Furthermore, they are real-valued.  It is not hard to
see that they are linearly independent (not multiples of each other).
Therefore, we have the following theorem.

\begin{theorem}
Take the equation
\begin{equation*}
ay'' + by' + cy = 0 .
\end{equation*}
If the characteristic equation has the roots $\alpha \pm i \beta$
(when $b^2 - 4ac < 0$),
then the general solution is
\begin{equation*}
y = C_1 e^{\alpha x} \cos (\beta x) + C_2 e^{\alpha x} \sin (\beta x) .
\end{equation*}
\end{theorem}

\begin{example} \label{example:sincossecondorder}
Find the general solution of $y'' + k^2 y = 0$, for a constant
$k > 0$.

The characteristic equation is $r^2 + k^2 = 0$.  Therefore,
the roots are $r = \pm ik$, and by the theorem, we have the general solution
\begin{equation*}
y = C_1 \cos (kx) + C_2 \sin (kx) .
\end{equation*}
\end{example}

\begin{example}
Find the solution of $y'' - 6 y' + 13 y = 0$, $y(0) = 0$, $y'(0) =
10$.

The characteristic equation is $r^2 - 6 r + 13 = 0$.  By completing the
square we get ${(r-3)}^2 + 2^2 = 0$ and hence the roots are
$r = 3 \pm 2i$.
By the theorem we have the general solution
\begin{equation*}
y = C_1 e^{3x} \cos (2x) + C_2 e^{3x} \sin (2x) .
\end{equation*}
To find the solution satisfying the initial conditions, we first plug in zero
to get
\begin{equation*}
0 = y(0) = C_1 e^{0} \cos 0 + C_2 e^{0} \sin 0  = C_1 .
\end{equation*}
Hence $C_1 = 0$ and $y = C_2 e^{3x} \sin (2x)$.  We differentiate
\begin{equation*}
y' = 3C_2 e^{3x} \sin (2x) + 2C_2 e^{3x} \cos (2x) .
\end{equation*}
We again plug in the initial condition and obtain $10 = y'(0) = 2C_2$, or
$C_2 = 5$.  Hence the solution we are seeking is
\begin{equation*}
y = 5 e^{3x} \sin (2x) .
\end{equation*}
\end{example}

\subsection{Exercises}

\begin{exercise}
Find the general solution of $2y'' + 2y' -4 y = 0$.
\end{exercise}

\begin{exercise}
Find the general solution of $y'' + 9y' - 10 y = 0$.
\end{exercise}

\begin{exercise}
Solve $y'' - 8y' + 16 y = 0$ for $y(0) = 2$, $y'(0) = 0$.
\end{exercise}

\begin{exercise}
Solve $y'' + 9y' = 0$ for $y(0) = 1$, $y'(0) = 1$.
\end{exercise}

\begin{exercise}
Find the general solution of $2y'' + 50y = 0$.
\end{exercise}

\begin{exercise}
Find the general solution of $y'' + 6 y' + 13 y = 0$.
\end{exercise}

\begin{exercise}
Find the general solution of $y'' = 0$ using the methods of this section.
\end{exercise}

\begin{exercise}
The method of this section applies to equations of other orders than two.
We will see
higher orders later.  Try to solve the first order equation
$2y' + 3y = 0$ using the methods of this section.
\end{exercise}

\begin{exercise}
Let us revisit the Cauchy--Euler equations\index{Cauchy--Euler equation} of
\exercisevref{sol:eulerex}.  Suppose now
that ${(b-a)}^2-4ac < 0$.  Find a formula for the general solution
of $a x^2 y'' + b x y' + c y = 0$.  Hint: Note that $x^r = e^{r \ln x}$.
\end{exercise}

\begin{exercise}
Find the solution to
$y''-(2\alpha) y' + \alpha^2 y=0$, $y(0) = a$, $y'(0)=b$,
where $\alpha$, $a$, and $b$ are real numbers.
\end{exercise}

\begin{exercise}
Construct an equation such that $y = C_1 e^{-2x} \cos(3x) + C_2 e^{-2x}
\sin(3x)$ is the general
solution.
\end{exercise}

\setcounter{exercise}{100}

\begin{exercise}
Find the general solution to
$y''+4y'+2y=0$.
\end{exercise}
\exsol{%
$y =
C_1 e^{(-2+\sqrt{2}) x}
+
C_2 e^{(-2-\sqrt{2}) x}$
}

\begin{exercise}
Find the general solution to
$y''-6y'+9y=0$.
\end{exercise}
\exsol{%
$y =
C_1 e^{3x}
+
C_2 x e^{3x}$
}

\begin{exercise}
Find the solution to
$2y''+y'+y=0$, $y(0) = 1$, $y'(0)=-2$.
\end{exercise}
\exsol{%
$y =
e^{-x/4} \cos\bigl((\nicefrac{\sqrt{7}}{4})x\bigr)
-
\sqrt{7}
e^{-x/4} \sin\bigl((\nicefrac{\sqrt{7}}{4})x\bigr)$
}

\begin{exercise}
Find the solution to
$2y''+y'-3y=0$, $y(0) = a$, $y'(0)=b$.
\end{exercise}
\exsol{%
$y = \frac{2(a-b)}{5} \, e^{-3x/2}+\frac{3 a+2 b}{5} \, e^x$
}

\begin{exercise}
Find the solution to
$z''(t) = -2z'(t)-2z(t)$, $z(0) = 2$, $z'(0)= -2$.
\end{exercise}
\exsol{%
$z(t) =
2e^{-t} \cos(t)$
}

\begin{exercise}
Find the solution to
$y''-(\alpha+\beta) y' + \alpha \beta y=0$, $y(0) = a$, $y'(0)=b$,
where $\alpha$, $\beta$, $a$, and $b$ are real numbers, and $\alpha \not=
\beta$.
\end{exercise}
\exsol{%
$y =
\frac{a \beta-b}{\beta-\alpha} e^{\alpha x} + 
\frac{b-a \alpha}{\beta-\alpha} e^{\beta x}$
}

\begin{exercise}
Construct an equation such that $y = C_1 e^{3x} + C_2 e^{-2x}$ is the general
solution.
\end{exercise}
\exsol{%
$y'' -y'-6y=0$
}

%%%%%%%%%%%%%%%%%%%%%%%%%%%%%%%%%%%%%%%%%%%%%%%%%%%%%%%%%%%%%%%%%%%%%%%%%%%%%%

\sectionnewpage
\section{Higher order linear ODEs} \label{sec:hol}

\sectionnotes{somewhat more than 1 lecture\EPref{, \S3.2 and \S3.3 in
\cite{EP}}\BDref{,
\S4.1 and \S4.2 in \cite{BD}}}

%After reading this lecture, it may be good to try
%Project III\index{IODE software!Project III} from the
%IODE website: \url{http://www.math.uiuc.edu/iode/}.
%
%\medskip

We briefly study higher order equations.
Equations appearing in applications tend to be second
order.  Higher order equations do appear from time to time, but
generally the world around us is \myquote{second order.}  

The basic results about linear ODEs of higher order are essentially 
the same as for second order equations, with 2 replaced by $n$.
The important concept
of linear independence is somewhat more complicated when more than two
functions are involved.
For higher order constant coefficient ODEs, the methods developed are also
somewhat harder to apply,
but we will not dwell on these complications.
It is also possible to use the methods for systems
of linear equations from \chapterref{sys:chapter} to solve higher order
constant coefficient equations.

Let us start with a general homogeneous linear equation
\begin{equation} \label{hol:eqlinhom}
y^{(n)} + p_{n-1}(x)y^{(n-1)} + \cdots + p_1(x) y' + p_0(x) y = 0 .
\end{equation}

\begin{theorem}[Superposition]\index{superposition}
Suppose $y_1$, $y_2$, \ldots, $y_n$ are solutions of the
homogeneous equation \eqref{hol:eqlinhom}.  Then 
\begin{equation*}
y(x) = C_1 y_1(x) + C_2 y_2(x) + \cdots + C_n y_n(x) 
\end{equation*}
also solves \eqref{hol:eqlinhom}
for arbitrary constants $C_1, C_2, \ldots, C_n$.
\end{theorem}

In other words, a \emph{\myindex{linear combination}} of solutions
to \eqref{hol:eqlinhom}
is also a solution to \eqref{hol:eqlinhom}.
We also have the existence and uniqueness theorem for nonhomogeneous linear
equations.

\begin{theorem}[Existence and uniqueness]\index{existence and uniqueness}
Suppose $p_0$ through $p_{n-1}$, and $f$ are continuous functions
on some interval $I$,
$a$ is a number in $I$,
and $b_0, b_1, \ldots, b_{n-1}$ are constants.
The equation
\begin{equation*} %\label{hol:eqlin}
y^{(n)} + p_{n-1}(x)y^{(n-1)} + \cdots + p_1(x) y' + p_0(x) y = f(x) 
\end{equation*}
has exactly one solution $y(x)$ defined on the same interval $I$
satisfying the initial conditions
\begin{equation*}
y(a) = b_0, \quad y'(a) = b_1, \quad \ldots, \quad y^{(n-1)}(a) = b_{n-1} .
\end{equation*}
\end{theorem}

\subsection{Linear independence}

When we had two functions $y_1$ and $y_2$ we said they were linearly
independent if one was not the multiple of the other.  Same idea holds for
$n$ functions.  In this case it is easier to state as follows. The functions
$y_1$, $y_2$, \ldots, $y_n$ are \emph{\myindex{linearly independent}} if
the equation
\begin{equation*}
c_1 y_1 + c_2 y_2 + \cdots + c_n y_n = 0 
\end{equation*}
has only the trivial solution $c_1 = c_2 = \cdots = c_n = 0$, where the
equation must hold for all $x$.  If we can
solve equation with some constants where for example $c_1 \not= 0$, then we
can solve for $y_1$ as a linear combination of the others.  If the functions
are not
linearly independent, they are \emph{\myindex{linearly dependent}}.

\begin{example}
Show that $e^x, e^{2x}, e^{3x}$ are linearly independent.

Let us give several ways to show this fact.
Many textbooks (including \cite{EP} and
\cite{F}) introduce Wronskians, but it is difficult to see why they work and
they are not really necessary here.

Let us write down
\begin{equation*}
c_1 e^x + c_2 e^{2x} + c_3 e^{3x} = 0.
\end{equation*}
We use rules of exponentials and write $z = e^x$.  Hence $z^2 = e^{2x}$
and $z^3 = e^{3x}$.  Then we have
\begin{equation*}
c_1 z + c_2 z^2 + c_3 z^3 = 0.
\end{equation*}
The left hand side is a third degree polynomial in $z$.
It is either identically zero,
or it has at most 3 zeros.  Therefore, it is identically zero,
$c_1 = c_2 = c_3 = 0$, and the functions are linearly independent.

Let us try another way.  As before we write
\begin{equation*}
c_1 e^x + c_2 e^{2x} + c_3 e^{3x} = 0.
\end{equation*}
This equation has to hold for all $x$.  We divide through
by $e^{3x}$ to get
\begin{equation*}
c_1 e^{-2x} + c_2 e^{-x} + c_3 = 0.
\end{equation*}
As the equation is true for all $x$, let $x \to \infty$.  After taking the
limit we see that $c_3 = 0$.  Hence our equation becomes
\begin{equation*}
c_1 e^x + c_2 e^{2x} = 0.
\end{equation*}
Rinse, repeat!

How about yet another way.  We again write
\begin{equation*}
c_1 e^x + c_2 e^{2x} + c_3 e^{3x} = 0.
\end{equation*}
We can evaluate the equation and its derivatives at different
values of $x$ to obtain equations for
$c_1$, $c_2$, and $c_3$.
Let us first
divide by $e^{x}$ for simplicity.
\begin{equation*}
c_1 + c_2 e^{x} + c_3 e^{2x} = 0.
\end{equation*}
We set $x=0$ to get the equation $c_1 + c_2 + c_3 = 0$.  Now differentiate
both sides
\begin{equation*}
c_2 e^{x} + 2 c_3 e^{2x} = 0 .
\end{equation*}
We set $x=0$ to get $c_2 + 2c_3 = 0$.  We divide by $e^x$ again and
differentiate to get
$2 c_3 e^{x} = 0$.  It is clear that $c_3$ is zero.  Then $c_2$ must be
zero as $c_2 = -2c_3$, and $c_1$ must be zero because $c_1 + c_2 + c_3 = 0$.

There is no one best way to do it.  All of these methods are perfectly valid.
The important thing is to understand why the functions are linearly
independent.
\end{example}

\begin{example}
On the other hand, the functions $e^x$, $e^{-x}$, and $\cosh x$ are linearly
dependent.  Simply apply definition of the hyperbolic cosine:
\begin{equation*}
\cosh x = \frac{e^x + e^{-x}}{2} 
\qquad
\text{or}
\qquad
2 \cosh x - e^x - e^{-x} = 0.
\end{equation*}
\end{example}

\subsection{Constant coefficient higher order ODEs}

When we have a higher order constant coefficient homogeneous linear
equation, the song and dance is exactly the same as it was for second order.
We just need to find more solutions.  If the equation is
$n^{\text{th}}$ order we need to find $n$ linearly independent solutions.
It is best seen by example.

\begin{example}
Find the general solution to
\begin{equation} \label{hol:cceq1}
y''' - 3 y'' - y' + 3y = 0 .
\end{equation}

Try: $y = e^{rx}$.  We plug in and get
\begin{equation*}
\underbrace{r^3 e^{rx}}_{y'''} - 3 \underbrace{r^2 e^{rx}}_{y''} -
\underbrace{r e^{rx}}_{y'} + 3 \underbrace{e^{rx}}_{y} = 0 .
\end{equation*}
We divide through by $e^{rx}$.  Then 
\begin{equation*}
r^3 - 3 r^2 - r + 3 = 0 .
\end{equation*}
The trick now is to find the roots.  There is a formula for the roots of
degree 3 and 4 polynomials but it is very complicated.  There is no formula
for higher degree polynomials.  That does not mean that the roots do not
exist.  There are always
$n$ roots for an $n^{\text{th}}$ degree polynomial.  They may be
repeated\index{repeated roots}
and they may be complex.  Computers are pretty good at finding roots
approximately for reasonable size polynomials.

A good place to start is to plot the polynomial and check where it is zero.
We can also simply try plugging in.  We
just start plugging
in numbers $r=-2,-1,0,1,2,\ldots$ and see if we get a hit (we can also
try complex numbers).  Even
if we do not get a hit, we may get an indication
of where the root is.  For example, we plug
$r=-2$ into our polynomial and get $-15$; we plug in $r=0$ and get 3.
That means there is a root between $r=-2$ and $r=0$,
because the sign changed.
If we find one root, say $r_1$, then we know $(r-r_1)$ is a factor
of our polynomial.  Polynomial long division can then be used.

A good strategy is to begin with $r=-1$, 1, or 0.  These are
easy to compute.  Our polynomial happens to have
two such roots, $r_1 = -1$
and $r_2 = 1$.  There should be 3 roots and the last root is reasonably
easy to find.  The constant
term in a monic polynomial such as this is the multiple of the negations of all the roots
because $r^3 - 3 r^2 - r + 3 = (r-r_1)(r-r_2)(r-r_3)$.
So
\begin{equation*}
3 = (-r_1)(-r_2)(-r_3) = (1)(-1)(-r_3) = r_3 .
\end{equation*}
You should check that $r_3 = 3$ really
is a root.  Hence we know that $e^{-x}$, $e^{x}$
and $e^{3x}$ are solutions to \eqref{hol:cceq1}.  They are linearly independent
as can easily be checked, and there are 3 of them, which happens to be exactly
the number we need.  Hence the general solution is
\begin{equation*}
y = C_1 e^{-x} + C_2 e^{x} + C_3 e^{3x} .
\end{equation*}

Suppose we were given some initial conditions $y(0) = 1$, $y'(0) = 2$,
and $y''(0) = 3$.  Then
\begin{align*}
1 = y(0) & = C_1 + C_2 + C_3 , \\
2 = y'(0) & = -C_1 + C_2 + 3C_3 , \\
3 = y''(0) & = C_1 + C_2 + 9C_3 .
\end{align*}
It is possible to find the solution by high school algebra, but it would be a
pain.
The sensible way to solve a system of equations such as this is to use
matrix algebra, see
\sectionref{sec:matrix}.  For now we note that the solution is $C_1 =
-\nicefrac{1}{4}$,
$C_2 = 1$, and $C_3 = \nicefrac{1}{4}$.  The specific solution
to the ODE is
\begin{equation*}
y = \frac{-1}{4}\, e^{-x} + e^x + \frac{1}{4}\, e^{3x} .
\end{equation*}
\end{example}

Next, suppose that we have real roots, but they are repeated.  Let us say
we have
a root $r$ repeated $k$ times.  In the spirit of the second
order solution, and for the same reasons, we have the solutions
\begin{equation*}
e^{rx}, \quad xe^{rx}, \quad x^2 e^{rx}, \quad \ldots, \quad x^{k-1} e^{rx} .
\end{equation*}
We take a linear combination of these solutions to find the general
solution.

\begin{example}
Solve
\begin{equation*}
y^{(4)} - 3 y''' + 3 y'' - y' =  0 .
\end{equation*}

We note that the characteristic equation is
\begin{equation*}
r^4 - 3r^3 + 3r^2 -r = 0 .
\end{equation*}
By inspection we note that $r^4 - 3r^3 + 3r^2 -r = r{(r-1)}^3$.  Hence
the roots given with \myindex{multiplicity} are $r = 0, 1, 1, 1$.  Thus the general
solution is
\begin{equation*}
y = \underbrace{(C_1 + C_2 x + C_3 x^2)\, e^x}_{\text{terms coming from }
r=1} + \underbrace{C_4}_{\text{from } r=0} .
\end{equation*}
\end{example}

The case of complex roots is similar
to second order equations.
Complex roots
always come in pairs $r = \alpha \pm i \beta$.  Suppose we have
two such complex roots, each repeated $k$ times.
The corresponding solution is
\begin{equation*}
( C_0 + C_1 x + \cdots + C_{k-1} x^{k-1} ) \, e^{\alpha x} \cos (\beta x)
+
( D_0 + D_1 x + \cdots + D_{k-1} x^{k-1} ) \, e^{\alpha x} \sin (\beta x) .
\end{equation*}
where $C_0$, \ldots, $C_{k-1}$, $D_0$, \ldots, $D_{k-1}$ are arbitrary
constants.

\begin{example}
Solve
\begin{equation*}
y^{(4)} - 4 y''' + 8 y'' - 8 y' + 4y = 0 .
\end{equation*}

The characteristic equation is
\begin{align*}
r^4 - 4 r^3 + 8 r^2 - 8 r + 4 & = 0 , \\
{(r^2-2r+2)}^2 & = 0 , \\
{\bigl({(r-1)}^2+1\bigr)}^2 & = 0 .
\end{align*}
Hence the roots are $1 \pm i$, both with multiplicity 2.  Hence the general
solution to the ODE is
\begin{equation*}
y = 
( C_1 + C_2 x ) \, e^{x} \cos x
+
( C_3 + C_4 x ) \, e^{x} \sin x .
\end{equation*}
The way we solved the characteristic equation above is really by guessing or
by inspection.  It is not so easy in general.  We could also have asked
a computer or an advanced calculator for the roots.
\end{example}

%FIXME: the operator stuff?

\subsection{Exercises}

\begin{exercise}
Find the general solution for $y''' - y'' + y' - y = 0$.
\end{exercise}

\begin{exercise}
Find the general solution for $y^{(4)} - 5 y''' + 6 y'' = 0$.
\end{exercise}

\begin{exercise}
Find the general solution for $y''' + 2 y'' + 2 y' = 0$.
\end{exercise}

\begin{exercise}
Suppose the characteristic equation for an ODE is
${(r-1)}^2{(r-2)}^2 = 0$.
\begin{tasks}
\task
Find such a differential equation.
\task
Find its general solution.
\end{tasks}
\end{exercise}

\begin{exercise} \label{hol:eqfromsolex}
Suppose that a fourth order equation has a solution
$y = 2 e^{4x} x \cos x$.  
\begin{tasks}
\task
Find such an equation.
\task
Find the initial conditions that the given
solution satisfies.
\end{tasks}
\end{exercise}

\begin{exercise}
Find the general solution for the equation of \exerciseref{hol:eqfromsolex}.
\end{exercise}

\begin{exercise}
Let
$f(x) = e^x - \cos x$, $g(x) = e^x + \cos x$, and $h(x) = \cos x$.
Are $f(x)$, $g(x)$, and $h(x)$
linearly independent?  If so, show
it, if not, find a linear combination that works.
\end{exercise}

\begin{exercise}
Let
$f(x) = 0$, $g(x) = \cos x$, and $h(x) = \sin x$.
Are $f(x)$, $g(x)$, and $h(x)$
linearly independent?  If so, show
it, if not, find a linear combination that works.
\end{exercise}

\begin{exercise}
Are $x$, $x^2$, and $x^4$
linearly independent?  If so, show
it, if not, find a linear combination that works.
\end{exercise}

\begin{exercise}
Are $e^x$, $xe^x$, and $x^2e^x$
linearly independent?  If so, show
it, if not, find a linear combination that works.
\end{exercise}

\begin{exercise}
Find an equation such that $y=xe^{-2x}\sin(3x)$ is a solution.
\end{exercise}

\setcounter{exercise}{100}

\begin{exercise}
Find the general solution of $y^{(5)}-y^{(4)}=0$
\end{exercise}
\exsol{%
$y=C_1 e^x +C_2 x^3 + C_3 x^2 +C_4 x + C_5$
}

\begin{exercise}
Suppose that the characteristic equation of a third order differential
equation has
roots $\pm 2i$ and 3.
\begin{tasks}
\task
What is the characteristic equation?
\task
Find the
corresponding differential equation.
\task
Find the general solution.
\end{tasks}
\end{exercise}
\exsol{%
a) $r^3-3r^2+4r-12 = 0$,
b) $y'''-3y''+4y'-12y = 0$,
c) $y = C_1 e^{3x} + C_2 \sin(2x) + C_3 \cos(2x)$
}

\begin{exercise}
Solve $1001y'''+3.2y''+\pi y'-\sqrt{4} y = 0$, $y(0)=0$, $y'(0) = 0$,
$y''(0) = 0$.
\end{exercise}
\exsol{%
$y=0$
}

\begin{exercise}
Are $e^{x}$, $e^{x+1}$, $e^{2x}$, $\sin(x)$ linearly independent?
If so, show it, if not find a linear combination that works.
\end{exercise}
\exsol{%
No.  $e^1 e^x -  e^{x+1} = 0$.
}

\begin{exercise}
Are $\sin(x)$, $x$, $x\sin(x)$ linearly independent?
If so, show it, if not find a linear combination that works.
\end{exercise}
\exsol{%
Yes.  (Hint: First note that $\sin(x)$ is bounded.  Then note that
$x$ and $x\sin(x)$ cannot be multiples of each other.)
}

\begin{exercise}
Find an equation such that $y=\cos(x)$, $y=\sin(x)$, $y=e^x$ are solutions.
\end{exercise}
\exsol{%
$y'''-y''+y'-y=0$
}

%%%%%%%%%%%%%%%%%%%%%%%%%%%%%%%%%%%%%%%%%%%%%%%%%%%%%%%%%%%%%%%%%%%%%%%%%%%%%%

\sectionnewpage
\section{Mechanical vibrations} \label{sec:mv}

\sectionnotes{2 lectures\EPref{, \S3.4 in \cite{EP}}\BDref{,
\S3.7 in \cite{BD}}}

Let us look at some applications of linear second order constant
coefficient equations.

\subsection{Some examples}

\begin{mywrapfigsimp}{2.0in}{2.3in}
\noindent
\inputpdft{massfigforce}
\end{mywrapfigsimp}
Our first example is a mass on a spring.  Suppose we have a mass $m > 0$
(in kilograms) connected
by a spring with spring constant $k > 0$ (in newtons per meter)
to a fixed wall.  There may be some external
force $F(t)$ (in newtons) acting on the mass.  Finally, there is some
friction measured by $c \geq 0$ (in newton-seconds per meter) as the mass
slides along the floor (or perhaps a damper is connected).

Let $x$ be the displacement of the mass ($x=0$ is the rest position), with
$x$ growing to the right (away from the wall).
The force exerted by the spring is proportional to the
compression of the spring by \myindex{Hooke's law}.
Therefore, it is $kx$ in the negative direction.
Similarly the amount of force exerted by friction is proportional
to the velocity of the mass.
By \myindex{Newton's second law} we know that force equals mass times acceleration
and hence $mx'' = F(t)-cx'-kx$ or
\begin{equation*}
mx'' + cx' + kx = F(t) .
\end{equation*}
This is a linear second order constant coefficient ODE\@.
We say the motion is
\begin{enumerate}[(i)]
\item \emph{forced\index{forced motion}}, if $F \not\equiv 0$ (if $F$ is not identically zero),
\item \emph{unforced\index{unforced motion}} or \emph{free\index{free
motion}}, if $F \equiv 0$ (if $F$ is identically zero),
\item \emph{damped\index{damped motion}}, if $c > 0$, and
\item \emph{undamped\index{undamped motion}}, if $c = 0$.
\end{enumerate}

This system appears in lots of applications even if it does not at first
seem like it.  Many real-world scenarios can be simplified to
a mass on a spring.  For example, a bungee jump setup is essentially a mass
and spring system (you are the mass).  It would be good if someone did the math
before you jump off the bridge, right?  Let us give two other examples.

\medskip

%5 is the number of lines, must be adjusted
\begin{mywrapfigsimp}[5]{1.35in}{1.65in}
\noindent
\inputpdft{mv-rlc}
\end{mywrapfigsimp}
Here is an example for electrical engineers.  Consider the
pictured \myindex{RLC circuit}.
There is a resistor with a resistance of $R$ ohms, an
inductor with an inductance of $L$ henries,
and a capacitor with a capacitance of $C$ farads.  There is also
an electric source (such as a battery) giving a voltage of $E(t)$ volts
at time $t$ (measured in seconds).
Let $Q(t)$ be the charge in coulombs on the capacitor
and $I(t)$ be the current in the circuit.  The relation between the two is
$Q' = I$.  By elementary principles we find 
$L I' + RI + \nicefrac{Q}{C} = E$.   We differentiate to get
\begin{equation*}
L I''(t) + R I'(t) + \frac{1}{C} I(t) = E'(t) .
\end{equation*}
This is a nonhomogeneous second order constant coefficient linear equation.
As $L, R$, and $C$ are all positive, this system behaves just like the
mass and spring system.  The position of the mass is replaced by the current.
Mass is replaced by the inductance, damping is replaced by resistance, and
the spring constant is replaced by one over the capacitance.  The change in
voltage becomes the forcing function---for constant voltage this is an
unforced motion.

\medskip

%10 is the number of lines, must be adjusted
\begin{mywrapfigsimp}[10]{1.8in}{2.16in}
\noindent
\inputpdft{mv-pend-deriv}
\end{mywrapfigsimp}
Our next example behaves like a mass and spring system only
approximately. Suppose a
mass $m$ hangs on a pendulum of length $L$.  We seek an equation for
the angle $\theta(t)$ (in radians).  Let $g$ be the force of gravity.
Elementary physics mandates that the equation is
\begin{equation*}
\theta'' + \frac{g}{L} \sin \theta = 0 .
\end{equation*}

Let us derive this equation using \myindex{Newton's second law}:
force equals mass times acceleration.  The acceleration is
$L \theta''$ and mass is $m$.  So $mL\theta''$ has to be equal
to the tangential component of the force given by the gravity, which is
$m g \sin \theta$ in the opposite direction.
So $mL\theta'' = -mg \sin \theta$.
The $m$ curiously cancels from the equation.

Now we make our approximation.  For small $\theta$ we have that approximately
$\sin \theta \approx \theta$.  This can be seen by looking at the graph.
In \figurevref{mv:sinthetafig} we can see that for approximately
$-0.5 < \theta < 0.5$ (in radians) the graphs of $\sin \theta$ and $\theta$ are almost the
same.

\begin{mywrapfig}{3.25in}
\capstart
\diffyincludegraphics{width=3in}{width=4.5in}{mv-sintheta}
\caption{The graphs of $\sin \theta$ and $\theta$ (in radians).\label{mv:sinthetafig}}
\end{mywrapfig}

Therefore, when the swings are small, $\theta$ is small and we can
model the behavior by the simpler linear equation
\begin{equation*}
\theta'' + \frac{g}{L} \theta = 0 .
\end{equation*}
The errors from this approximation build up.
So after a
long time, the state of the real-world system might be substantially
different from our solution.  Also we will
see that in a mass-spring system, the amplitude is independent of the
period.
This is not true for a pendulum.  Nevertheless, for reasonably short periods of time
and small swings (that is, only small angles $\theta$),
the approximation is reasonably good.

In real-world problems it is often necessary to make these types of
simplifications.  We must understand both the mathematics and
the physics of the situation to see if the simplification is valid in the
context of the questions we are trying to answer.

\subsection{Free undamped motion}

In this section we only consider free or unforced motion,
as we do not know yet how to solve nonhomogeneous equations.  Let us start with
\myindex{undamped} motion where $c=0$.  The equation is
\begin{equation*}
mx'' + kx = 0 .
\end{equation*}
We divide by $m$ and let $\omega_0 = \sqrt{\nicefrac{k}{m}}$ to rewrite the equation as
\begin{equation*}
x'' + \omega_0^2 x = 0 .
\end{equation*}
The general solution to this equation is
\begin{equation*}
x(t) = A \cos (\omega_0 t) + B \sin (\omega_0 t) .
\end{equation*}
By a trigonometric identity
\begin{equation*}
A \cos (\omega_0 t) + B \sin (\omega_0 t) =
C \cos ( \omega_0 t - \gamma ) ,
\end{equation*}
for two different constants $C$ and $\gamma$.
It is not hard to compute that $C= \sqrt{A^2 + B^2}$ and $\tan \gamma =
\nicefrac{B}{A}$.  Therefore, we let
$C$ and $\gamma$ be our arbitrary constants and write
$x(t) = C \cos ( \omega_0 t - \gamma )$.

\begin{exercise}
Justify the above identity and verify the equations for $C$
and $\gamma$.  Hint: Start with
$\cos (\alpha-\beta) = \cos (\alpha) \cos
(\beta) + \sin (\alpha)\sin (\beta)$ and multiply by $C$.  Then think what should
$\alpha$ and $\beta$ be.
\end{exercise}

While it is generally easier to use the first form with $A$ and $B$
to solve for the initial conditions, the second form is much
more natural.  The constants $C$ and $\gamma$ have nice physical interpretation.
Write the solution as
\begin{equation*}
x(t) = C \cos ( \omega_0 t - \gamma ) .
\end{equation*}
This is a pure-frequency oscillation (a sine wave).
The \emph{\myindex{amplitude}} is $C$, $\omega_0$ is the (angular)
\emph{\myindex{frequency}}\index{angular frequency},
and $\gamma$ is the so-called \emph{\myindex{phase shift}}.
The phase shift just shifts the
graph left or right.
We call $\omega_0$ the \emph{\myindex{natural (angular) frequency}}.
This entire setup is usually  
called \emph{\myindex{simple harmonic motion}}.

Let us pause to explain the word \emph{angular}
before the word \emph{frequency}.
The units of
$\omega_0$ are radians per unit time, not cycles per unit time 
as is the usual measure of frequency.  Because one cycle is $2
\pi$ radians, the usual frequency is given by $\frac{\omega_0}{2\pi}$.
It is simply a matter of where we put the constant $2\pi$, and that is a
matter of taste.

The \emph{\myindex{period}} of the motion is one over the frequency (in cycles per unit
time) and hence $\frac{2\pi}{\omega_0}$.  That is the amount of time it takes
to complete one full cycle.


\begin{example}
Suppose that $m=\unit[2]{kg}$ and $k=\unitfrac[8]{N}{m}$.
The whole mass and spring setup is sitting on
a truck that was traveling at \unitfrac[1]{m}{s}.
The truck crashes and hence stops.
The mass was held in place 0.5 meters forward from the rest position.  During
the crash the mass gets loose.  That is, the mass is now 
moving forward at \unitfrac[1]{m}{s}, while the other end of the
spring is held
in place.  The mass therefore starts oscillating.
What is the frequency of the resulting oscillation?  What is the amplitude?
The units are the mks units\index{mks units} (meters-kilograms-seconds).

The setup means that the mass was at half a meter in the positive
direction during the crash and
relative to the wall the spring is mounted to, the mass was moving forward
(in the positive direction) at \unitfrac[1]{m}{s}.  This gives us the initial
conditions.

So the equation with initial conditions is
\begin{equation*}
2 x'' + 8 x = 0 , \qquad x(0) = 0.5, \qquad x'(0) = 1.
\end{equation*}
We directly compute $\omega_0 = \sqrt{\nicefrac{k}{m}} = \sqrt{4} = 2$.
Hence the angular frequency is 2.  The usual frequency in Hertz (cycles per
second) is $\nicefrac{2}{2\pi} = \nicefrac{1}{\pi} \approx 0.318$.

The general solution is
\begin{equation*}
x(t) = A \cos (2t) + B \sin (2t) .
\end{equation*}
Letting $x(0) = 0.5$ means $A = 0.5$.  Then $x'(t) = - 2(0.5) \sin (2t)
+ 2B \cos (2t)$.
Letting $x'(0) = 1$ we get $B = 0.5$.  Therefore, the amplitude is
$C = \sqrt{A^2+B^2} = \sqrt{0.25+0.25} = \sqrt{0.5} \approx 0.707$.  The solution is
\begin{equation*}
x(t) = 0.5 \cos (2t) + 0.5 \sin (2t) .
\end{equation*}
A plot of $x(t)$ is shown in \figurevref{mv:undampedfig}.
\end{example}

\begin{mywrapfig}{3.25in}
\capstart
\diffyincludegraphics{width=3in}{width=4.5in}{mv-undamped}
\caption{Simple undamped oscillation.\label{mv:undampedfig}}
\end{mywrapfig}

In general, for free undamped motion, a solution of the
form
\begin{equation*}
x(t) = A \cos (\omega_0 t) + B \sin (\omega_0 t) ,
\end{equation*}
corresponds to the initial conditions $x(0) = A$ and $x'(0) = \omega_0 B$.
Therefore, it is easy to figure out $A$ and $B$ from the initial
conditions. 
The amplitude and the phase shift can then be computed from $A$ and $B$.
In the example, we have already found the amplitude $C$.  Let us
compute the phase shift.  We know that $\tan \gamma = \nicefrac{B}{A} = 1$.  We take the
arctangent of 1 and get $\nicefrac{\pi}{4}$ or approximately 0.785.
We still need to check if this $\gamma$ is in the correct quadrant
(and add $\pi$ to $\gamma$ if it is not).
Since both $A$ and $B$ are positive, then $\gamma$ should be in the first
quadrant, $\nicefrac{\pi}{4}$ radians is in the first quadrant, so $\gamma =
\nicefrac{\pi}{4}$.

Note: Many
calculators and computer software have not only the
\texttt{atan}\index{atan} function
for arctangent, but also what is sometimes called \texttt{atan2}\index{atan2}.
This function
takes two arguments, $B$ and $A$, and returns a $\gamma$ in the
correct quadrant for you.

\subsection{Free damped motion}

%mbxINTROSUBSUBSECTION

Let us now focus on \myindex{damped} motion.  Let us rewrite the equation
\begin{equation*}
m x'' + c x' + kx = 0,
\end{equation*}
as
\begin{equation*}
x'' + 2p x' + \omega_0^2 x = 0,
\end{equation*}
where
\begin{equation*}
\omega_0 = \sqrt{\frac{k}{m}}, \qquad p = \frac{c}{2m} .
\end{equation*}
The characteristic equation is
\begin{equation*}
r^2 + 2 pr + \omega_0^2 = 0 .
\end{equation*}
Using the quadratic formula we get that the roots are
\begin{equation*}
r = -p \pm \sqrt{p^2 - \omega_0^2} .
\end{equation*}
The form of the solution depends on whether we get complex or real roots.
We get real roots if and only if the following number is nonnegative:
\begin{equation*}
p^2 - \omega_0^2 = {\left( \frac{c}{2m} \right)}^2 - \frac{k}{m}
= \frac{c^2 - 4km}{4m^2} .
\end{equation*}
The sign of $p^2-\omega_0^2$ is the same as the sign of
$c^2 - 4km$.  Thus we get real roots if and only if $c^2-4km$ is
nonnegative, or in other words if $c^2 \geq 4km$.

\subsubsection{Overdamping}

%15 is the number of lines, must be adjusted
%mbxSTARTIGNORE
\begin{mywrapfig}[15]{3.25in}
\capstart
\diffyincludegraphics{width=3in}{width=4.5in}{mv-overdamped}
\caption{Overdamped motion for several different initial conditions.\label{mv:overdampedfig}}
\end{mywrapfig}
%mbxENDIGNORE
%
% make sure the MBX below is synced!
%

When
$c^2 - 4km > 0$, we say the system is \emph{\myindex{overdamped}}.  In this case,
there are two distinct real roots $r_1$ and $r_2$.  Both roots are
negative:  As $\sqrt{p^2 - \omega_0^2}$ is always less than $p$,
then
$-p \pm \sqrt{p^2 - \omega_0^2}$ is negative in either case.


The solution is
\begin{equation*}
x(t) = C_1 e^{r_1 t} + C_2 e^{r_2 t} .
\end{equation*}
Since $r_1, r_2$ are negative, $x(t) \to 0$ as $t \to \infty$.
Thus the mass will tend towards the rest position as
time goes to infinity.  For a few sample plots for different initial
conditions, see \figurevref{mv:overdampedfig}.

%mbxlatex \begin{myfig}
%mbxlatex \diffyincludegraphics{width=3in}{width=4.5in}{mv-overdamped}
%mbxlatex \caption{Overdamped motion for several different initial conditions.\label{mv:overdampedfig}}
%mbxlatex \end{myfig}

Do note that no oscillation happens.  In fact, the graph crosses the
$x$-axis at most once.  To see why, we try to solve
$0 = C_1 e^{r_1 t} + C_2 e^{r_2 t}$.
Therefore, $C_1 e^{r_1 t} = - C_2 e^{r_2 t}$ and using laws of exponents we
obtain
\begin{equation*}
\frac{-C_1}{C_2} = e^{(r_2-r_1) t} .
\end{equation*}
This equation has at most one solution $t \geq 0$.
For some initial conditions the graph never crosses the $x$-axis, as is
evident from the sample graphs.

\begin{example}
Suppose the mass is released from rest.  That is
$x(0) = x_0$ and $x'(0) = 0$.
Then
\begin{equation*}
x(t) = \frac{x_0}{r_1-r_2} \left(r_1 e^{r_2 t} - r_2 e^{r_1 t} \right) .
\end{equation*}
It is not hard to see that this satisfies the initial conditions.
\end{example}

\subsubsection{Critical damping}

When
$c^2 - 4km = 0$, we say the system is \emph{\myindex{critically damped}}.  In this case,
there is one root of multiplicity 2 and this root is $-p$.  Our solution is
\begin{equation*}
x(t) = C_1 e^{-pt} + C_2 t e^{-pt} .
\end{equation*}
The behavior of a critically damped system is very similar to an overdamped
system.  After all a critically damped system is in some sense a limit
of overdamped systems.  Since these equations are really only an
approximation to the real world, in reality we are never critically
damped, it is a place we can only reach in theory.  We are always
a little bit underdamped or a little bit overdamped.  It is better not to
dwell on critical damping.

\subsubsection{Underdamping}

When
$c^2 - 4km < 0$, the system is \emph{\myindex{underdamped}}.  In this case,
the roots are complex.
\begin{equation*}
\begin{split}
r & =
-p \pm \sqrt{p^2 - \omega_0^2} \\
& = 
-p \pm \sqrt{-1}\sqrt{\omega_0^2 - p^2} \\
& = 
-p \pm i \omega_1 ,
\end{split}
\end{equation*}
where $\omega_1 =\sqrt{\omega_0^2 - p^2}$.  Our solution is
\begin{equation*}
x(t) = e^{-pt} \bigl( A \cos (\omega_1 t) + B \sin (\omega_1 t) \bigr) ,
\end{equation*}
or
\begin{equation*}
x(t) = C e^{-pt} \cos ( \omega_1 t - \gamma ) .
\end{equation*}
An example plot is given in \figurevref{mv:underdampedfig}.  Note that we
still have that $x(t) \to 0$ as $t \to \infty$.

\begin{mywrapfig}{3.25in}
\capstart
\diffyincludegraphics{width=3in}{width=4.5in}{mv-underdamped}
\caption{Underdamped motion with the envelope curves shown.\label{mv:underdampedfig}}
\end{mywrapfig}
The figure also 
shows the \emph{\myindex{envelope curves}}
$C e^{-pt}$ and $-C e^{-pt}$.  The solution
is the oscillating line between the two envelope curves.
The envelope curves give
the maximum amplitude of the oscillation at any given point in time.  For
example, if you are bungee jumping, you are really interested in computing the
envelope curve as not to hit the concrete with your head.

The phase shift $\gamma$ shifts the oscillation left or right, but within the
envelope curves (the envelope curves do not change if $\gamma$
changes).


Finally notice that the angular \emph{\myindex{pseudo-frequency}}
\footnote{We do not call $\omega_1$ a frequency
since the solution is not really a periodic function.}  becomes
smaller when the damping $c$ (and hence $p$) becomes larger.  This makes sense.
When we change the damping just a little bit, we do not
expect the behavior of the solution to change dramatically.
If we keep making $c$ larger, then
at some point the solution should start looking 
like the solution for critical damping or overdamping, where no oscillation
happens.  So if $c^2$ approaches $4km$, we want $\omega_1$ to approach 0.

On the other hand when $c$ becomes smaller, $\omega_1$ approaches $\omega_0$
($\omega_1$ is always smaller than $\omega_0$), and the solution looks more and more like the steady
periodic motion of the undamped case.  The envelope curves become flatter and
flatter as $c$ (and hence $p$) goes to 0.

\subsection{Exercises}

\begin{samepage}
\begin{exercise} \label{mv:ex1}
Consider a mass and spring system with a mass $m=2$, spring constant $k=3$, and
damping constant $c=1$.
\begin{tasks}
\task Set up and find the general solution of the system.
\task Is the system underdamped, overdamped or critically damped?
\task If the system is not critically damped, find a $c$ that makes the system
critically damped.
\end{tasks}
\end{exercise}
\end{samepage}

\begin{exercise}
Do \exerciseref{mv:ex1} for
$m=3$, $k=12$, and $c=12$.
\end{exercise}

\begin{exercise} \label{mv:exwt1}
Using the mks units (meters-kilograms-seconds)\index{mks units},
suppose you have a spring with spring constant \unitfrac[4]{N}{m}.
You want to use
it to weigh items.  Assume no friction.  You place the mass on
the spring and put it in motion.
\begin{tasks}
\task You count and find that the frequency is
\unit[0.8]{Hz} (cycles per second).  What is the mass?
\task Find a formula for the mass $m$
given the frequency $\omega$ in \unit{Hz}.
\end{tasks}
\end{exercise}

\begin{exercise}
Suppose we add possible friction to \exerciseref{mv:exwt1}.
Further, suppose you do not know the spring constant, but you have
two reference weights \unit[1]{kg} and \unit[2]{kg} to calibrate your setup.
You put each in motion on your spring and measure the
frequency.  For the \unit[1]{kg}
weight you measured \unit[1.1]{Hz}, for the \unit[2]{kg} weight you
measured \unit[0.8]{Hz}.
\begin{tasks}
\task Find $k$ (spring constant) and $c$ (damping constant).
\task Find a formula for the mass in terms of the frequency in Hz.  \emph{Note that
there may be more than one possible mass for a given frequency.}
\task For an unknown object you measured \unit[0.2]{Hz}, what is the mass of the
object?  Suppose that you know that the mass of the unknown object
is more than a kilogram.
\end{tasks}
\end{exercise}

\begin{exercise}
Suppose you wish to measure the friction a mass of \unit[0.1]{kg} experiences
as it slides along a floor (you wish to find $c$).  You have a spring with
spring constant $k=\unitfrac[5]{N}{m}$.  You take the spring, you attach it
to the mass and fix it to a wall.  Then you pull on the spring and let the
mass go.  You find that the mass oscillates with frequency \unit[1]{Hz}.
What is the friction?
\end{exercise}

\setcounter{exercise}{100}

\begin{exercise}
A mass of $2$ kilograms is on a spring with spring constant $k$ newtons per
meter with no damping.  Suppose the system is at rest and at time $t=0$ the
mass is kicked and starts traveling at 2 meters per second.  How large
does $k$ have to be to so that the mass does not go further than 3 meters
from the rest position?
\end{exercise}
\exsol{%
$k=\nicefrac{8}{9}$ (and larger)
}

\begin{exercise}
Suppose we have an RLC circuit with a resistor of 100 milliohms (0.1 ohms),
inductor of inductance of 50 millihenries (0.05 henries), and a capacitor of 5 farads, with
constant voltage.
\begin{tasks}
\task Set up the ODE equation for the current $I$.
\task Find the general solution.
\task Solve for $I(0) = 10$ and $I'(0) = 0$.
\end{tasks}
\end{exercise}
\exsol{%
\\
a) $0.05 I'' + 0.1 I' + (\nicefrac{1}{5}) I = 0$
\\
b) $x(t) = C e^{-t} \cos(\sqrt{3} \, t - \gamma)$
\\
c) $x(t) = 10 e^{-t} \cos(\sqrt{3} \, t) + \frac{10}{\sqrt{3}} e^{-t}
\sin(\sqrt{3} \, t)$
}

\pagebreak[2]
\begin{exercise}
A \unit[5000]{kg} railcar hits a bumper (a spring) at \unitfrac[1]{m}{s},
and the spring compresses by \unit[0.1]{m}.  Assume no damping.
\begin{tasks}
\task Find $k$.
\task Find out how far does the spring compress when a
\unit[10000]{kg} railcar hits the spring at the same speed.
\task If the spring
would break if it compresses further than \unit[0.3]{m}, what is the maximum
mass of a railcar that can hit it at \unitfrac[1]{m}{s}?
\task What is
the maximum mass of a railcar that can hit the spring without breaking
at \unitfrac[2]{m}{s}?
\end{tasks}
\end{exercise}
\exsol{%
a) $k=500000$,
b) $\frac{1}{5\sqrt{2}} \approx 0.141$,
c) \unit[45000]{kg},
d) \unit[11250]{kg}
}

\begin{exercise}
A mass of $m$ \unit{kg} is on a spring with $k=\unitfrac[3]{N}{m}$ and
$c=\unitfrac[2]{Ns}{m}$.  Find the mass $m_0$ for which there is critical
damping.  If $m < m_0$, does the system oscillate or not, that is, is it
underdamped or overdamped.
\end{exercise}
\exsol{%
$m_0 = \frac{1}{3}$.  If $m < m_0$, then the system is overdamped and will
not oscillate.
}

%%%%%%%%%%%%%%%%%%%%%%%%%%%%%%%%%%%%%%%%%%%%%%%%%%%%%%%%%%%%%%%%%%%%%%%%%%%%%%

\sectionnewpage
\section{Nonhomogeneous equations}
\label{sec:nonhom}

\sectionnotes{2 lectures\EPref{, \S3.5 in \cite{EP}}\BDref{,
\S3.5 and \S3.6 in \cite{BD}}}

\subsection{Solving nonhomogeneous equations}

We have solved linear constant coefficient homogeneous 
equations.
What about nonhomogeneous linear ODEs?
For example, the equations for forced mechanical vibrations.
%Now suppose that we drop the requirement of homogeneity.
%This
%usually corresponds to some outside input to the system we are trying to
%model, like the forcing function for the mechanical vibrations of last
%section.
That is, suppose we have an equation such as
\begin{equation} \label{eq3.5:nh}
y'' + 5y'+ 6y = 2x+1 .
\end{equation}
%We still say this equation is constant coefficient equation.  We
%only require constants in front of the $y''$, $y'$, and $y$.

We will write $Ly = 2x+1$ when the exact form of the operator is not
important.
We solve \eqref{eq3.5:nh} in the following manner.  First, we find the general
solution $y_c$
to the \emph{\myindex{associated homogeneous equation}}
\begin{equation} \label{eq3.5:h}
y'' + 5y'+ 6y = 0 .
\end{equation}
We call $y_c$ the \emph{\myindex{complementary solution}}.
Next, we find a
single \emph{\myindex{particular solution}} $y_p$ to \eqref{eq3.5:nh} in some
way.  Then
\begin{equation*}
y = y_c + y_p
\end{equation*}
is the general solution to \eqref{eq3.5:nh}.  
We have $L y_c = 0$ and $L y_p = 2x+1$.  As
$L$ is a \emph{\myindex{linear operator}}
we verify that $y$ is a solution, $L y = L ( y_c + y_p) = L y_c + L y_p = 0
+ (2x+1)$.  Let us see
why we obtain the \emph{general} solution.

Let $y_p$ and $\tilde{y}_p$ be two different
particular solutions 
to \eqref{eq3.5:nh}.
Write the difference as
$w = y_p - \tilde{y}_p$.  Then plug $w$
into the left hand side of the equation to get
\begin{equation*}
w'' + 5w'+ 6w =
(y_p'' + 5y_p'+ 6y_p) -
(\tilde{y}_p'' + 5\tilde{y}_p'+ 6\tilde{y}_p) =
(2x+1) - (2x+1) = 0 .
\end{equation*}
Using the operator notation the calculation becomes simpler.
%As $L$ is a \emph{\myindex{linear operator}} and so we could just
As $L$ is a linear operator we write
\begin{equation*}
Lw = L(y_p - \tilde{y}_p) =
Ly_p - L\tilde{y}_p =
(2x+1)-(2x+1) = 0 .
\end{equation*}
So $w = y_p - \tilde{y}_p$ is a solution to \eqref{eq3.5:h}, that is
$Lw = 0$.  Any two
solutions of \eqref{eq3.5:nh} differ by a solution to the homogeneous
equation \eqref{eq3.5:h}.  The solution $y = y_c + y_p$ includes \emph{all}
solutions to \eqref{eq3.5:nh},
since $y_c$ is the general solution to the associated homogeneous equation.

\begin{theorem}
Let $Ly=f(x)$ be a linear ODE (not necessarily constant
coefficient).  Let $y_c$ be the complementary solution
(the general
solution to the associated homogeneous equation $Ly = 0$) and let $y_p$
be any particular solution to $Ly=f(x)$.  Then the general
solution to $Ly=f(x)$ is
\begin{equation*}
y = y_c + y_p.
\end{equation*}
\end{theorem}

The moral of the story is that we can find the particular solution in any old
way.  If we find a different particular solution (by a different method,
or simply by guessing),
then we still get the same general solution.
The formula may 
look different, and the constants we will have to choose to
satisfy
the initial conditions may be different, but it is the same solution.

\subsection{Undetermined coefficients}
\index{undetermined coefficients}

The trick is to somehow, in a smart way, guess one particular solution to
\eqref{eq3.5:nh}.  Note that $2x+1$ is a polynomial, and the left hand 
side of the equation will be a polynomial if we let $y$ be a polynomial of
the same degree.  Let us try
\begin{equation*}
y_p = Ax + B .
\end{equation*}
We plug in to obtain
\begin{equation*}
y_p'' + 5y_p'+ 6y_p =
(Ax+B)'' + 5(Ax+B)' + 6(Ax+B) = 
0 + 5A + 6Ax + 6B = 6Ax+ (5A+6B) .
\end{equation*}
So $6Ax+(5A+6B) = 2x+1$.  Therefore, $A = \nicefrac{1}{3}$ and $B = \nicefrac{-1}{9}$.
That means
$y_p = \frac{1}{3}\, x - \frac{1}{9} = \frac{3x-1}{9}$.
Solving the complementary
problem (exercise!) we get
\begin{equation*}
y_c = C_1 e^{-2x} + C_2 e^{-3x}.
\end{equation*}
Hence the general solution to \eqref{eq3.5:nh} is
\begin{equation*}
y = C_1 e^{-2x} + C_2 e^{-3x} + \frac{3x-1}{9} .
\end{equation*}
Now suppose we are further given some initial conditions.  For example, $y(0) = 0$ and
$y'(0) = \nicefrac{1}{3}$.  First find $y' = - 2C_1 e^{-2x} - 3C_2 e^{-3x}
+ \nicefrac{1}{3}$.
Then
\begin{equation*}
0 = y(0) = C_1 + C_2 -\frac{1}{9} , \qquad
\frac{1}{3} = y'(0) = - 2C_1 - 3C_2 + \frac{1}{3} .
\end{equation*}
We solve to get $C_1 = \nicefrac{1}{3}$ and $C_2 = \nicefrac{-2}{9}$.
The particular solution we want is
\begin{equation*}
y(x) = \frac{1}{3} e^{-2x} - \frac{2}{9} e^{-3x} + \frac{3x-1}{9} =
\frac{3 e^{-2x} - 2 e^{-3x} + 3x-1}{9} .
\end{equation*}

\begin{exercise}
Check that $y$ really solves the equation \eqref{eq3.5:nh}
and the given initial conditions.
\end{exercise}

Note: A common mistake is to solve for constants using the initial
conditions with $y_c$ and only add the particular solution $y_p$ after that.
That will \emph{not} work.  You need to first compute $y = y_c + y_p$ and
\emph{only then} solve for the constants using the initial conditions.

\medskip

A right hand side consisting of exponentials, sines, and cosines
can be handled similarly.  For example,
\begin{equation*}
y''+2y'+2y = \cos (2x) .
\end{equation*}
Let us find some $y_p$.  We start by guessing the solution
includes some multiple of $\cos(2x)$.
We may have to also
add a multiple of $\sin (2x)$ to our guess since derivatives of cosine are
sines.  We try
\begin{equation*}
y_p = A \cos (2x) + B \sin (2x) .
\end{equation*}
We plug $y_p$ into the equation and we get
\begin{equation*}
-4 A \cos (2x) - 4 B \sin (2x) 
-4A \sin (2x) + 4B \cos (2x)
+ 2A \cos (2x) + 2B \sin (2x)
= \cos (2x) .
\end{equation*}
The left hand side must equal to right hand side.  We group terms and
we get that $-4A + 4B + 2A = 1$ and
$-4B - 4A + 2B = 0$.  So $-2A+4B =1$ and $2A+B=0$ and hence
$A=\nicefrac{-1}{10}$ and $B=\nicefrac{1}{5}$.  So
\begin{equation*}
y_p = A \cos (2x) + B \sin (2x) = \frac{-\cos (2x) + 2 \sin (2x)}{10} .
\end{equation*}

Similarly, if the right hand side contains exponentials we try
exponentials.  For example, for
%if the equation is (where $L$ is a linear constant coefficient operator)
\begin{equation*}
Ly = e^{3x},
\end{equation*}
we will try $y = A e^{3x}$ as our guess and try to solve for $A$.

\medskip

When the right hand side is a multiple of sines, cosines, exponentials,
and polynomials, we can use the product rule
for differentiation to come up with a guess.  We
need to guess a
form for $y_p$ such that $Ly_p$ is of the same form, and 
has all the terms needed to for 
the right hand side.
For example,
\begin{equation*}
Ly = (1+3x^2)\,e^{-x}\cos (\pi x) .
\end{equation*}
For this equation, we will guess
\begin{equation*}
y_p = (A + Bx + Cx^2)\,e^{-x} \cos (\pi x) + 
(D + Ex + Fx^2)\,e^{-x} \sin (\pi x) .
\end{equation*}
We will plug in and then hopefully get equations that we can solve for
$A$, $B$, $C$, $D$, $E$, and $F$.
As you can see this can make for a very long and
\myindex{tedious} % a bit of fun
calculation very quickly.  C'est \myindex{la vie}! %bit more fun

\medskip

There is one hiccup in all this.  It could be that our guess actually
solves the associated homogeneous equation.  That is, suppose we have
\begin{equation*}
y'' - 9y = e^{3x} .
\end{equation*}
We would love to guess $y = Ae^{3x}$, but if we plug this into the left
hand side of the equation we get
\begin{equation*}
y''-9y = 9Ae^{3x} - 9Ae^{3x} = 0 \not= e^{3x} .
\end{equation*}
There is no way we can choose $A$ to make the left hand side be $e^{3x}$.
The trick in
this case
is to multiply our guess by $x$ to get rid of duplication with the
complementary solution.  That is first we compute $y_c$ (solution to $Ly =
0$)
\begin{equation*}
y_c = C_1 e^{-3x} + C_2 e^{3x} ,
\end{equation*}
and we note that the $e^{3x}$ term is a duplicate with our desired guess.
We modify our guess to $y = Axe^{3x}$ so that there is no
duplication anymore.  Let us try:
$y' = Ae^{3x} + 3Axe^{3x}$ and 
$y'' = 6Ae^{3x} + 9Axe^{3x}$, so
\begin{equation*}
y'' -9y = 6Ae^{3x} + 9Axe^{3x} - 9Axe^{3x} = 
6Ae^{3x} .
\end{equation*}
Thus $6Ae^{3x}$ is supposed to equal $e^{3x}$.  Hence,
$6A = 1$ and so $A=\nicefrac{1}{6}$.  We can now write the general
solution as
\begin{equation*}
y = y_c + y_p = 
C_1 e^{-3x} + C_2 e^{3x} + \frac{1}{6}\,xe^{3x} .
\end{equation*}

\medskip

It is possible that
multiplying by $x$ does not get rid of all
duplication.  For example,
\begin{equation*}
y''-6y'+9y = e^{3x} .
\end{equation*}
The complementary solution is
$y_c = C_1 e^{3x} + C_2 x e^{3x}$.  Guessing $y=A xe^{3x}$
would not get us anywhere.  In this case we want to guess
$y_p = Ax^2e^{3x}$. Basically, we want to multiply our guess by $x$
until all duplication is gone.  \emph{But no more!}  Multiplying too many
times will not work.

\medskip

Finally, what if the right hand side has several terms, such as
\begin{equation*}
Ly = e^{2x} + \cos x .
\end{equation*}
In this case we find $u$ that solves $Lu = e^{2x}$ and $v$ that
solves $Lv = \cos x$ (that is, do each term separately).  Then note
that if $y = u+ v$, then $Ly = e^{2x} + \cos x$.  This is because
$L$ is linear; we have
$Ly = L(u+v) = Lu + Lv = e^{2x} + \cos x$.

\subsection{Variation of parameters}

The method of undetermined coefficients works for many basic
problems that crop up.  But it does not work all the time.  It only works
when the right hand side of the equation $Ly = f(x)$ has finitely many
linearly independent derivatives, so that we can write a guess that consists
of them all.  Some equations are a bit tougher.  Consider
\begin{equation*}
y''+y = \tan x .
\end{equation*}
Each new derivative of $\tan x$ looks completely different and
cannot be written as a linear combination of the previous derivatives.
If we start differentiating $\tan x$, we get:
\begin{equation*}
\sec^2 x, \quad
2\sec^2 x \, \tan x, \quad
4 \sec^2 x \, \tan^2 x + 2 \sec^4 x, \quad
8 \sec^2 x \, \tan^3 x + 16 \sec^4 x \, \tan x, \quad \ldots
\end{equation*}

This equation calls for a different method.  We present the method of
\emph{\myindex{variation of parameters}}, which handles any equation of
the form $Ly = f(x)$,
provided we can solve certain integrals.  For simplicity, we restrict
ourselves to second order constant coefficient equations,
but the method works for higher
order equations just as well (the computations become more
\myindex{tedious}). % a bit of fun 
The method also works for equations with nonconstant coefficients,
provided we can solve the associated homogeneous equation.

Perhaps it is best to explain this method by example.
Let us try to solve the equation
\begin{equation*}
Ly = y''+y = \tan x .
\end{equation*}
First we find the complementary solution (solution to $Ly_c = 0$).  
We get $y_c = C_1 y_1 + C_2 y_2$, where $y_1 = \cos x$ and $y_2 = \sin x$.
To find a particular solution to the nonhomogeneous equation we try
\begin{equation*}
y_p = y = u_1 y_1 + u_2 y_2 ,
\end{equation*}
where $u_1$ and $u_2$ are \emph{functions} and not constants.
We are trying to satisfy $Ly = \tan x$.  That gives us one condition on the
functions $u_1$ and $u_2$.
Compute (note the product rule!)
\begin{equation*}
y' = (u_1' y_1 + u_2' y_2) + (u_1 y_1' + u_2 y_2').
\end{equation*}
We can still
impose one more condition at our discretion to simplify computations (we have two unknown functions,
so we should be allowed two conditions).  We require that
$(u_1' y_1 + u_2' y_2) = 0$.  This makes computing the second derivative
easier.
\begin{align*}
& y' = u_1 y_1' + u_2 y_2' , \\
& y'' = (u_1' y_1' + u_2' y_2') + (u_1 y_1'' + u_2 y_2'') .
\end{align*}
Since $y_1$ and $y_2$ are solutions to $y''+y = 0$, we know
that $y_1'' = - y_1$
and $y_2'' = - y_2$.
(Note: If the equation was instead $y''+p(x)y' +q(x)y = 0$ we would have
$y_i'' = -p(x)y_i' -q(x)y_i$.) So 
\begin{equation*}
y'' = (u_1' y_1' + u_2' y_2') - (u_1 y_1 + u_2 y_2) .
\end{equation*}
We have $(u_1 y_1 + u_2 y_2) = y$ and so
\begin{equation*}
y'' = (u_1' y_1' + u_2' y_2') - y ,
\end{equation*}
and hence
\begin{equation*}
y'' + y = Ly = u_1' y_1' + u_2' y_2' .
\end{equation*}
For $y$ to satisfy $Ly = f(x)$ we must have
$f(x) = u_1' y_1' + u_2' y_2'$.

So what we need to solve are the two equations (conditions) we imposed
on $u_1$ and $u_2$:
\begin{equation*}
\boxed{~~
\begin{aligned}
& u_1' y_1 + u_2' y_2 = 0 ,\\
& u_1' y_1' + u_2' y_2' = f(x) .
\end{aligned}
~~}
\end{equation*}
We now solve for $u_1'$ and $u_2'$ in terms of $f(x)$, $y_1$ and $y_2$.
We always get these formulas for any $Ly = f(x)$, where $Ly =
y''+p(x)y'+q(x)y$.  There is a general
formula for the solution we can just plug into, but it is better to
just repeat what we do below.  In our case the two equations are
\begin{align*}
u_1' \cos (x) + u_2' \sin (x) &= 0 ,\\
-u_1' \sin (x) + u_2' \cos (x) &= \tan (x) .
\end{align*}
Hence
\begin{align*}
u_1' \cos (x) \sin (x) + u_2' \sin^2 (x) & = 0 ,\\
-u_1' \sin (x) \cos (x) + u_2' \cos^2 (x) & = \tan (x) \cos (x) = \sin (x) .
\end{align*}
And thus
\begin{align*}
& u_2' \bigl(\sin^2 (x) + \cos^2 (x)\bigr) = \sin (x) , \\
& u_2' = \sin (x) , \\
& u_1' = \frac{- \sin^2 (x)}{\cos (x)} = - \tan (x) \sin (x) .
\end{align*}
We integrate $u_1'$ and $u_2'$ to get $u_1$ and $u_2$.
\begin{align*}
& u_1 = \int u_1'~dx 
= \int - \tan (x) \sin (x)~dx
= \frac{1}{2}
\ln \left\lvert \frac{\sin (x)-1}{\sin (x) + 1} \right\rvert
+ \sin (x) , \\
& u_2 = \int u_2'~dx 
= \int \sin (x)~dx = -\cos (x) .
\end{align*}
So our particular solution is
\begin{multline*}
y_p = u_1 y_1 + u_2 y_2 =
\frac{1}{2} \cos (x) \ln \left\lvert \frac{\sin (x)-1}{\sin (x) + 1}
\right\rvert
+ \cos (x) \sin (x)
-\cos (x) \sin (x)
= \\ =
\frac{1}{2} \cos (x) \ln \left\lvert \frac{\sin (x)-1}{\sin (x) + 1}
\right\rvert .
\end{multline*}
The general solution to $y'' + y = \tan x$ is, therefore,
\begin{equation*}
y = C_1 \cos (x) + C_2 \sin (x) +
\frac{1}{2} \cos (x) \ln \left\lvert \frac{\sin (x)-1}{\sin (x) + 1}
\right\rvert .
\end{equation*}

\subsection{Exercises}

\begin{exercise}
Find a particular solution of
$y''-y' -6y = e^{2x}$.
\end{exercise}

\begin{exercise}
Find a particular solution of
$y''-4y' +4y = e^{2x}$.
\end{exercise}

\begin{exercise}
Solve the initial value problem
$y''+9y = \cos (3x) + \sin (3x)$ for $y(0) = 2$, $y'(0) = 1$.
\end{exercise}

\begin{exercise}
Set up the form of the particular solution but do not solve
for the coefficients for $y^{(4)}-2y'''+y'' = e^x$.
\end{exercise}

\begin{exercise}
Set up the form of the particular solution but do not solve
for the coefficients for $y^{(4)}-2y'''+y'' = e^x + x + \sin x$.
\end{exercise}

\begin{exercise}
{\ }
\begin{tasks}
\task Using variation of parameters find a particular solution of
$y''-2y'+y = e^x$.
\task Find a particular solution using undetermined
coefficients.
\task Are the two solutions you found the same?
See also \exerciseref{exercise:diffvarparunder}.
\end{tasks}
\end{exercise}

\begin{exercise}
Find a particular solution of
$y''-2y' +y = \sin (x^2)$.  It is OK to leave the answer as a definite
integral.
\end{exercise}

\begin{exercise}
For an arbitrary constant $c$ find a particular solution
to $y''-y=e^{cx}$.  Hint: Make sure to handle every possible real $c$.
\end{exercise}

\begin{exercise} \label{exercise:diffvarparunder}
{\ }
\begin{tasks}
\task Using variation of parameters find a particular solution of
$y''-y = e^x$.
\task Find a particular solution using undetermined
coefficients.
\task Are the two solutions you found the same?
What is going on?
\end{tasks}
\end{exercise}

\setcounter{exercise}{100}

\begin{exercise}
Find a particular solution to $y''-y'+y=2\sin(3x)$
\end{exercise}
\exsol{%
$y=\frac{-16\sin(3x)+6\cos(3x)}{73}$
}

\begin{samepage}
\begin{exercise}
{\ }
\begin{tasks}
\task Find a particular solution to $y''+2y=e^x + x^3$.
\task Find the general solution.
\end{tasks}
\end{exercise}
\end{samepage}
\exsol{%
a) $y=\frac{2e^x+3x^3-9x}{6}$,
b) $y=C_1 \cos(\sqrt{2} x) + C_2 \sin(\sqrt{2} x) + \frac{2e^x+3x^3-9x}{6}$
}

\begin{exercise}
Solve $y''+2y'+y = x^2$, $y(0)=1$, $y'(0)=2$.
\end{exercise}
\exsol{%
$y(x) = x^2-4 x+6+e^{-x}(x-5)$
}

\begin{exercise}
Use variation of parameters to
find a particular solution of $y''-y = \frac{1}{e^x+e^{-x}}$.
\end{exercise}
\exsol{%
$y = \frac{2xe^x-(e^x+e^{-x})\log(e^{2x}+1)}{4}$
}

\begin{exercise}
For an arbitrary constant $c$ find the general solution
to $y''-2y=\sin(x+c)$.
\end{exercise}
\exsol{%
$y=\frac{-\sin(x+c)}{3}+C_1 e^{\sqrt{2}\,x}+C_2 e^{-\sqrt{2}\,x}$
}



%%%%%%%%%%%%%%%%%%%%%%%%%%%%%%%%%%%%%%%%%%%%%%%%%%%%%%%%%%%%%%%%%%%%%%%%%%%%%%

\sectionnewpage
\section{Forced oscillations and resonance} \label{forcedo:section}

\sectionnotes{2 lectures\EPref{, \S3.6 in \cite{EP}}\BDref{,
\S3.8 in \cite{BD}}}

\begin{mywrapfigsimp}{2.0in}{2.3in}
\noindent
\inputpdft{massfigforce}
\end{mywrapfigsimp}
Let us return back to the example of a mass on a spring.  We examine
the case of forced oscillations, which we did not yet handle.  That is, we consider the equation
\begin{equation*}
mx'' + cx' + kx = F(t) ,
\end{equation*}
for some nonzero $F(t)$.  The setup
is again: $m$ is mass, $c$ is friction, $k$ is the spring constant, and
$F(t)$ is an external force acting on the mass.

We are interested in periodic
forcing, such as noncentered rotating parts, or perhaps loud sounds, or
other sources of periodic force.  Once we learn about Fourier series in
\chapterref{FS:chapter}, we
will see that we cover all periodic functions
by simply considering $F(t) = F_0 \cos (\omega t)$ (or sine instead of cosine,
the calculations are essentially the same).

\subsection{Undamped forced motion and resonance}

First let us consider undamped ($c=0$) motion for simplicity.
We have the equation
\begin{equation*}
mx'' + kx = F_0 \cos (\omega t) .
\end{equation*}
This equation has the complementary solution (solution to the associated homogeneous
equation)
\begin{equation*}
x_c = C_1 \cos (\omega_0 t) + C_2 \sin (\omega_0 t) ,
\end{equation*}
where $\omega_0 = \sqrt{\nicefrac{k}{m}}$ is the
\emph{\myindex{natural frequency}} (angular).  It is the frequency
at which the system \myquote{wants to oscillate} without external interference.

Let us suppose that $\omega_0 \not= \omega$.  We try the solution
$x_p = A \cos (\omega t)$ and solve for $A$.  Note that we do not need a sine
in our trial solution as after plugging in we only have cosines.
If you include a sine, it is fine; you will find that its
coefficient is zero (I could not find a second rhyme).

We solve using the method of undetermined coefficients.  We find that
\begin{equation*}
x_p = \frac{F_0}{m(\omega_0^2 - \omega^2)} \cos (\omega t) .
\end{equation*}
We leave it as an exercise to do the algebra required.

The general solution is
\begin{equation*}
\boxed{
~~
x = C_1 \cos (\omega_0 t) + C_2 \sin (\omega_0 t) +
\frac{F_0}{m(\omega_0^2 - \omega^2)} \cos (\omega t) .
~~
}
\end{equation*}
Written another way
\begin{equation*}
x = C \cos (\omega_0 t - \gamma) +
\frac{F_0}{m(\omega_0^2 - \omega^2)} \cos (\omega t) .
\end{equation*}
The solution is a superposition of two cosine waves at different frequencies.

\begin{example}
Take
\begin{equation*}
0.5 x'' + 8 x = 10 \cos (\pi t), \qquad x(0)=0, \qquad x'(0)=0 .
\end{equation*}

Let us compute.  First we read off the parameters:
$\omega = \pi$, $\omega_0 = \sqrt{\nicefrac{8}{0.5}} = 4$, $F_0 = 10$,
$m=0.5$.  The general solution is
\begin{equation*}
x = C_1 \cos (4 t) + C_2 \sin (4 t) +
\frac{20}{16 - \pi^2} \cos (\pi t) .
\end{equation*}

%mbxSTARTIGNORE
\begin{mywrapfig}{3.25in}
\capstart
\diffyincludegraphics{width=3in}{width=4.5in}{3-6-beating}
\caption{Graph of
$\frac{20}{16 - \pi^2} \bigl( \cos (\pi t)- \cos (4 t) \bigr)$.\label{3.6:beatingfig}}
\end{mywrapfig}
%mbxSTARTIGNORE
%
% make sure the MBX below is synced!
%


Solve for $C_1$ and $C_2$ using the initial conditions.
It is easy to see that
$C_1 = \frac{-20}{16 - \pi^2}$ and $C_2 = 0$.  Hence
\begin{equation*}
x = 
\frac{20}{16 - \pi^2} \bigl( \cos (\pi t)- \cos (4 t) \bigr) .
\end{equation*}

%mbxlatex \begin{myfig}
%mbxlatex \capstart
%mbxlatex \diffyincludegraphics{width=3in}{width=4.5in}{3-6-beating}
%mbxlatex \caption{Graph of
%mbxlatex $\frac{20}{16 - \pi^2} \bigl( \cos (\pi t)- \cos (4 t) \bigr)$.\label{3.6:beatingfig}}
%mbxlatex \end{myfig}

Notice the \myquote{beating} behavior\index{beating}
in \figurevref{3.6:beatingfig}.  First
use the 
trigonometric identity
\begin{equation*}
2\,\sin \left( \frac{A-B}{2} \right) \,\sin \left( \frac{A+B}{2} \right) =
\cos B -\cos A 
\end{equation*}
to get 
\begin{equation*}
x = 
\frac{20}{16 - \pi^2} \left( 2 \, \sin \left(\frac{4-\pi}{2} \, t \right)
\,\sin \left( \frac{4+\pi}{2}\, t \right) \right) .
\end{equation*}
The function $x$ is a high frequency wave modulated by a low frequency
wave.
\end{example}

Now suppose $\omega_0 = \omega$.  Obviously, we cannot try
the solution $A \cos (\omega t)$ and then use the method of undetermined
coefficients.  We notice that $\cos (\omega t)$ solves the associated
homogeneous equation.  Therefore,
we 
try $x_p = A t \cos (\omega t) + B t \sin (\omega t)$.  This time we need
the sine
term, since the second derivative of $t \cos (\omega t)$ contains sines.
We write the equation
\begin{equation*}
x'' + \omega^2 x = \frac{F_0}{m} \cos ( \omega t) .
\end{equation*}
Plugging $x_p$ into the left hand side we get
\begin{equation*}
2 B \omega \cos (\omega t) - 2 A \omega \sin (\omega t) = 
\frac{F_0}{m} \cos (\omega t) .
\end{equation*}
Hence $A = 0$ and $B = \frac{F_0}{2m\omega}$.  Our particular solution is
$\frac{F_0}{2m\omega} \, t \sin (\omega t)$ and our general solution is
\begin{equation*}
x = C_1 \cos (\omega t) + C_2 \sin (\omega t)
+ \frac{F_0}{2m\omega} \, t \sin (\omega t) .
\end{equation*}

The important term is the last one (the particular solution we found).  
This term grows without bound as $t \to \infty$.  In fact it
oscillates 
between $\frac{F_0 t}{2m\omega}$ and
$\frac{- F_0 t}{2m\omega}$.  The first two terms only oscillate between
$\pm\sqrt{C_1^2 + C_2^2}$, which becomes smaller and smaller in proportion to
the oscillations of the last term as $t$ gets larger.  In
\figurevref{3.6:resonancefig} we see the graph with $C_1=C_2=0$, $F_0 = 2$,
$m=1$, $\omega = \pi$.

\begin{mywrapfig}{3.25in}
\capstart
\diffyincludegraphics{width=3in}{width=4.5in}{3-6-resonance}
\caption{Graph of
$\frac{1}{\pi} t \sin (\pi t)$.\label{3.6:resonancefig}}
\end{mywrapfig}

By forcing the system in just the right frequency we produce very wild
oscillations.  This kind of behavior is called \emph{\myindex{resonance}} or
perhaps
\emph{\myindex{pure resonance}}.  Sometimes resonance is
desired.  For
example, remember when as a kid you could start swinging by just moving back
and forth on the swing seat in the \myquote{correct frequency}?  You were trying to
achieve resonance.  The force of each one of your moves was small, but after a
while it produced large swings.

On the other hand resonance can be destructive.
In an earthquake some buildings collapse while
others may be relatively undamaged.  This is due to different buildings
having different resonance frequencies.  So figuring out the resonance
frequency can be very important.

A common (but wrong) example of destructive force of resonance is the Tacoma
Narrows bridge failure.  It turns out there was a different
phenomenon at play%
\footnote{K.\ Billah and R.\ Scanlan, \emph{Resonance, Tacoma Narrows
Bridge Failure, and Undergraduate Physics Textbooks}, American Journal of
Physics, 59(2), 1991, 118--124,
\url{http://www.ketchum.org/billah/Billah-Scanlan.pdf}}.

\subsection{Damped forced motion and practical resonance}

In real life things are not as simple as they were above.  There is,
of course, some damping.  Our equation becomes
\begin{equation} \label{3.6:deq}
mx'' + cx' + kx = F_0 \cos (\omega t) ,
\end{equation}
for some $c > 0$.  We solved the homogeneous problem before.  We let
\begin{equation*}
p = \frac{c}{2m},  \qquad \omega_0 = \sqrt{\frac{k}{m}} .
\end{equation*}
We replace equation \eqref{3.6:deq} with
\begin{equation*}
x'' + 2px' + \omega_0^2x = \frac{F_0}{m} \cos (\omega t) .
\end{equation*}
The roots of the characteristic equation of the associated
homogeneous problem are $r_1,r_2 = -p \pm \sqrt{p^2 - \omega_0^2}$.  The form
of the general solution of the associated homogeneous equation
depends on the sign of $p^2 - \omega_0^2$, or
equivalently on the sign of $c^2 - 4km$, as we have
seen before.  That is,
\begin{equation*}
x_c =
\begin{cases}
C_1 e^{r_1 t} + C_2 e^{r_2 t} & \text{if } \; c^2 > 4km , \\
C_1 e^{-p t} + C_2 t e^{-p t} & \text{if } \; c^2 = 4km , \\
e^{-p t} \bigl( C_1 \cos (\omega_1 t) + C_2 \sin (\omega_1 t) \bigr) &
  \text{if } \; c^2 < 4km ,
\end{cases}
\end{equation*}
where $\omega_1 = \sqrt{\omega_0^2 - p^2}$.  In any case, we see that
$x_c(t) \to 0$ as $t \to \infty$.  Furthermore,
there can be no conflicts when trying to solve for the
undetermined coefficients by trying $x_p = A \cos (\omega t)
+ B \sin (\omega t)$.
%Hence, we will never get the kind of catastrophic scenario we have seen
%before.
%A slightly different notion of \myquote{resonance} will still occur.
Let us plug
in and solve for $A$ and $B$.
We get (the \myindex{tedious} % a bit of fun
details are left to reader)
\begin{equation*}
\bigl((\omega_0^2  - \omega^2)B - 2\omega p A\bigr) \sin (\omega t)
+
\bigl((\omega_0^2  - \omega^2)A + 2\omega p B\bigr) \cos (\omega t)
=
\frac{F_0}{m} \cos (\omega t) .
\end{equation*}

We solve for $A$ and $B$:
\begin{align*}
& A=\frac{(\omega_0^2-\omega^2) F_0}
{m{(2\omega p)}^2+m{(\omega_0^2-\omega^2)}^2} , \\
& B=\frac{2 \omega p F_0}
{m{(2\omega p)}^2+m{(\omega_0^2-\omega^2)}^2} .
\end{align*}
We also compute $C = \sqrt{A^2+B^2}$
to be
\begin{equation*}
C = \frac{F_0}{m \sqrt{{(2\omega p)}^2+{(\omega_0^2-\omega^2)}^2}} .
\end{equation*}
Thus our particular solution is
\begin{equation*}
x_p = 
\frac{(\omega_0^2-\omega^2) F_0}
{m{(2\omega p)}^2+m{(\omega_0^2-\omega^2)}^2} \cos (\omega t) +
\frac{2 \omega p F_0}
{m{(2\omega p)}^2+m{(\omega_0^2-\omega^2)}^2} \sin (\omega t) .
\end{equation*}
Or in the alternative notation we have amplitude $C$ and phase shift $\gamma$
where (if $\omega \not= \omega_0$)
\begin{equation*}
\tan \gamma = \frac{B}{A} = \frac{2\omega p}{\omega_0^2-\omega^2} .
\end{equation*}
Hence
\begin{equation*}
\boxed{~~
x_p = 
\frac{F_0}{m \sqrt{{(2\omega p)}^2+{(\omega_0^2-\omega^2)}^2}} 
\cos ( \omega t - \gamma ) .
~~}
\end{equation*}
If $\omega = \omega_0$, then $A=0$, $B = C = \frac{F_0}{2m\omega p}$,
and $\gamma = \nicefrac{\pi}{2}$.

The exact formula is not as important as the idea.  Do not memorize
the above formula, you should instead remember the ideas involved.
For a different forcing function $F$, you will get a different formula
for $x_p$.
So there is no point in memorizing this specific
formula.  You can always recompute it later or look it up if you really need
it.

\medskip

For reasons we will explain in a moment, we call $x_c$ the
\emph{\myindex{transient solution}}
and denote it by $x_{tr}$.  We call the
$x_p$ we found above the \emph{\myindex{steady periodic solution}} and denote it
by $x_{sp}$.
The general solution to our problem is
\begin{equation*}
x = x_c + x_p = x_{tr} + x_{sp} .
\end{equation*}

%mbxSTARTIGNORE
\begin{mywrapfig}{3.25in}
\capstart
\diffyincludegraphics{width=3in}{width=4.5in}{3-6-transbeh}
\caption{Solutions with different initial conditions for parameters
$k=1$, $m=1$, $F_0 = 1$, $c=0.7$, and $\omega=1.1$.\label{3.6:transbehfig}}
\end{mywrapfig}
%mbxENDIGNORE
%
% make sure the MBX below is synced!
%

The transient solution $x_c = x_{tr}$ goes to zero as $t \to \infty$,
as all the terms involve an exponential with a negative exponent.  So
for large $t$, the effect of $x_{tr}$ is negligible and we see essentially
only $x_{sp}$.
Hence the name \emph{transient}.
Notice that $x_{sp}$ involves no arbitrary constants, and
the initial conditions only affect $x_{tr}$.  This means that the effect
of the initial conditions is negligible after some period of time.
Because of this behavior,
we might as well focus on the
steady periodic solution and ignore the transient solution.  See
\figurevref{3.6:transbehfig} for a graph given several different initial conditions.

%mbxlatex \begin{myfig}
%mbxlatex \capstart
%mbxlatex \diffyincludegraphics{width=3in}{width=4.5in}{3-6-transbeh}
%mbxlatex \caption{Solutions with different initial conditions for parameters
%mbxlatex $k=1$, $m=1$, $F_0 = 1$, $c=0.7$, and $\omega=1.1$.\label{3.6:transbehfig}}
%mbxlatex \end{mywrapfig}

The speed at which $x_{tr}$ goes to zero depends on $p$ (and
hence $c$).  The
bigger $p$ is (the bigger $c$ is), the \myquote{faster} $x_{tr}$ becomes negligible. 
So the smaller the damping, the longer the \myquote{transient region.}  This agrees
with the observation that when $c=0$, the initial conditions affect the
behavior for all time (i.e.\ an infinite \myquote{transient region}).

\medskip

Let us describe what we mean by resonance when damping is present.
Since there were no conflicts when solving with undetermined coefficient,
there is no term that goes to infinity.  We look at the
maximum value of the amplitude of the steady periodic solution.
Let $C$ be the amplitude of $x_{sp}$.
If we plot $C$ as a function of $\omega$ (with all other
parameters fixed) we can find its maximum.
We call the $\omega$ that achieves this maximum
the \emph{\myindex{practical resonance frequency}}.
We call the maximal amplitude $C(\omega)$
the \emph{\myindex{practical resonance amplitude}}.
Thus when damping is present we talk of \emph{\myindex{practical resonance}}
rather than pure resonance.
A sample plot for three different
values of $c$ is given in \figurevref{3.6:pracresfig}.  As you can see the
practical resonance amplitude grows as damping gets smaller, and 
practical resonance can disappear altogether when damping is large.

\begin{myfig}
\capstart
\diffyincludegraphics{width=3in}{width=4.5in}{3-6-pracres}
\caption{Graph of $C(\omega)$ showing practical resonance with parameters
$k=1$, $m=1$, $F_0 = 1$. The top line is with $c=0.4$, the middle line with
$c=0.8$, and the bottom line with
$c=1.6$.\label{3.6:pracresfig}}
\end{myfig}

To find the maximum we need to find the derivative $C'(\omega)$.
Computation shows
\begin{equation*}
C'(\omega) =
\frac{- 2\omega( 2p^2+\omega^2-\omega_0^2)F_0}
{m {\bigl({(2\omega p)}^2+{(\omega_0^2-\omega^2)}^2\bigr)}^{3/2}} .
\end{equation*}
This is zero either when $\omega = 0$ or when
$2p^2+\omega^2-\omega_0^2 = 0$.  In other words, $C'(\omega) = 0$ when
\begin{equation*}
\boxed{
~~
\omega = \sqrt{\omega_0^2 - 2p^2} \quad \text{or} \quad \omega = 0 .
~~
}
\end{equation*}
If $\omega_0^2 - 2p^2$ is positive, then
$\sqrt{\omega_0^2 - 2p^2}$ is the practical resonance frequency (that is the
point where $C(\omega)$ is maximal).  This follows by the first derivative
test for example as then $C'(\omega) > 0$ for small $\omega$ in this case.
If on the other hand $\omega_0^2 - 2p^2$ is not positive, then
$C(\omega)$ achieves its maximum at
$\omega=0$, and
there is no practical resonance since we assume $\omega > 0$
in our system.  In this case the amplitude gets larger as the forcing
frequency gets smaller.

If practical resonance occurs, the frequency is smaller than
$\omega_0$.  As the damping $c$ (and hence $p$) becomes smaller, the
practical resonance frequency
goes to $\omega_0$.  So when damping is very
small, $\omega_0$ is a good estimate of the practical resonance frequency.  This
behavior
agrees with the observation that when $c=0$, then $\omega_0$ is the resonance
frequency.

Another interesting observation to make is that when $\omega \to \infty$,
then $C \to 0$.  This means that if the forcing frequency gets too high it
does not manage to get the mass moving in the mass-spring system.  This is
quite rasonable intuitively.
If we wiggle back and forth really fast while sitting on a swing, we will
not get it moving at all, no matter how forceful.  Fast
vibrations just cancel each other out before the mass has any chance of
responding by moving one way or the other.

The behavior is more complicated if the forcing function is not an
exact cosine wave, but for example a \myindex{square wave}.
A general periodic function will be the sum (superposition) of many
cosine waves of different frequencies.
The reader is encouraged to come
back to this section once we have learned about the Fourier series.

\subsection{Exercises}

\begin{exercise}
Derive a formula for $x_{sp}$ if the equation is
$m x'' + c x' + kx = F_0 \sin (\omega t)$.  Assume $c > 0$.
\end{exercise}

\begin{exercise}
Derive a formula for $x_{sp}$ if the equation is
$m x'' + c x' + kx = F_0 \cos (\omega t) + F_1 \cos (3\omega t)$.
Assume $c > 0$.
\end{exercise}

\begin{exercise}
Take $m x'' + c x' + kx = F_0 \cos (\omega t)$.
Fix $m > 0$, $k > 0$, and $F_0 > 0$.  Consider the function $C(\omega)$.
For what values of $c$ (solve in terms of $m$, $k$, and $F_0$) will there be no
practical resonance (that is, for what values of $c$ is there no maximum of
$C(\omega)$ for $\omega > 0$)?
\end{exercise}

\begin{exercise}
Take $m x'' + c x' + kx = F_0 \cos (\omega t)$.
Fix $c > 0$, $k > 0$, and $F_0 > 0$.  Consider the function $C(\omega)$.
For what values of $m$ (solve in terms of $c$, $k$, and $F_0$) will there be no
practical resonance (that is, for what values of $m$ is there no maximum of
$C(\omega)$ for $\omega > 0$)?
\end{exercise}

\begin{exercise}
A water tower in an earthquake acts as a mass-spring system.
Assume that the container on top is full and the water does not move around.
The container then acts as the mass and the support acts as the spring, where
the induced vibrations are horizontal.  The container with water
has a mass of $m=\unit[10,000]{kg}$.  It takes a force of 1000 newtons
to displace the container 1 meter.  For simplicity assume no friction.
When the earthquake hits the water tower is at rest (it is not moving).
%
The earthquake induces an external force 
$F(t) = m A \omega^2 \cos (\omega t)$.
\begin{tasks}
\task
What is the natural frequency of the water tower?
\task
If $\omega$ is not the natural frequency, find a formula for the maximal
amplitude of the resulting oscillations of the water container (the maximal
deviation from the rest position).  The motion will be a high frequency wave
modulated by a low frequency wave, so simply find the constant in front of the
sines.
\task
Suppose $A = 1$ and an earthquake with frequency 0.5 cycles per second
comes.  What is the amplitude of the oscillations?  Suppose that if the water
tower moves more than 1.5 meter from the rest position, the tower collapses.
Will the tower collapse?
\end{tasks}
\end{exercise}


\setcounter{exercise}{100}

\begin{exercise}
A mass of \unit[4]{kg} on a spring with $k=\unitfrac[4]{N}{m}$ and a damping
constant $c=\unitfrac[1]{Ns}{m}$.
Suppose that $F_0 = \unit[2]{N}$.  Using forcing function $F_0 \cos (\omega t)$,
find the $\omega$ that causes practical resonance and find the amplitude.
\end{exercise}
\exsol{%
$\omega = \frac{\sqrt{31}}{4\sqrt{2}} \approx 0.984$ \quad
$C(\omega) = \frac{16}{3\sqrt{7}} \approx 2.016$
}

\begin{exercise}
Derive a formula for $x_{sp}$ for
$mx''+cx'+kx = F_0 \cos(\omega t) + A$,
where $A$ is some constant.  Assume $c > 0$.
\end{exercise}
\exsol{%
$x_{sp} = 
\frac{(\omega_0^2-\omega^2) F_0}
{m{(2\omega p)}^2+m{(\omega_0^2-\omega^2)}^2} \cos (\omega t) +
\frac{2 \omega p F_0}
{m{(2\omega p)}^2+m{(\omega_0^2-\omega^2)}^2} \sin (\omega t)
+ \frac{A}{k}$,
where
$p = \frac{c}{2m}$ and $\omega_0 = \sqrt{\frac{k}{m}}$.
}

\begin{exercise}
Suppose there is no damping in a mass and spring system with
$m = 5$, $k= 20$, and $F_0 = 5$.  Suppose $\omega$ is chosen
to be precisely the resonance frequency.
\begin{tasks}
\task
Find $\omega$.
\task
Find the amplitude of the oscillations at time $t=100$, given the system is at
rest at $t=0$.
\end{tasks}
\end{exercise}
\exsol{%
a) $\omega = 2$,
b) $25$
}
