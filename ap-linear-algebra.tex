\chapter{Linear algebra} \label{linalg:appendix}

%%%%%%%%%%%%%%%%%%%%%%%%%%%%%%%%%%%%%%%%%%%%%%%%%%%%%%%%%%%%%%%%%%%%%%%%%%%%%%

\section{Vectors, mappings, and linear systems}
\label{vecsandmaps:section}

\sectionnotes{? lectures}

In real life, there is most often more than one variable.
So we wish to organize dealing with multiple variables in a consistent
manner, and in particular organize dealing with linear equations and linear
mappings, as those are rather easy to handle if done properly.
Mathematicians joke that
\myquote{to an engineer every problem is linear, and everything is a
matrix.}
And well, they (the engineers) are not wrong. 
Quite often, solving an engineering problem really is figuring out the
right finite dimensional linear problem to solve, which can be then
solved with some matrix manipulation.
Most importantly linear problems are the ones that we know how to solve
and we have lots of tools to solve them.
For engineers, mathematicians, physicists, and anybody in a technical
field it is absolutely vital to learn linear algebra.

As motivation, suppose we wish to solve
\begin{equation*}
\begin{aligned}
& x-y = 2 , \\
& 2x+y = 4 ,
\end{aligned}
\end{equation*}
for $x$ and $y$.
What you could do is add the equations together to find
\begin{equation*}
x+2x-y+y = 2+4, \qquad \text{or} \qquad 3x = 6 .
\end{equation*}
And now we find $x=3$.  Once we have that, we can plug in $x=2$ into the
first equation to find $2-y=2$, so $y=0$.  OK\@, that was easy.  What is all
this fuss about linear equations.  Well, try doing this if you have
5000 unknowns\footnote{One of the downsides of making everything look like a
linear problem is that the number of variables tends to become huge.}.
Also, we have such equations not of just numbers,
but of functions and derivatives of functions in differential equations.
Clearly we need a more systematic way of doing things.
A consequence of making things systematic and simpler to write down
is that it also becomes much easier to have computers do the work for us.
Computers are rather stupid, they do not think,
but are very good at doing lots of repetitive
tasks precisely, as long as we figure out a systematic way for them to
perform the tasks.

\subsection{Vectors and operations on vectors}

Let us start with vectors.  Consider $n$ real numbers as an
$n$-tuple
\begin{equation*}
(x_1,x_2,\ldots,x_n). 
\end{equation*}
These points span the so-called
\emph{$n$-dimensional space}\index{n-dimensional space@$n$-dimensional space},
often denoted by ${\mathbb R}^n$.
Sometimes we call this the $n$-dimensional
\emph{\myindex{euclidean space}}\footnote{FIXME: Named after Euclid}.
For example in two dimensions, ${\mathbb R}^2$ is the
\emph{\myindex{cartesian plane}}\footnote{FIXME: Named after Descartes}.
So we have say
the point $(1,2)$, which is one unit to the right and two units up from the
origin.

When we do algebra with these $n$-tuples of numbers we call them
\emph{vectors}\index{vector}.  Mathematicians are keen on separating
what is a vector and what is a point of the space or in the plane,
and it turns out
to be an important distinction, however, for the purposes of linear algebra
we can think of everything being represented by a vector.
A way to think of a vector, which is especially useful in calculus
and differential equations, is an arrow.  It is an object that has
a direction and a magnitude.  For example, the vector $(1,2)$
is the arrow from the origin to the point $(1,2)$ in the plane.
See figure FIXME.

FIXME: figure.

Since vectors are arrows, when we want to talk about a vector and we give it
a name, we will write a little arrow above it:
\begin{equation*}
\vec{x}
\end{equation*}
Another popular notations is $\mathbf{x}$, although we will use the arrows.
Mathematicians often don't even write the arrows.  A mathematician would
write $x$ and just remember that $x$ is a vector and not a number.
Just like you remember that Bob is your uncle, and you don't have to
keep repeating \myquote{Uncle Bob.}  In this book, however, we will call Bob
uncle, and write vectors with arrows above them.

For reasons that will become clear in the next section, it will often be
useful to write vectors as so-called
\emph{column vectors}\index{column vector}:
\begin{equation*}
\vec{x} = 
\begin{bmatrix}
x_{1} \\ x_2 \\ \vdots \\ x_n
\end{bmatrix} .
\end{equation*}
Don't worry, it is just a different way of writing the same thing, and
it will be useful later.  For example, the vector $(1,2)$ can be written as
\begin{equation*}
\begin{bmatrix}
1 \\ 2
\end{bmatrix} .
\end{equation*}

The fact that we write arrows above vectors will allow us to write several
vectors $\vec{x}_1$, $\vec{x}_2$, etc., without confusing these with the
components of some other vector $\vec{x}$.


Now where does the algebra come in?  Well, arrows are easy to add, subtract
and multiply by numbers.  First adding.  If we have two arrows, we simply
move along one, and then along the other.  See Figure FIXME.

FIXME: figure.

It is rather easy to see what it does to the numbers that represent the
vectors.  Suppose I want to add $(1,2)$ to $(-1,3)$.  So I travel along $(1,2)$
and then travel along $(-1,3)$.
What I did was travel one unit right, two units up, and then
I travelled one unit left (the negative one), and three units up.  That
means that I added up at $(1-1,2+3) = (0,5)$.  And that's how addition
always works:
\begin{equation*}
\begin{bmatrix}
x_{1} \\ x_2 \\ \vdots \\ x_n
\end{bmatrix} +
\begin{bmatrix}
y_{1} \\ y_2 \\ \vdots \\ y_n
\end{bmatrix} =
\begin{bmatrix}
x_1 + y_{1} \\ x_2+ y_2 \\ \vdots \\ x_n + y_n
\end{bmatrix} .
\end{equation*}

Another intuitive thing to do to a vector is to scale it.
We represent this by multiplication of a number with a vector.
Because of this, when we wish to distinguish between vectors and numbers, we
call the numbers \emph{scalars}\index{scalar}.
For example,
suppose we want to travel three times further.  If the vector is $(1,2)$,
travelling 3 times further means going 3 units to the right and 6 units up,
so we get the vector $(3,6)$.
That is, we just multiply each number in the vector by 3.  So if $\alpha$
is a number we say
\begin{equation*}
\alpha
\begin{bmatrix}
x_{1} \\ x_2 \\ \vdots \\ x_n
\end{bmatrix} =
\begin{bmatrix}
\alpha x_{1} \\ \alpha x_2 \\ \vdots \\ \alpha x_n
\end{bmatrix} .
\end{equation*}
When the scalar is negative, then when we multiply a vector by it, the
vector is not only scaled, but it also changes orientation.  So for example
multiplying $(1,2)$ by $-3$ means we should go 3 times further but in the
opposite direction, so 3 units to the left and 6 units down, or in other
words, $(-3,-6)$.

Let's compute a small example:
\begin{equation*}
3
\begin{bmatrix}
1 \\ 2
\end{bmatrix}
+
2
\begin{bmatrix}
-4 \\ -1
\end{bmatrix} 
-
3
\begin{bmatrix}
-2 \\ 2
\end{bmatrix} 
=
\begin{bmatrix}
3(1)+2(-4)-3(-2) \\ 3(2)+2(-1)-3(2)
\end{bmatrix}
=
\begin{bmatrix}
1 \\ -2
\end{bmatrix}
.
\end{equation*}

\subsection{Linear mappings and matrices}

We begin with vector-valued functions.  That is a vector valued function
$F$ is a rule that takes a vector $\vec{x}$ and assigns it another vector
$\vec{y}$.  For example, $F$ could be a scaling that doubles the size of
vectors.  So
\begin{equation*}
F(\vec{x}) = 2 \vec{x} ,
\end{equation*}
or
\begin{equation*}
F
\left( \begin{bmatrix} 1 \\ 3 \end{bmatrix} \right)
=
2
\begin{bmatrix} 1 \\ 3 \end{bmatrix}
=
\begin{bmatrix} 2 \\ 6 \end{bmatrix} .
\end{equation*}
If $F$ is a mapping that takes vectors in
${\mathbb R}^2$ to 
${\mathbb R}^2$ (such as the above), we write
\begin{equation*}
F \colon {\mathbb R}^2 \to {\mathbb R}^2 .
\end{equation*}
The words \emph{function} and \emph{mapping} are used rather interchangeably,
although more often than not, \emph{mapping} is used when talking about a
vector-valued function, and the word \emph{function} is often used when the
function is scalar-valued.

Any beginning student of mathematics (and many a seasoned mathematician),
when they see an expression such as
\begin{equation*}
f(3x+8y)
\end{equation*}
yearns to write
\begin{equation*}
3f(x)+8f(y) .
\end{equation*}
After all, who hasn't wanted to write $\sqrt{x+y} = \sqrt{x} + \sqrt{y}$ or
something like that at some point in their mathematical lives.
Wouldn't life be simple if we could do that?
Of course we can't always do that (for example, not with the square roots!)
It turns out there are many functions where
we can do exactly the above.  Such functions are called \emph{linear}.

More explicitly a mapping $F \colon {\mathbb R}^n \to {\mathbb R}^m$
is called \emph{linear}\index{linear mapping} if
\begin{equation*}
F(\vec{x}+\vec{y}) = F(\vec{x})+F(\vec{y})
\end{equation*}
and also
\begin{equation*}
F(\alpha \vec{x}) = \alpha F(\vec{x}) .
\end{equation*}
The example $F$ above that doubles the size of all vectors is linear.  Let
us check
\begin{equation*}
F(\vec{x}+\vec{y})
=
2(\vec{x}+\vec{y})
=
2\vec{x}+2\vec{y}
=
F(\vec{x})+F(\vec{y})
\end{equation*}
and also
\begin{equation*}
F(\alpha \vec{x}) = 2 \alpha \vec{x} = \alpha 2 \vec{x} = \alpha F(\vec{x}) .
\end{equation*}

When a mapping is linear we often do not write the parentheses.  So we write
just
\begin{equation*}
F \vec{x}
\end{equation*}
instead of $F(\vec{x})$.  That is because the linearity means that $F$
behaves sort of like a multiplication by something.
We also call a linear function a
\emph{linear transformation}\index{transformation}.
If you want to be really fancy and impress your friends, you can call it a
\emph{linear operator}\index{operator}.

There are lots of linear mappings, a lot more complicated than the scaling
example above.  They really are a multiplication by \myquote{something.}
That something is a matrix.

A \emph{\myindex{matrix}}
is an $m
\times n$ array of numbers ($m$ rows and $n$ columns).  For example, we denote
a $3 \times 5$ matrix as follows
\begin{equation*}
A = 
\begin{bmatrix}
a_{11} & a_{12} & a_{13} & a_{14} & a_{15} \\
a_{21} & a_{22} & a_{23} & a_{24} & a_{25} \\
a_{31} & a_{32} & a_{33} & a_{34} & a_{35}
\end{bmatrix} .
\end{equation*}
The numbers $a_{ij}$ are called \emph{elements}\index{element of a matrix}
or \emph{entries}\index{entry of a matrix}.

Notice that a column vector is simply an $m \times 1$ matrix.  Similarly to
a column vector there is also a 
\emph{\myindex{row vector}}, which would be a $1 \times n$ matrix.
If we have an $n \times n$ matrix, then we will say that it is a
\emph{\myindex{square matrix}}.

Now how does a matrix $A$ relate to a linear mapping?
Well a matrix tells you where
certain vectors go.  Let's give a name to certain special vectors.
The \emph{\myindex{standard basis vectors}} of ${\mathbb R}^n$ are vectors
\begin{equation*}
\vec{e}_1 =
\begin{bmatrix}
1 \\ 0 \\ 0 \\ \vdots \\ 0
\end{bmatrix} ,
\qquad
\vec{e}_2 =
\begin{bmatrix}
0 \\ 1 \\ 0 \\ \vdots \\ 0
\end{bmatrix} ,
\qquad
\vec{e}_3 =
\begin{bmatrix}
0 \\ 0 \\ 1 \\ \vdots \\ 0
\end{bmatrix} ,
\qquad
\cdots ,
\qquad
\vec{e}_n =
\begin{bmatrix}
0 \\ 0 \\ 0 \\ \vdots \\ 1
\end{bmatrix} .
\end{equation*}
For example, in ${\mathbb R}^3$ these vectors are
\begin{equation*}
\vec{e}_1 =
\begin{bmatrix}
1 \\ 0 \\ 0
\end{bmatrix} ,
\qquad
\vec{e}_2 =
\begin{bmatrix}
0 \\ 1 \\ 0
\end{bmatrix} ,
\qquad
\vec{e}_3 =
\begin{bmatrix}
0 \\ 0 \\ 1
\end{bmatrix} .
\end{equation*}
You may recall from calculus of several variables that these are
sometimes called $\vec{\imath}$, $\vec{\jmath}$, $\vec{k}$.

The reason these are called a \emph{\myindex{basis}} is that every other
vector can be written as a \emph{\myindex{linear combination}} of them.
For example, in ${\mathbb R}^3$ the vector $(4,5,6)$ can be written as
\begin{equation*}
4 \vec{e}_1 + 
5 \vec{e}_2 + 
6 \vec{e}_3
=
4
\begin{bmatrix}
1 \\ 0 \\ 0
\end{bmatrix}
+
5
\begin{bmatrix}
0 \\ 1 \\ 0
\end{bmatrix}
+
6
\begin{bmatrix}
0 \\ 0 \\ 1
\end{bmatrix}
=
\begin{bmatrix}
4 \\ 5 \\ 6
\end{bmatrix} .
\end{equation*}

So how does a matrix represent a linear mapping?
Well, the columns of the matrix are the vectors where $A$ as a linear
mapping takes $\vec{e}_1$, $\vec{e}_2$, etc.
For example, 
\begin{equation*}
M = 
\begin{bmatrix}
1 & 2 \\ 3 & 4
\end{bmatrix} .
\end{equation*}
Then as a linear mapping $M \colon {\mathbb R}^2 \to {\mathbb R}^2$ takes
$\vec{e}_1 = \left[ \begin{smallmatrix} 1 \\ 0 \end{smallmatrix} \right]$ to
$\left[ \begin{smallmatrix} 1 \\ 3 \end{smallmatrix} \right]$
and
$\vec{e}_2 = \left[ \begin{smallmatrix} 0 \\ 1 \end{smallmatrix} \right]$ to
$\left[ \begin{smallmatrix} 2 \\ 4 \end{smallmatrix} \right]$.  In other
words
\begin{equation*}
M \vec{e}_1 =
\begin{bmatrix}
1 & 2 \\ 3 & 4
\end{bmatrix}
\begin{bmatrix}
1 \\ 0
\end{bmatrix}
=
\begin{bmatrix}
1 \\ 3
\end{bmatrix},
\qquad
\text{and}
\qquad
M \vec{e}_2 =
\begin{bmatrix}
1 & 2 \\ 3 & 4
\end{bmatrix}
\begin{bmatrix}
0 \\ 1
\end{bmatrix}
=
\begin{bmatrix}
2 \\ 4
\end{bmatrix}.
\end{equation*}

More generally, if we have an $n \times m$ matrix $A$, that is we have $n$ rows
and $m$ columns, then the mapping $A \colon {\mathbb R}^m \to {\mathbb R}^n$
and it takes $\vec{e}_j$ to the $j^{\text{th}}$ column of $A$.
For example,
\begin{equation*}
A = 
\begin{bmatrix}
a_{11} & a_{12} & a_{13} & a_{14} & a_{15} \\
a_{21} & a_{22} & a_{23} & a_{24} & a_{25} \\
a_{31} & a_{32} & a_{33} & a_{34} & a_{35}
\end{bmatrix}
\end{equation*}
represents a mapping from ${\mathbb R}^5$ to ${\mathbb R}^3$ that does
\begin{equation*}
A \vec{e}_1 =
\begin{bmatrix}
a_{11} \\ a_{21} \\ a_{31}
\end{bmatrix} ,
\qquad
A \vec{e}_2 =
\begin{bmatrix}
a_{12} \\ a_{22} \\ a_{32}
\end{bmatrix} ,
\qquad
A \vec{e}_3 =
\begin{bmatrix}
a_{13} \\ a_{23} \\ a_{33}
\end{bmatrix} ,
\qquad
A \vec{e}_4 =
\begin{bmatrix}
a_{14} \\ a_{24} \\ a_{34}
\end{bmatrix} ,
\qquad
A \vec{e}_5 =
\begin{bmatrix}
a_{15} \\ a_{25} \\ a_{35}
\end{bmatrix} .
\end{equation*}

But what if I have another vector $\vec{x}$?  Where does it go?  Well we use
linearity.  First write the vector as a linear combination of the standard
basis vectors:
\begin{equation*}
\vec{x} =
\begin{bmatrix}
x_1 \\ x_2 \\ x_3 \\ x_4 \\ x_5
\end{bmatrix}
=
x_1
\begin{bmatrix}
1 \\ 0 \\ 0 \\ 0 \\ 0
\end{bmatrix}
+
x_2
\begin{bmatrix}
0 \\ 1 \\ 0 \\ 0 \\ 0
\end{bmatrix}
+
x_3
\begin{bmatrix}
0 \\ 0 \\ 1 \\ 0 \\ 0
\end{bmatrix}
+
x_4
\begin{bmatrix}
0 \\ 0 \\ 0 \\ 1 \\ 0
\end{bmatrix}
+
x_5
\begin{bmatrix}
0 \\ 0 \\ 0 \\ 0 \\ 1
\end{bmatrix}
=
x_1 \vec{e}_1 + 
x_2 \vec{e}_2 + 
x_3 \vec{e}_3 + 
x_4 \vec{e}_4 + 
x_5 \vec{e}_5 .
\end{equation*}
Then
\begin{equation*}
A \vec{x}
=
A ( 
x_1 \vec{e}_1 + 
x_2 \vec{e}_2 + 
x_3 \vec{e}_3 + 
x_4 \vec{e}_4 + 
x_5 \vec{e}_5 
)
=
x_1 A\vec{e}_1 + 
x_2 A\vec{e}_2 + 
x_3 A\vec{e}_3 + 
x_4 A\vec{e}_4 + 
x_5 A\vec{e}_5 .
\end{equation*}
So if I know where $A$ takes all the basis vectors, I know where it takes
all vectors.

As an example, suppose $M$ is the $2 \times 2$ matrix from above, and
suppose we wish to find
\begin{equation*}
M
\begin{bmatrix}
-2 \\ 0.1
\end{bmatrix}
=
\begin{bmatrix}
1 & 2 \\
3 & 4
\end{bmatrix}
\begin{bmatrix}
-2 \\ 0.1
\end{bmatrix}
=
-2
\begin{bmatrix}
1 \\ 3
\end{bmatrix}
+
0.1
\begin{bmatrix}
2 \\ 4
\end{bmatrix}
=
\begin{bmatrix}
-1.8 \\ -5.6
\end{bmatrix} .
\end{equation*}

Every linear mapping from ${\mathbb R}^m$ to ${\mathbb R}^n$ can be
represented by an $n \times m$ matrix.  You just figure out where does it
take the standard basis vectors.  Conversely, every $n \times m$ matrix
represents a linear mapping.  Hence, we can just think of matrices being
linear mappings, and linear mappings being matrices.

Or can we?  Well in this book we study mostly linear differential operators,
and linear differential operators are linear mappings, just acting on
an infinite dimensional space of functions:
\begin{equation*}
L f = g
\end{equation*}
for a function $f$ we get a function $g$, and it is linear in the sense that
\begin{equation*}
L (\alpha f + \beta h) = \alpha Lf + \beta Lh
\end{equation*}
for any numbers (scalars) $\alpha$ and $\beta$.

So the answer is not really.  But if we consider vectors in finite
dimensional spaces ${\mathbb R}^n$ then yes, every linear mapping is a
matrix.
And as we have mentioned at the beginning of this section, that we can
\myquote{make everything a vector}, well that's not strictly true, but
approximately.  Those \myquote{infinite dimensional} spaces of functions can
be approximated by a finite dimensional space, and then linear operators
are just matrices.  So approximately, this is true.  And as far as actual
computations that we can do on a computer, we have to work only with
finitely many dimensions anyway.  If you ask a computer to plot a function,
what it does is it samples the function at finitely many points and then
connects the dots\footnote{If you have every used Matlab, you may have
noticed that to plot a function, we take a vector of inputs, ask Matlab
to compute the corresponding vector of values of the function, and then we ask
it to plot the result.}.
It doesn't actually give you infinitely many values.
So the way that you have been using the computer so far has in fact
already been a certain approximation of the space of functions by a finite
dimensional space.

\medskip

To end the section, we notice how $A \vec{x}$ can be written more succintly.
Suppose
\begin{equation*}
A = 
\begin{bmatrix}
a_{11} & a_{12} & a_{13} \\
a_{21} & a_{22} & a_{23}
\end{bmatrix}
\qquad \text{and} \qquad
\vec{x} = 
\begin{bmatrix}
x_1 \\ x_2 \\ x_3 
\end{bmatrix} .
\end{equation*}
Then
\begin{equation*}
A \vec{x} = 
\begin{bmatrix}
a_{11} & a_{12} & a_{13} \\
a_{21} & a_{22} & a_{23}
\end{bmatrix}
\begin{bmatrix}
x_1 \\ x_2 \\ x_3 
\end{bmatrix} 
=
\begin{bmatrix}
a_{11} x_1 + a_{12} x_2 + a_{13} x_3 \\
a_{21} x_1 + a_{22} x_2 + a_{23} x_3
\end{bmatrix}  .
\end{equation*}
For example,
\begin{equation*}
\begin{bmatrix}
1 & 2 \\ 3 & 4
\end{bmatrix}
\begin{bmatrix}
2 \\ -1
\end{bmatrix} 
=
\begin{bmatrix}
1 \cdot 2 + 2 \cdot (-1) \\
3 \cdot 2 + 4 \cdot (-1)
\end{bmatrix}
=
\begin{bmatrix}
0 \\ 2
\end{bmatrix}  .
\end{equation*}

In other words, you take a row of the matrix, you multiply them by the
entries in your vector, you add things up, and that's the corresponding
entry in the resulting vector.


\subsection{Exercises}

FIXME



%%%%%%%%%%%%%%%%%%%%%%%%%%%%%%%%%%%%%%%%%%%%%%%%%%%%%%%%%%%%%%%%%%%%%%%%%%%%%%

\sectionnewpage
\section{Matrix algebra}
\label{matalg:section}

\sectionnotes{? lectures}

\subsection{One-by-one matrices}

Let us motivate what we want to achieve with matrices.
Real valued linear mappings of the real line, that is linear functions
which eat numbers and spit out numbers, are just multiplications by a
number.  Consider a mapping defined by multiplying by a
number, call that $\alpha$, it takes $x$ to $\alpha x$.  What we can do is
to \emph{add} such mappings and we can \emph{multiply} them.
If we have another mapping $\beta$, then
\begin{equation*}
\alpha x + \beta x = (\alpha + \beta) x .
\end{equation*}
So we get a new mapping $\alpha+\beta$ that multiplies $x$ by, well,
$\alpha+\beta$.  If $D$ is a mapping that doubles things, 
$Dx = 2x$, and $T$ is a mapping that triples, $Tx = 3x$, then
$D+T$ is a mapping that multiplies by $5$, $(D+T)x = 5x$.

Similarly we can \emph{compose} such mappings, that
is, we could apply one and then the other.  We take $x$, we run it through
the first mapping $\alpha$ to get $\alpha$ times $x$, then we run
$\alpha x$ through the second mapping $\beta$.  In other words,
\begin{equation*}
\beta ( \alpha x ) = (\beta \alpha) x .
\end{equation*}
So we just multiply those two numbers.  For example, using our doubling
and tripling mappings, if we double and then triple, that is $T(Dx)$ then
we obtain $6x$.  Then the composition $TD$ is the mapping that multiplies
by $6$.  For larger matrices, composition also ends up being a kind of
multiplication, and that is why we write it as $TD$.

\subsection{Matrix addition and scalar multiplication}

The mappings that multiply by numbers are just $1 \times 1$ matrices.  The
number $\alpha$ above could be written as a matrix $[\alpha]$.
Well, so perhaps we would want to do the same things to all matrices that we
did to those $1 \times 1$ matrices above.  First, let us add matrices.
If we have a matrix $A$ and a matrix $B$ that are of the same size,
say $m \times n$, then they are mappings from
${\mathbb{R}}^n$ to ${\mathbb{R}}^m$.  The mapping $A+B$ should also be a mapping from
${\mathbb{R}}^n$ to ${\mathbb{R}}^m$, and it should do the following to
vectors:
\begin{equation*}
(A+B) \vec{x} = A\vec{x} + B \vec{x} .
\end{equation*}
It turns out you just add the matrices element-wise:  If the
$ij^{\text{th}}$ entry of $A$ is $a_{ij}$, and the
$ij^{\text{th}}$ entry of $B$ is $b_{ij}$, then the
$ij^{\text{th}}$ entry of $A+B$ is $a_{ij} + b_{ij}$.  If
\begin{equation*}
A = 
\begin{bmatrix}
a_{11} & a_{12} & a_{13}  \\
a_{21} & a_{22} & a_{23}
\end{bmatrix}
\qquad \text{and} \qquad
B = 
\begin{bmatrix}
b_{11} & b_{12} & b_{13}  \\
b_{21} & b_{22} & b_{23}
\end{bmatrix} ,
\end{equation*}
then
\begin{equation*}
A+B = 
\begin{bmatrix}
a_{11} + b_{11} & a_{12} + b_{12} & a_{13} + b_{13}  \\
a_{21} + b_{21} & a_{22} + b_{22} & a_{23} + b_{23}
\end{bmatrix} .
\end{equation*}
Let us illustrate on a more concrete example:
\begin{equation*}
\begin{bmatrix}
1 & 2 \\
3 & 4 \\
5 & 6
\end{bmatrix}
+
\begin{bmatrix}
7 & 8 \\
9 & 10 \\
11 & -1
\end{bmatrix}
=
\begin{bmatrix}
1+7 & 2+8 \\
3+9 & 4+10 \\
5+11 & 6-1
\end{bmatrix}
=
\begin{bmatrix}
8 & 10 \\
12 & 14 \\
16 & 5
\end{bmatrix} .
\end{equation*}
Let's check that this does the right thing to a vector.  Let's use some
of the vector algebra that we already know, and
regroup things:
\begin{equation*}
\begin{split}
\begin{bmatrix}
1 & 2 \\
3 & 4 \\
5 & 6
\end{bmatrix}
\begin{bmatrix}
2 \\ -1
\end{bmatrix}
+
\begin{bmatrix}
7 & 8 \\
9 & 10 \\
11 & -1
\end{bmatrix}
\begin{bmatrix}
2 \\ -1
\end{bmatrix}
& =
\left(
2
\begin{bmatrix}
1 \\
3 \\
5
\end{bmatrix}
-
\begin{bmatrix}
2 \\
4 \\
6
\end{bmatrix}
\right)
+
\left(
2
\begin{bmatrix}
7 \\
9 \\
11
\end{bmatrix}
-
\begin{bmatrix}
8 \\
10 \\
-1
\end{bmatrix}
\right)
\\
& = 
2
\left(
\begin{bmatrix}
1 \\
3 \\
5
\end{bmatrix}
+
\begin{bmatrix}
7 \\
9 \\
11
\end{bmatrix}
\right)
-
\left(
\begin{bmatrix}
2 \\
4 \\
6
\end{bmatrix}
+
\begin{bmatrix}
8 \\
10 \\
-1
\end{bmatrix}
\right)
\\
& = 
2
\begin{bmatrix}
1+7 \\
3+9 \\
5+11
\end{bmatrix}
-
\begin{bmatrix}
2+8 \\
4+10 \\
6-1
\end{bmatrix}
=
2
\begin{bmatrix}
8 \\
12 \\
16
\end{bmatrix}
-
\begin{bmatrix}
10 \\
14 \\
5
\end{bmatrix}
\\
& =
\begin{bmatrix}
8 & 10 \\
12 & 14 \\
16 & 5
\end{bmatrix}
\begin{bmatrix}
2 \\
-1
\end{bmatrix} 
\quad
\left(
=
\begin{bmatrix}
2(8)- 10 \\
2(12) - 14 \\
2(16) - 5
\end{bmatrix}
=
\begin{bmatrix}
6 \\
10 \\
27
\end{bmatrix}
\right) .
\end{split}
\end{equation*}
If we replaced the numbers by letters that would constitute a proof!
You'll notice that we didn't really have to even compute what the
result is to convince ourselves that the two expressions were equal.

If the sizes of the matrices do not match, then addition is not defined.
So if $A$ is $3 \times 2$ and $B$ is $2 \times 5$, then we just cannot add
these matrices.  We don't know what that could possibly mean.

\medskip

It is also useful to have a matrix that if we add it to any other matrix
nothing happens.  This is the zero matrix, the matrix of all zeros:
\begin{equation*}
\begin{bmatrix}
1 & 2 \\
3 & 4
\end{bmatrix}
+
\begin{bmatrix}
0 & 0 \\
0 & 0
\end{bmatrix}
=
\begin{bmatrix}
1 & 2 \\
3 & 4
\end{bmatrix} .
\end{equation*}
We often denote the zero matrix
by $0$ without specifying size.  We would then just write $A + 0$, where we
just assume that $0$ is the matrix of the same size as $A$.

\medskip

There are really two things we can multiply matrices by.  We could multiply
matrices by scalars or we could multiply by other matrices.  Let's first
consider multiplication by scalars.
For a matrix $A$ and a scalar $\alpha$ we want $\alpha A$ to be the matrix
that accomplishes
\begin{equation*}
(\alpha A) \vec{x} = \alpha (A \vec{x}) .
\end{equation*}
That is just scaling the result by $\alpha$.  If you think about it,
scaling every term in $A$ by $\alpha$ will accomplish just that:
If
\begin{equation*}
A = 
\begin{bmatrix}
a_{11} & a_{12} & a_{13}  \\
a_{21} & a_{22} & a_{23}
\end{bmatrix},
\qquad\text{then} \qquad
\alpha A = 
\begin{bmatrix}
\alpha a_{11} & \alpha a_{12} & \alpha a_{13}  \\
\alpha a_{21} & \alpha a_{22} & \alpha a_{23}
\end{bmatrix} .
\end{equation*}
For example,
\begin{equation*}
2
\begin{bmatrix}
1 & 2 & 3 \\
4 & 5 & 6
\end{bmatrix} =
\begin{bmatrix}
2 & 4 & 6 \\
8 & 10 & 12
\end{bmatrix} .
\end{equation*}

Let us list some properties of matrix addition and scalar multiplication.
Denote by $0$ the zero matrix, by
$\alpha$, $\beta$ scalars, and by $A$, $B$, $C$ matrices.  Then:
\begin{align*}
A + 0 & = A = 0 + A , \\
A + B & = B + A , \\
(A + B) + C & = A + (B + C) , \\
\alpha(A+B) & = \alpha A+\alpha B, \\
(\alpha+\beta)A & = \alpha A + \beta A.
\end{align*}
These rules should look very familiar.

\subsection{Matrix multiplication}

As we mentioned above, composition of linear mappings is also a
multiplication of matrices.  Suppose $A$ is an $m \times n$ matrix,
that is it takes
${\mathbb R}^n$ to
${\mathbb R}^m$,
and $B$ is an $n \times p$ matrix, that is it takes
${\mathbb R}^p$ to
${\mathbb R}^n$.  The composition $AB$ should work as follows
\begin{equation*}
AB\vec{x} = A(B\vec{x}) .
\end{equation*}
That is, a vector $\vec{x}$ in ${\mathbb R}^p$ gets taken to 
the vector $B\vec{x}$ in
${\mathbb R}^n$.  Then the mapping $A$ takes it to the vector $A(B\vec{x})$
in ${\mathbb R}^m$.  In other words, the composition $AB$ should be an $m
\times p$ matrix.  So in terms of sizes we should have
\begin{equation*}
\text{``} \quad
[ m \times n ]
\,
[ n \times p ]
=
[ m \times p ] . \quad \text{''}
\end{equation*}
Notice how the middle size must match.

OK, so now we know what sizes of matrices we should be able to multiply.
Let us now define matrix multiplication.  We start with the so-called
\emph{\myindex{dot product}} (or \emph{\myindex{inner product}}) of two vectors.
Usually this will be a row vector multiplied
with a column vector of the same size.  For the dot product we multiply
each pair of entries from the first and the second vector and we sum these
products.  The result is a single number.
For example,
\begin{equation*}
\begin{bmatrix}
a_1 & a_2 & a_3
\end{bmatrix}
\cdot
\begin{bmatrix}
b_1 \\
b_2 \\
b_3
\end{bmatrix}
= a_1 b_1 + a_2 b_2 + a_3 b_3 .
\end{equation*}
And similarly for larger (or smaller) vectors.
A dot product is really a product of two matrices: a $1 \times n$ matrix
and an $n \times 1$ matrix resulting in a $1 \times 1$ matrix, that is, a
number.

Armed with the dot product we define the
\emph{\myindex{product of matrices}}\index{matrix product}.
First let us denote by $\operatorname{row}_i(A)$ the $i^{\text{th}}$ row
of $A$ and by
$\operatorname{column}_j(A)$ the $j^{\text{th}}$ column of $A$.
For an $m \times n$ matrix $A$ and an $n \times p$ matrix $B$
we can define the product $AB$.  We let $AB$ be an $m \times p$
matrix whose $ij^{\text{th}}$ entry is the dot product
\begin{equation*}
\operatorname{row}_i(A) \cdot
\operatorname{column}_j(B) .
\end{equation*}
For example, given a $2 \times 3$ and a $3 \times 2$ matrix
we should end up with a $2 \times 2$ matrix:
\begin{equation} \label{linalg:eqmatrixmulex}
\begin{bmatrix}
a_{11} & a_{12} & a_{13} \\
a_{21} & a_{22} & a_{23}
\end{bmatrix}
\begin{bmatrix}
b_{11} & b_{12} \\
b_{21} & b_{22} \\
b_{31} & b_{32}
\end{bmatrix}
=
\begin{bmatrix}
a_{11} b_{11} + 
a_{12} b_{21} + 
a_{13} b_{31} & &
a_{11} b_{12} + 
a_{12} b_{22} + 
a_{13} b_{32} \\
a_{21} b_{11} + 
a_{22} b_{21} + 
a_{23} b_{31} & &
a_{21} b_{12} + 
a_{22} b_{22} + 
a_{23} b_{32}
\end{bmatrix} ,
\end{equation}
or with some numbers:
\begin{equation*}
\begin{bmatrix}
1 & 2 & 3 \\
4 & 5 & 6
\end{bmatrix}
\begin{bmatrix}
-1 & 2 \\
-7 & 0 \\
1 & -1
\end{bmatrix}
=
\begin{bmatrix}
1\cdot (-1) + 2\cdot (-7) + 3 \cdot 1 &  &
1\cdot 2 + 2\cdot 0 + 3 \cdot (-1) \\
4\cdot (-1) + 5\cdot (-7) + 6 \cdot 1 &  &
4\cdot 2 + 5\cdot 0 + 6 \cdot (-1)
\end{bmatrix}
=
\begin{bmatrix}
-12 & -1 \\
-33 & 2
\end{bmatrix} .
\end{equation*}

An useful consequence of the definition is that $A \vec{x}$ for a matrix
and a vector is also matrix multiplication.  That is really why we think of
vectors as column vectors, or $n \times 1$ matrices.
For example,
\begin{equation*}
\begin{bmatrix}
1 & 2 \\ 3 & 4
\end{bmatrix}
\begin{bmatrix}
2 \\ -1
\end{bmatrix} 
=
\begin{bmatrix}
1 \cdot 2 + 2 \cdot (-1) \\
3 \cdot 2 + 4 \cdot (-1)
\end{bmatrix}
=
\begin{bmatrix}
0 \\ 2
\end{bmatrix}  .
\end{equation*}
If you look at the last section, that is precisely the last
example we gave.

You should stare at the definition of multiplication and
the previous definition of $A\vec{x}$ as a mapping
for a moment.
Essentially what we are doing with matrix multiplication is applying
the mapping $A$ to the columns of $B$.  This is usually written as follows.
Suppose we write the $n \times p$ matrix
$B = [ \vec{b}_1 ~ \vec{b}_2 ~ \cdots ~ \vec{b}_p ]$, where
$\vec{b}_1, \vec{b}_2, \ldots, \vec{b}_p$ are the columns of $B$.  Then
with a $m \times n$ matrix $A$ we have
\begin{equation*}
AB = 
A [ \vec{b}_1 ~ \vec{b}_2 ~ \cdots ~ \vec{b}_p ]
=
[ A\vec{b}_1 ~ A\vec{b}_2 ~ \cdots ~ A\vec{b}_p ] .
\end{equation*}
In other words, the columns of the $m \times p$ matrix $AB$
are precisely the
vectors $A\vec{b}_1, A\vec{b}_2, \ldots, A\vec{b}_p$.
For example, in \eqref{linalg:eqmatrixmulex},
the columns of 
\begin{equation*}
\begin{bmatrix}
a_{11} & a_{12} & a_{13} \\
a_{21} & a_{22} & a_{23}
\end{bmatrix}
\begin{bmatrix}
b_{11} & b_{12} \\
b_{21} & b_{22} \\
b_{31} & b_{32}
\end{bmatrix}
\end{equation*}
are
\begin{equation*}
\begin{bmatrix}
a_{11} & a_{12} & a_{13} \\
a_{21} & a_{22} & a_{23}
\end{bmatrix}
\begin{bmatrix}
b_{11} \\
b_{21} \\
b_{31}
\end{bmatrix}
\qquad
\text{and}
\qquad
\begin{bmatrix}
a_{11} & a_{12} & a_{13} \\
a_{21} & a_{22} & a_{23}
\end{bmatrix}
\begin{bmatrix}
b_{12} \\
b_{22} \\
b_{32}
\end{bmatrix} .
\end{equation*}
This is a very useful way to understand what matrix multiplication is.
It should also make it easier to remember how to perform matrix multiplication.

\subsection{Some rules of matrix algebra}

For multiplication we want an analogue of a 1.  That is,
we desire a matrix which just leaves everything as it found it.
This analogue is the
so-called \emph{\myindex{identity matrix}}.
The identity matrix is a square matrix with 1s on the
main diagonal and zeros everywhere else.  It is usually denoted by $I$.
For each size we have a different identity matrix and so sometimes we may denote
the size as a subscript.  For example, the $I_3$ would be the $3 \times 3$
identity matrix
\begin{equation*}
I = I_3 =
\begin{bmatrix}
1 & 0 & 0 \\
0 & 1 & 0 \\
0 & 0 & 1
\end{bmatrix} .
\end{equation*}
Let us see how the matrix works on a smaller example,
\begin{equation*}
\begin{bmatrix}
a_{11} & a_{12} \\
a_{21} & a_{22} 
\end{bmatrix}
\begin{bmatrix}
1 & 0 \\
0 & 1
\end{bmatrix} =
\begin{bmatrix}
a_{11} \cdot 1 + a_{12} \cdot 0
& &
a_{11} \cdot 0 + a_{12} \cdot 1
\\
a_{21} \cdot 1 + a_{22} \cdot 0
& &
a_{21} \cdot 0 + a_{22} \cdot 1
\end{bmatrix}
=
\begin{bmatrix}
a_{11} & a_{12} \\
a_{21} & a_{22} 
\end{bmatrix} .
\end{equation*}
Multiplication by the identity from the left looks similar, and also
does not touch anything.

\medskip

We have the following rules for matrix multiplication.  Suppose that
$A$, $B$, $C$ are matrices of the correct sizes so that the following
make sense.  Let $\alpha$ denote a scalar (number).
\begin{align*}
A(BC) & = (AB)C, \\
A(B+C) & = AB + AC, \\
(B+C)A & = BA + CA, \\
\alpha(AB) & = (\alpha A)B = A(\alpha B), \\
IA & = A = AI .
\end{align*}

FIXME: examples?

There is one rule you have seen in primary school
that is quite conspicuously missing.  That is, matrix multiplication is
not commutative.  Firstly, just because $AB$ makes sense, it may be
that $BA$ is not even defined.  For example, if $A$ is $2 \times 3$, and
$B$ is $3 \times 4$, the we can multiply $AB$ but not $BA$.

Even if $AB$ and $BA$ are both defined, does not mean that they are equal.
For example, take
$A = \left[ \begin{smallmatrix} 1 & 1 \\ 1 & 1 \end{smallmatrix} \right]$
and
$B = \left[ \begin{smallmatrix} 1 & 0 \\ 0 & 2 \end{smallmatrix} \right]$:
\begin{equation*}
AB = 
\begin{bmatrix} 1 & 1 \\ 1 & 1 \end{bmatrix}
\begin{bmatrix} 1 & 0 \\ 0 & 2 \end{bmatrix}
=
\begin{bmatrix} 1 & 2 \\ 1 & 2 \end{bmatrix}
\qquad
\not=
\qquad
\begin{bmatrix} 1 & 1 \\ 2 & 2 \end{bmatrix}
=
\begin{bmatrix} 1 & 0 \\ 0 & 2 \end{bmatrix}
\begin{bmatrix} 1 & 1 \\ 1 & 1 \end{bmatrix}
=
BA .
\end{equation*}

\subsection{Inverse}

There are a few other rules you know from primary school that do not quite
generalize easily to matrices:
\begin{enumerate}[(i)]
\item $AB = AC$ does not necessarily imply $B=C$, even if $A$ is not 0.
\item $AB = 0$ does not necessarily mean that $A=0$ or $B=0$.
\end{enumerate}
For example:
\begin{equation*}
\begin{bmatrix} 0 & 1 \\ 0 & 0 \end{bmatrix}
\begin{bmatrix} 0 & 1 \\ 0 & 0 \end{bmatrix}
=
\begin{bmatrix} 0 & 0 \\ 0 & 0 \end{bmatrix}
=
\begin{bmatrix} 0 & 1 \\ 0 & 0 \end{bmatrix}
\begin{bmatrix} 0 & 2 \\ 0 & 0 \end{bmatrix} .
\end{equation*}

To make these rules hold, we do not just need one of the matrices to not be zero,
we would need to \myquote{divide} by
a matrix.  This is where the \emph{\myindex{matrix inverse}} comes in.
Suppose that $A$ and $B$ are $n \times n$ matrices such that
\begin{equation*}
AB = I = BA .
\end{equation*}
Then we call $B$ the inverse of $A$ and we denote $B$ by $A^{-1}$.
Perhaps not surprisingly, ${(A^{-1})}^{-1} = A$, since if the inverse of $A$ 
is $B$, then the inverse of $B$ is $A$.

If the inverse of $A$ exists, then we call $A$
\emph{invertible\index{invertible matrix}}.
If $A$ is not invertible we sometimes say $A$ is
\emph{singular\index{singular matrix}}.

The proper formulation of the cancellation rule is:
\emph{If $A$ is invertible,
then
$AB = AC$ implies $B=C$.}
Similarly:
\emph{If $A$ is invertible,
then
$BA = CA$ implies $B=C$.}
The computation is what you would do in regular algebra with numbers,
but you have to
be careful never to commute matrices:
\begin{align*}
AB & = AC \\
A^{-1}AB & = A^{-1}AC \\
IB & = IC \\
B & = C
\end{align*}
And similarly for cancellation on the right.

Notice that the rule says, among other things, that the
inverse of a matrix is unique if it exists:  If $AB = I = AC$ then $A$ is
invertible and $B=C$.

We will see later how to compute an inverse of a matrix
in general.  For now,
let us note that there is a simple formula for the inverse of
a $2 \times 2$ matrix
\begin{equation*}
\begin{bmatrix}
a & b \\
c & d
\end{bmatrix}^{-1}
=
\frac{1}{ad-bc}
\begin{bmatrix}
d & -b \\
-c & a
\end{bmatrix} .
\end{equation*}

For example:
\begin{equation*}
\begin{bmatrix}
1 & 1 \\
2 & 4
\end{bmatrix}^{-1}
=
\frac{1}{1\cdot 4-1 \cdot 2}
\begin{bmatrix}
4 & -1 \\
-2 & 1
\end{bmatrix}
\begin{bmatrix}
2 & \nicefrac{-1}{2} \\
-1 & \nicefrac{1}{2}
\end{bmatrix} .
\end{equation*}
Let's try it:
\begin{equation*}
\begin{bmatrix}
1 & 1 \\
2 & 4
\end{bmatrix}
\begin{bmatrix}
2 & \nicefrac{-1}{2} \\
-1 & \nicefrac{1}{2}
\end{bmatrix}
=
\begin{bmatrix}
1 & 0 \\
0 & 1
\end{bmatrix}
\qquad
\text{and}
\qquad
\begin{bmatrix}
2 & \nicefrac{-1}{2} \\
-1 & \nicefrac{1}{2}
\end{bmatrix}
\begin{bmatrix}
1 & 1 \\
2 & 4
\end{bmatrix}
=
\begin{bmatrix}
1 & 0 \\
0 & 1
\end{bmatrix} .
\end{equation*}
Of course, just as we cannot divide by every number, not every matrix is
invertible.  In the case of matrices however we may have singular
matrices that are not zero.  For example,
\begin{equation*}
\begin{bmatrix}
1 & 1 \\
2 & 2
\end{bmatrix}
\end{equation*}
is a singular matrix.  But didn't we just give a formula for an inverse?
Let us try it:
\begin{equation*}
\begin{bmatrix}
1 & 1 \\
2 & 2
\end{bmatrix}^{-1}
=
\frac{1}{1\cdot 2-1 \cdot 2}
\begin{bmatrix}
2 & -1 \\
-2 & 1
\end{bmatrix}
=
%mbxSTARTIGNORE
\text{\Huge ?}
%mbxENDIGNORE
%mbx \text{???}
\end{equation*}
We get into a bit of trouble, we are trying to divide by zero.
The matrix is not invertible, it is singular.

So a $2 \times 2$ matrix $A$ is invertible whenever
\begin{equation*}
ad - bc \not= 0
\end{equation*}
and otherwise it is singular.  The expression $ad-bc$ is called
the \emph{determinant} and we will study it in a later section
more carefully.  There is a similar expression for a square
matrix of any size.

\subsection{Diagonal matrices}

A simple (and surprisingly useful) type of a square matrix is a so-called
\emph{\myindex{diagonal matrix}}.  It is a matrix whose entries are all zero
except those on the main diagonal from top left to bottom right.  For
example a $4 \times 4$ diagonal matrix is of the form
\begin{equation*}
\begin{bmatrix}
d_1 & 0 & 0 & 0 \\
0 & d_2 & 0 & 0 \\
0 & 0 & d_3 & 0 \\
0 & 0 & 0 & d_4
\end{bmatrix}
\end{equation*}
Such matrices have nice properties when we multiply by them.  If we multiply
them by a vector, they multiply the $k^{\text{th}}$ entry by $d_k$.  For
example,
\begin{equation*}
\begin{bmatrix}
1 & 0 & 0 \\
0 & 2 & 0 \\
0 & 0 & 3
\end{bmatrix}
\begin{bmatrix}
2 \\ 2 \\ 2
\end{bmatrix}
=
\begin{bmatrix}
2 \\ 4 \\ 6
\end{bmatrix} .
\end{equation*}
Similarly, when they multiply another matrix from the left, they multiply
the $k^{\text{th}}$ row by $d_k$.  For example,
\begin{equation*}
\begin{bmatrix}
2 & 0 & 0 \\
0 & 3 & 0 \\
0 & 0 & -1
\end{bmatrix}
\begin{bmatrix}
1 & 1 & 1 \\
1 & 1 & 1 \\
1 & 1 & 1 
\end{bmatrix}
=
\begin{bmatrix}
2 & 2 & 2 \\
3 & 3 & 3 \\
-1 & -1 & -1 
\end{bmatrix} .
\end{equation*}
On the other hand, multiplying on the right, they multiply the columns:
\begin{equation*}
\begin{bmatrix}
1 & 1 & 1 \\
1 & 1 & 1 \\
1 & 1 & 1 
\end{bmatrix}
\begin{bmatrix}
2 & 0 & 0 \\
0 & 3 & 0 \\
0 & 0 & -1
\end{bmatrix}
=
\begin{bmatrix}
2 & 3 & -1 \\
2 & 3 & -1 \\
2 & 3 & -1 
\end{bmatrix} .
\end{equation*}
And it is really easy to multiply two diagonal matrices together:
\begin{equation*}
\begin{bmatrix}
1 & 0 & 0 \\
0 & 2 & 0 \\
0 & 0 & 3 
\end{bmatrix}
\begin{bmatrix}
2 & 0 & 0 \\
0 & 3 & 0 \\
0 & 0 & -1
\end{bmatrix}
=
\begin{bmatrix}
1 \cdot 2 & 0 & 0 \\
0 & 2 \cdot 3 & 0 \\
0 & 0 & 3 \cdot (-1) 
\end{bmatrix}
=
\begin{bmatrix}
2 & 0 & 0 \\
0 & 6 & 0 \\
0 & 0 & -3 
\end{bmatrix} .
\end{equation*}

For this last reason, they are easy to invert, you simply invert
each diagonal element:
\begin{equation*}
\begin{bmatrix}
d_1 & 0 & 0 \\
0 & d_2 & 0 \\
0 & 0 & d_3 
\end{bmatrix}^{-1}
=
\begin{bmatrix}
d_1^{-1} & 0 & 0 \\
0 & d_2^{-1} & 0 \\
0 & 0 & d_3^{-1} 
\end{bmatrix} .
\end{equation*}
Let us check an example
\begin{equation*}
\underbrace{
\begin{bmatrix}
2 & 0 & 0 \\
0 & 3 & 0 \\
0 & 0 & 4 
\end{bmatrix}^{-1}
}_{A^{-1}}
\underbrace{
\begin{bmatrix}
2 & 0 & 0 \\
0 & 3 & 0 \\
0 & 0 & 4 
\end{bmatrix}
}_{A}
=
\underbrace{
\begin{bmatrix}
\frac{1}{2} & 0 & 0 \\
0 & \frac{1}{3} & 0 \\
0 & 0 & \frac{1}{4} 
\end{bmatrix}
}_{A^{-1}}
\underbrace{
\begin{bmatrix}
2 & 0 & 0 \\
0 & 3 & 0 \\
0 & 0 & 4 
\end{bmatrix}
}_{A}
=
\underbrace{
\begin{bmatrix}
1 & 0 & 0 \\
0 & 1 & 0 \\
0 & 0 & 1 
\end{bmatrix}
}_{I} .
\end{equation*}
It is no wonder that the way we solve many problems in linear algebra
(and in differential equations) is to try to reduce the problem to the
case of diagonal matrices.

\subsection{Transpose}

There is really nothing that says that vectors must always be column
vectors, that is just a convention.
Swapping rows and columns is sometimes useful.  This really swaps 
the order of multiplication, and from time to time it is needed.

The operation that swaps rows and columns is the so-called
\emph{\myindex{transpose}}.
The transpose of $A$ is denoted by $A^T$.  Example:
\begin{equation*}
\begin{bmatrix}
1 & 2 & 3 \\
4 & 5 & 6
\end{bmatrix}^T =
\begin{bmatrix}
1 & 4 \\
2 & 5 \\
3 & 6 
\end{bmatrix}
\end{equation*}
So transpose takes an $m \times n$ matrix to an $n \times m$ matrix.

A key fact about the transpose is that if the product $AB$ makes sense
then $B^TA^T$ also makes sense, at least from the point of view of sizes.
In fact, we get precisely the transpose of $AB$.  That is:
\begin{equation*}
{(AB)}^T = B^TA^T .
\end{equation*}
For example,
\begin{equation*}
{\left(
\begin{bmatrix}
1 & 2 & 3 \\
4 & 5 & 6
\end{bmatrix}
\begin{bmatrix}
0 & 1 \\
1 & 0 \\
2 & -2
\end{bmatrix}
\right)}^T =
\begin{bmatrix}
0 & 1 & 2 \\
1 & 0 & -2
\end{bmatrix}
\begin{bmatrix}
1 & 4 \\
2 & 5 \\
3 & 6 
\end{bmatrix} .
\end{equation*}
It is left to the reader to verify that computing the matrix product on the
left and then transposing is the same as computing the matrix product on the
right.

If we have a column vector $\vec{x}$ which to which we apply a matrix $A$,
then the row vector $\vec{x}^T$ applies to $A^T$ from the left:
\begin{equation*}
{(A\vec{x})}^T = \vec{x}^TA^T .
\end{equation*}

Another place where transpose is useful is when we wish to apply the dot
product\footnote{As a side note, mathematicians
write $\vec{y}^T\vec{x}$ and physicists
write $\vec{x}^T\vec{y}$.  Shhh\ldots don't tell anyone, but the physicists
are probably right on this.}
to two column vectors:
\begin{equation*}
\vec{x} \cdot \vec{y} = \vec{y}^T \vec{x} .
\end{equation*}
That is the way that one often writes the dot product in software.

Finally, we say a matrix $A$ is \emph{symmetric}\index{symmetric matrix}
if $A = A^T$.  For example,
\begin{equation*}
\begin{bmatrix}
1 & 2 & 3 \\
2 & 4 & 5 \\
3 & 5 & 6
\end{bmatrix}
\end{equation*}
is a symmetric matrix.  Notice that a symmetric matrix is always
square, that is, $n \times n$.  Symmetric matrices 
have many nice properties\footnote{Although so far we do not know enough
about matrices to really appreciate them.},
and come up quite often in applications.

\subsection{Exercises}

FIXME:

%%%%%%%%%%%%%%%%%%%%%%%%%%%%%%%%%%%%%%%%%%%%%%%%%%%%%%%%%%%%%%%%%%%%%%%%%%%%%%

\sectionnewpage
\section{Elimination}
\label{elim:section}

\sectionnotes{? lectures}


One application of matrices is to solve systems of
linear equations\footnote{Although perhaps we have this backwards,
quite often we solve a linear system of equations
to find out something about matrices, rather than vice versa.}.
Consider the following system of linear equations
\begin{align*}
          2 x_1 +           2 x_2 +           2 x_3 & = 2 , \\
\phantom{9} x_1 + \phantom{9} x_2 +           3 x_3 & = 5 , \\
\phantom{9} x_1 +           4 x_2 + \phantom{9} x_3 & = 10 .
\end{align*}

Without changing the solution,
we can swap equations in this system,
we can multiply any of the equations by a nonzero number, and
we can add a multiple of one equation to another equation.
These operations always suffice to find a solution.
The procedure is called \emph{\myindex{elimination}}, since
we will be eliminating the variables from the system
one by one until we end with a single equation in one unknown, or
in fact 3 equations each depending only one unknown.

A system of linear equations can be written as a matrix equation:
\begin{equation*}
A \vec{x} = \vec{b} .
\end{equation*}
For the example system above, we find
\begin{equation*}
\underbrace{
\begin{bmatrix}
2 & 2 & 2 \\
1 & 1 & 3 \\
1 & 4 & 1 
\end{bmatrix}
}_{A}
\underbrace{
\begin{bmatrix}
x_1 \\
x_2 \\
x_3
\end{bmatrix} 
}_{\vec{x}}
=
\underbrace{
\begin{bmatrix}
2 \\
5 \\
10
\end{bmatrix}
}_{\vec{b}} .
\end{equation*}
If we know the inverse of $A$, then we would be done.  We could
just solve:
\begin{equation*}
\vec{x} = A^{-1} A \vec{x} = A^{-1} \vec{b} .
\end{equation*}
Well, but that is the problem, we do not know how to compute the
inverse.  We will see later that to compute the inverse we need
to do elimination.

FIXME FIXME FIXME FIXME FIXME


To solve the system we put the coefficient matrix (the matrix on the left
hand side of the equation) together with the vector on the right and side
and get the
so-called
\emph{\myindex{augmented matrix}}
\begin{equation*}
\left[
\begin{array}{ccc|c}
2 & 2 & 2 & 2 \\
1 & 1 & 3 & 5 \\
1 & 4 & 1 & 10
\end{array}
\right] .
%\qquad
%\text{or just}
%\qquad
%\begin{bmatrix}
%2 & 2 & 2 & 2 \\
%1 & 1 & 3 & 5 \\
%1 & 4 & 1 & 10
%\end{bmatrix} .
\end{equation*}
We apply the following three elementary operations.
%\pagebreak[2]%
\begin{enumerate}[(i)]
\item Swap two rows.
\item Multiply a row by a nonzero number.
\item Add a multiple of one row to another row.
\end{enumerate}
We keep doing these operations until we get into a state where it is
easy to read off the answer, or until we get into a contradiction indicating
no solution, for example if we come up with an equation such as $0=1$.

Let us work through the example.  First multiply the first row by
$\nicefrac{1}{2}$ to obtain
\begin{equation*}
\left[
\begin{array}{ccc|c}
1 & 1 & 1 & 1 \\
1 & 1 & 3 & 5 \\
1 & 4 & 1 & 10
\end{array}
\right] .
\end{equation*}
Now subtract the first row from the second and third row.
\begin{equation*}
\left[
\begin{array}{ccc|c}
1 & 1 & 1 & 1 \\
0 & 0 & 2 & 4 \\
0 & 3 & 0 & 9
\end{array}
\right]
\end{equation*}
Multiply the last row by $\nicefrac{1}{3}$ and the second row by $\nicefrac{1}{2}$.
\begin{equation*}
\left[
\begin{array}{ccc|c}
1 & 1 & 1 & 1 \\
0 & 0 & 1 & 2 \\
0 & 1 & 0 & 3
\end{array}
\right]
\end{equation*}
Swap rows 2 and 3.
\begin{equation*}
\left[
\begin{array}{ccc|c}
1 & 1 & 1 & 1 \\
0 & 1 & 0 & 3 \\
0 & 0 & 1 & 2
\end{array}
\right]
\end{equation*}
Subtract the last row from the first, then subtract the second row
from the first.
\begin{equation*}
\left[
\begin{array}{ccc|c}
1 & 0 & 0 & -4 \\
0 & 1 & 0 & 3 \\
0 & 0 & 1 & 2
\end{array}
\right]
\end{equation*}
If we think about what equations this augmented matrix represents, we see that
$x_1 = -4$, $x_2 = 3$, and $x_3 = 2$.  We try this solution in the original
system and, voil\`a, it works!

\begin{exercise}
Check that the solution above really solves the given equations.
\end{exercise}

We write this equation in matrix notation as
\begin{equation*}
A \vec{x} = \vec{b} ,
\end{equation*}
where $A$ is the matrix
$\left[ \begin{smallmatrix}
2 & 2 & 2 \\
1 & 1 & 3 \\
1 & 4 & 1 
\end{smallmatrix} \right]$ and $\vec{b}$ is the vector
$\left[ \begin{smallmatrix}
2 \\
5 \\
10
\end{smallmatrix} \right]$.  The solution can also be computed via the
inverse,
\begin{equation*}
\vec{x} = A^{-1} A \vec{x} = A^{-1} \vec{b} .
\end{equation*}

\medskip

It is
possible that the solution is not unique, or that no solution exists.
It is easy to tell if a solution does not exist.  If during the row
reduction you come up with a row where all the entries except the last one
are zero (the last entry in a row corresponds to the right hand side of the
equation) the system is \emph{inconsistent\index{inconsistent system}} and
has no solution.  For
example, for a system of 3 equations and 3 unknowns, if you find a row
such as $[\,0 \quad 0 \quad 0 ~\,|\,~ 1\,]$ in the augmented matrix,
you know the system is inconsistent.  That row corresponds to $0=1$.

\medskip

You generally try to use row operations until the following conditions
are satisfied.  The first nonzero entry in each row is called the
\emph{\myindex{leading entry}}.
\begin{enumerate}[(i)]
\item There is only one leading entry in each column.
\item All the entries above and below a leading entry are zero.
\item All leading entries are 1.
\end{enumerate}
Such a matrix is said to be in
\emph{\myindex{reduced row echelon form}}.  The variables
corresponding to columns with no leading entries are said to be
\emph{free variables\index{free variable}}.
Free variables mean that we can pick those variables
to be anything we want and then solve for the rest of the unknowns.

\begin{example}
The following augmented matrix is in reduced row echelon form.
\begin{equation*}
\left[
\begin{array}{ccc|c}
1 & 2 & 0 & 3 \\
0 & 0 & 1 & 1 \\
0 & 0 & 0 & 0
\end{array}
\right]
\end{equation*}
Suppose
the variables are $x_1$, $x_2$, and $x_3$.  Then $x_2$ is the
free variable, $x_1 = 3 - 2x_2$, and $x_3 = 1$.

\medskip

On the other hand if during the row reduction process you come up with the
matrix
\begin{equation*}
\left[
\begin{array}{ccc|c}
1 & 2 & 13 & 3 \\
0 & 0 & 1 & 1 \\
0 & 0 & 0 & 3
\end{array}
\right]
,
\end{equation*}
there is no need to go further.  The last row corresponds to
the equation $0 x_1 + 0 x_2 + 0 x_3 = 3$, which is preposterous.  Hence, no
solution exists.
\end{example}

\subsection{Computing the inverse}

If the matrix $A$ is square and there exists a unique solution
$\vec{x}$ to $A \vec{x} = \vec{b}$ for any $\vec{b}$ (there are no free
variables), then $A$ is invertible.
Multiplying both sides by $A^{-1}$, you can see that $\vec{x} =
A^{-1} \vec{b}$.  So it is useful to compute the inverse if you want to
solve the equation for many different right hand sides $\vec{b}$.

We have a formula for
the $2 \times 2$ inverse, but it is also not hard
to compute inverses of larger matrices.
While we will not have too much occasion to compute inverses for larger
matrices than $2 \times 2$ by hand, let us touch on how to do it.
Finding the inverse of $A$ is actually just solving a bunch of linear
equations.  If we can solve $A \vec{x}_k = \vec{e}_k$ where $\vec{e}_k$ is
the vector with all zeros except a 1 at the $k^{\text{th}}$ position, then
the inverse is the matrix with the columns $\vec{x}_k$ for $k=1,2,\ldots,n$
(exercise: why?).  Therefore, to find the inverse we write a larger $n
\times 2n$ augmented matrix $[ \,A ~|~ I\, ]$, where $I$ is the identity
matrix.
We then perform row reduction.
The reduced row echelon form of $[ \,A ~|~ I\, ]$ 
will be of the form $[ \,I ~|~ A^{-1}\, ]$ if and only if
$A$ is invertible.  We then just read off the inverse $A^{-1}$.

\subsection{Exercises}

FIXME

%%%%%%%%%%%%%%%%%%%%%%%%%%%%%%%%%%%%%%%%%%%%%%%%%%%%%%%%%%%%%%%%%%%%%%%%%%%%%%

\sectionnewpage
\section{Determinant}
\label{det:section}

\sectionnotes{? lectures}

FIXME: below is just a dump of the stuff from chapter 3

For square matrices we define a useful quantity called the
\emph{\myindex{determinant}}.  We define
the determinant of a $1 \times 1$ matrix as the value of its only entry.
For a $2 \times 2$ matrix we define
\begin{equation*}
\det \left(
\begin{bmatrix}
a & b \\
c & d
\end{bmatrix}
\right)
\overset{\text{def}}{=}
ad-bc .
\end{equation*}

Before trying to define the
determinant for larger matrices, let us note
the meaning of the determinant.  Consider an $n \times n$ matrix
as a mapping of the $n$ dimensional euclidean space ${\mathbb{R}}^n$ to 
itself, where $\vec{x}$ gets sent to $A \vec{x}$.
In particular, a $2 \times 2$ matrix $A$ is a mapping of
the plane to itself.  The
determinant of 
$A$ is the factor by which the area of objects changes. 
If we take the unit square (square of side 1) in the plane, then
$A$ takes the square to a parallelogram of area $\lvert\det(A)\rvert$.  The sign
of $\det(A)$ denotes changing of orientation (negative if the axes get flipped).  For
example, let
\begin{equation*}
A =
\begin{bmatrix}
1 & 1 \\
-1 & 1
\end{bmatrix} .
\end{equation*}
Then $\det(A) = 1+1 = 2$.  Let us see where the (unit) square with vertices
$(0,0)$, $(1,0)$, $(0,1)$, and $(1,1)$ gets sent.  Clearly $(0,0)$ gets sent
to $(0,0)$.  
\begin{equation*}
\begin{bmatrix}
1 & 1 \\
-1 & 1
\end{bmatrix}
\begin{bmatrix}
1 \\ 0
\end{bmatrix} =
\begin{bmatrix}
1 \\
-1 
\end{bmatrix}
,
\qquad
\begin{bmatrix}
1 & 1 \\
-1 & 1
\end{bmatrix}
\begin{bmatrix}
0 \\ 1
\end{bmatrix} =
\begin{bmatrix}
1 \\
1 
\end{bmatrix}
,
\qquad
\begin{bmatrix}
1 & 1 \\
-1 & 1
\end{bmatrix}
\begin{bmatrix}
1 \\ 1
\end{bmatrix} =
\begin{bmatrix}
2 \\
0 
\end{bmatrix}
.
\end{equation*}
The image of the square is another square with vertices $(0,0)$, $(1,-1)$,
$(1,1)$, and $(2,0)$.  The
image square has
a side of length $\sqrt{2}$ and is therefore of area 2.

If you think back to high school geometry, you may have seen a formula for
computing the area of a \myindex{parallelogram}
with vertices $(0,0)$, $(a,c)$, $(b,d)$
and $(a+b,c+d)$.  And it is precisely
\begin{equation*}
\left\lvert \, \det \left(
\begin{bmatrix} a & b \\ c & d \end{bmatrix}
\right) \, \right\lvert.
\end{equation*}
The vertical lines above mean absolute value.
The matrix $\left[ \begin{smallmatrix} a & b \\ c & d \end{smallmatrix}
\right]$
carries the unit square to the given parallelogram.

\medskip

Let us look at the determinant for larger matrices.  We define $A_{ij}$ as
the matrix $A$ with the $i^{\text{th}}$ row and the $j^{\text{th}}$ column
deleted.  To compute the determinant of a matrix, pick one row, say the
$i^{\text{th}}$ row and compute:
\begin{equation*}
\boxed{~~
\det (A) =
\sum_{j=1}^n
{(-1)}^{i+j}
a_{ij} \det (A_{ij}) .
~~}
\end{equation*}
For the first row we get
\begin{equation*}
\det (A) =
a_{11} \det (A_{11}) - 
a_{12} \det (A_{12}) + 
a_{13} \det (A_{13}) - 
\cdots
\begin{cases}
+ a_{1n} \det (A_{1n}) & \text{if } n \text{ is odd,} \\
- a_{1n} \det (A_{1n}) & \text{if } n \text{ even.}
\end{cases}
\end{equation*}
We alternately add and subtract the determinants of the submatrices
$A_{ij}$ multiplied by $a_{ij}$ for a fixed $i$ and all $j$.
For a $3 \times 3$ matrix,
picking the first row, we get $\det (A) = a_{11} \det (A_{11}) -
a_{12} \det (A_{12}) + a_{13} \det (A_{13})$.  For example,
\begin{equation*}
\begin{split}
\det \left(
\begin{bmatrix}
1 & 2 & 3 \\
4 & 5 & 6 \\
7 & 8 & 9
\end{bmatrix}
\right)
& =
1 \cdot
\det \left(
\begin{bmatrix}
5 & 6 \\
8 & 9
\end{bmatrix}
\right)
-
2 \cdot
\det \left(
\begin{bmatrix}
4 & 6 \\
7 & 9
\end{bmatrix}
\right)
+
3 \cdot
\det \left(
\begin{bmatrix}
4 & 5 \\
7 & 8
\end{bmatrix}
\right) \\
& =
1 (5 \cdot 9 - 6 \cdot 8)
-
2 (4 \cdot 9 - 6 \cdot 7)
+
3 (4 \cdot 8 - 5 \cdot 7)
= 0 .
\end{split}
\end{equation*}

The numbers ${(-1)}^{i+j}\det(A_{ij})$ are called
\emph{cofactors\index{cofactor}}
of the matrix and
this way of computing the determinant is called the
\emph{\myindex{cofactor expansion}}.
No matter which row you pick, you always get the same number.
It is also possible to compute the
determinant by expanding
along columns (picking a column instead of a row above).
It is true that $\det(A) = \det(A^T)$.

A common notation for the determinant is a pair of vertical
lines:
\begin{equation*}
\begin{vmatrix}
a & b \\
c & d
\end{vmatrix}
=
\det \left(
\begin{bmatrix}
a & b \\
c & d
\end{bmatrix}
\right) .
\end{equation*}
I personally find this notation confusing as vertical lines usually
mean a positive quantity, while determinants can be negative.  Also
think about how to write the absolute value of a determinant.  I will not
use this notation in this book.

\medskip

Think of the determinants telling you the scaling of a mapping.  
If $B$ doubles the sizes of geometric objects and $A$ tripples them,
then $AB$ (which applies $B$ to an object and then $A$) should make size
go up by a factor of $6$.  This is true in general:
\begin{equation*}
\det(AB) = \det(A)\det(B) .
\end{equation*}
This property is one of the most useful, and it is employed often to 
actually compute determinants.  A particularly interesting consequence is to
note what it means for existence of inverses.
Take $A$ and $B$ to be inverses, that is $AB=I$.  Then
\begin{equation*}
\det(A)\det(B) = \det(AB) = \det(I) = 1 .
\end{equation*}
Neither $\det(A)$ nor $\det(B)$ can be zero.
Let us state this as a theorem
as it will be very important in the context of this course.

\begin{theorem}
An $n \times n$ matrix $A$ is invertible if and only if $\det (A) \not= 0$.
\end{theorem}

In fact, $\det(A^{-1}) \det(A) = 1$ says that $\det(A^{-1}) =
\frac{1}{\det(A)}$.  So we even know what the determinant of $A^{-1}$ is
before we know how to compute $A^{-1}$.

There is a simple formula for the inverse of a $2 \times 2$ matrix
\begin{equation*}
\begin{bmatrix}
a & b \\
c & d
\end{bmatrix}^{-1}
=
\frac{1}{ad-bc}
\begin{bmatrix}
d & -b \\
-c & a
\end{bmatrix} .
\end{equation*}
Notice the determinant of the matrix
$[\begin{smallmatrix}a&b\\c&d\end{smallmatrix}]$
in the denominator of the fraction.
The formula only works if the determinant is nonzero, otherwise we are
dividing by zero.

FIXME


FIXME

\subsection{Exercises}

FIXME
